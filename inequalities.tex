\chapter{Inequalities}
Ineuqalities come up in different branches of mathematics; for example in algebra, geometry and trigonometry. They are very
useful in establishing many relations among various quantities. Certain inequalities are very useful in studying properties of
many common expressions which lead to interesting observations. In this chapter we will only study algebraic inequalitites. The
problems given are quite basic and simple. We start with some useful theorems for these inequalities.

There are some facts which are the very important for proving inequalities. Some of them are as follows:

\begin{enumerate}
\item If $x\geq y$ and $y\geq z$ then $x\geq z$, for any $x, y, z\in\mathbb{R}$.
\item If $x\geq a$ and $y\geq b$ then $x + a\geq y + b$, for any $x, y, a, b\in\mathbb{R}$.
\item If $x\geq y$ then $x + z\geq y + z$, for any $x, y, z\in\mathbb{R}$.
\item If $x\geq y$ and $a\geq b$ then $xa\geq yb$, for any $x, y\in\mathbb{R}^+$ or $a, b\in\mathbb{R}^+$.
\item If $x\in\mathbb{R}$ then $x^2\geq 0$, with equality holding if and only if $x = 0$. More generally for $a_i\in\mathbb{R}^+$
  and $x_i\in\mathbb{R}, i = 1, 2, \ldots, n$ holds $a_ix_i^2 + a_2x_2^2 + \cdots + a_nx_n^2\geq 0$, with equality holding if and
  only if $x_1 = x_2 = \cdots = x_n = 0$.
\end{enumerate}

\section{Strum's Method}
Strum's method is given by the German mathematician {\sl Friedrich Otto Rudolf Sturm}. Sturm's method helps prove a large number of
different inequalities under certain conditions along with various other applications.

\begin{theorem}
\label{th:strum:1}
Prove that if the product of positive numbers $x_1, x_2, \cdots, x_n (n\geq 2)$ is euqal to $1$, then $x_1 + x_2 + \cdots + x_n\geq n$.
\end{theorem}
\begin{proof}
  If $x_1 = \cdots = x_n$, then $x_1 + \cdots + x_n = n$. So we see that the statement is true if all the numbers are equal and are
  unity. Now we consider the case when at least two numbers are different such that one is greater than $1$ and the other one is
  smaller. Let us assume that these are $x_1$ and $x_2$ which does not cause loss of generality, and that $x_1 < 1 < x_2$. Note
  that $x_1 + x_2 > 1 + x_1x_2 [\because (1 - x_1)(x2 - 1) > 0]$. If given numbers are substitued by $1, x_1x_2, x_3, \ldots, x_n$,
  then the product is equal to $1$ and $1 + x_1x_2 + x_3 + \cdots + x_n < x_1 + x_2 + \cdots + x_n$. Repeating this we will find $n
  - 1$ numbers equal to $1$ and the $n$th number equal to $x_1x_2\ldots x_n$. Thus, $x_1 + x_2 + \cdots + x_n < 1$. We see that
  equality holds if and only if $x_1 = x_2 = \cdots = x_n = 1$
\end{proof}

\begin{theorem}
\label{th:strum:2}
  Prove that if the sum of the numbers $x_1, x_2, \ldots, x_n (n\geq 2)$ is equal to $1$, then prove that $x_1^2 + x_2^2 + \cdots +
  x_n^2 \geq \frac{1}{n}$.
\end{theorem}
\begin{proof}
  If $x_1 = x_2 = \ldots = x_n = \frac{1}{n}$ then $x_1^2 + x_2^2 + \cdots + x_n^2 = \frac{1}{n}$. Like previous theorem we
  consider two numbers $x_1$ and $x_2$ such that one of them is greater than $\frac{1}{n}$ while the other is smaller than
  $\frac{1}{n}$. Assume that these two numbers are $x_1$ and $x_2$, which does not cause loss of generality, and that $x_1 <
  \frac{1}{n}$ and $x_2 > \frac{1}{n}$. So we obtain a sequence of numbers $\frac{1}{n}, x_1 + x_2 - \frac{1}{n}, x_3, \ldots, x_n$
  suhc that their sum remains equal to $1$. We can easily prove that $x_1^2 + x_2^2 > \frac{1}{n^2} + \left(x_1 + x_2 -
  \frac{1}{n}\right)^2$, and hence $$x_1^2 + x_2^2 + \cdots + x_n^2 > \frac{1}{n^2} + \left(x_1 + x_2 - \frac{1}{n}\right)^2 +
  x_3^2 + \cdots + x_n^2.$$

  Repeating this we obtain a sequence in which all terms will be equal to $\frac{1}{n}$, and sum of their square is less than the
  sum of squares of numbers $x_1, x_2, \ldots, x_n$ i.e. $x_1^2 + x_2^2 + \cdots + x_n^2 > \frac{1}{n^2} +
  \text{~to~}n\text{~times}$. From this it follows that equality holds if and only if $x_1 = x_2 = \cdots = x_n$.
\end{proof}

\section{A.M., G.M., H.M. and Q.M.}
\begin{theorem}
  \textbf{\rm\bf (A.M.-- G.M. -- H.M. -- Q.M. Inequality)} Let $x_1, x_2, \ldots, x_n$ be positive real numbers, then
  \begin{equation}
  \frac{n}{\frac{1}{x_1} + \frac{1}{x_2} + \cdots + \frac{1}{x_n}} \leq \sqrt[n]{x_1x_2 \ldots x_n}\leq \frac{x_1 + x_2 + \cdots
    + x_n}{n}\leq \sqrt{\frac{x_1^2 + x_2^2 + \cdots + x_n^2}{n}}.
  \end{equation}
\end{theorem}

\begin{proof}
  Consider the numbers $\frac{x_1}{\sqrt[n]{x_1x_2\cdots x_n}}, \frac{x_2}{\sqrt[n]{x_1x_2\cdots x_n}}, \cdots,
  \frac{x_n}{\sqrt[n]{x_1x_2\cdots x_n}}$, we see that product is equal to $1$. From theorem~\ref{th:strum:1}, we have that
  $$\frac{x_1}{\sqrt[n]{x_1x_2\cdots x_n}} + \frac{x_2}{\sqrt[n]{x_1x_2\cdots x_n}} + \cdots + \frac{x_n}{\sqrt[n]{x_1x_2\cdots
      x_n}}\geq n \Rightarrow \frac{x_1 + x_2 + \cdots + x_n}{n}\geq \sqrt[n]{x_1x_2\cdots x_n}.$$
  The above inequality is also known as Cauchy's inequality.

  In the above inequality, if we substitute $x_i = \frac{1}{x_i}$, then $$\frac{n}{\frac{1}{x_1} + \frac{1}{x_2} + \cdots +
    \frac{1}{x_n}} \leq \sqrt[n]{x_1x_2 \cdots x_n}.$$

  Consider the numbers $\frac{x_1}{x_1 + x_2 + \cdots + x_n}, \frac{x_2}{x_1 + x_2 + \cdots + x_n}, \cdots, \frac{x_n}{x_1 + x_2 +
    \cdots + x_n}$, and note that their sum is equal to $1$. According to theorem~\ref{th:strum:2}, we have $$\left(\frac{x_1}{x_1 +
    x_2 + \cdots + x_n}\right)^2 + \left(\frac{x_2}{x_1 + x_2 + \cdots + x_n}\right)^2 + \cdots + \left(\frac{x_n}{x_1 + x_2 +
    \cdots + x_n}\right)^2 \geq \frac{1}{n}$$
  $$\Rightarrow \frac{x_1^2 + x_2^2 + \cdots + x_n^2}{n}\geq \left(\frac{x_1 + x_2 + \cdots + x_n}{n}\right)^2.$$

  Hence, all the inequalities have been proven.
\end{proof}

\section{Cauchy–Bunyakovsky–Schwarz Inequality}
\begin{theorem}
  {\rm \bf (Cauchy–Bunyakovsky–Schwarz Inequality)} Let $a_1, a_2, \ldots, a_n,$ $b_1, b_2, \ldots, b_n \in R$. Then
  \begin{equation}
    (a_1^2 + a_2^2 + \cdots + a_n^2)(b_1^2 + b_2^2 + \cdots + b_n)^2 \geq (a_1b_1 + a_2b_2 + \cdots a_nb_n)^2.
  \end{equation}
\end{theorem}

\begin{proof}
  Let $x_k = \sqrt{(a_1^2 + a_2^2 + \cdots + a_k^2)(b_1^2 + b_2^2 + \cdots + b_k^2)}$, where $k = 1, 2, \ldots, n$.
  In this case,

  $$x_{k + 1} = \sqrt{(a_1^2 + a_2^2 + \cdots + a_k^2 + a_{k + 1}^2)(b_1^2 + b_2^2 + \cdots + b_k^2 + b_{k + 1}^2)}$$
  $$\sqrt{\left[\left(\sqrt{a_1^2 + a_2^2 + \cdots + a_k^2}\right)^2 + a_{k + 1}^2\right]\left[\left(\sqrt{b_1^2 + b_2^2 + \cdots +
        b_k}\right)^2 + b_{k + 1}^2\right]}$$
  $$\geq \sqrt{\left(\sqrt{a_1^2 + a_2^2 + \cdots + a_k^2}.\sqrt{b_1^2 + b_2^2 + \cdots + b_k^2} + a_{k + 1}.b_{k + 1}\right)^2} =
  x_k + a_{k + 1}b_{k + 1}$$
  {\it Alternative Proof.}
  $$(a_1^2 + a_2^2 + \cdots + a_n^2)(b_1^2 + b_2^2 + \cdots + b_n)^2 - (a_1b_1 + a_2b_2 + \cdots a_nb_n)^2 =
  \sum_{i,j=1\\i\geq j}^n(a_ib_j - b_ja_i)^2\geq 0.$$
\end{proof}

\subsection{Titu's Lemma}
\begin{lemma}
  Let $a_1, a_2, \ldots, a_n, b_1, b_2, \ldots, b_n$ be positive real numbers then
  \begin{equation}
  \frac{a_1^2}{b_1} + \frac{a_2^2}{b_2} + \ldots
    + \frac{a_n^2}{b_n} \geq \frac{(a_1 + a_2 + \cdots + a_n)^2}{b_1 + b_2 + \cdots + b_n}
  \end{equation}
\end{lemma}

\begin{proof}
  This is a direct consequence of \textit{Cauchy–Bunyakovsky–Schwarz Inequality}. It is obtained by substituting $a_i =
  \frac{x_i}{\sqrt{y_i}}$ and $b_i = \sqrt{y_i}$ into Cauchy–Bunyakovsky–Schwarz Inequality. Equality holds if and only if $a_i =
  kb_i$ for a non-zero real constant $k$.
\end{proof}

\section{Chebyshev's Inequality}
\begin{theorem}
  Let $a_1, a_2, \ldots, a_n, b_1, b_2, \ldots, b_n$ be real numbers such that  $a_1\leq _2\leq a_2\leq \cdots\leq a_n$ and $b_1\leq
  b_2\leq\cdots\leq b_n$ or $a_1\geq _2\geq a_2\geq \cdots\geq a_n$ and $b_1\geq b_2\geq\cdots\geq b_n$, then the inequality
  \begin{equation}
    \left(\frac{a_1 + a_2 + \cdots + a_n}{n}\right)\left(\frac{b_1 + b_2 + \cdots + b_n}{n}\right)\leq \frac{a_1b_1 + a_2b_2 +
      \cdots + a_nb_n}{n}
  \end{equation}
  holds. The inequality is strict unless at least one of the sequences is a constant sequence.
\end{theorem}

\begin{proof}
  We have $$\sum_{i=1}^n\sum_{j=1}^n(a_ib_i - a_jb_j) = \sum_{i=1}^n\left(na_ib_i - a_i\sum_{j=1}^nb_j\right) =
  n\sum_{i=1}^na_ib_i - \sum_{i=1}^na_i\sum_{j=1}^nb_j$$
  Simiarly
  $$\sum_{i=1}^n\sum_{j=1}^n\left(a_jb_j - a_jb_i\right) = n\sum_{j=1}^na_jb_j - \sum_{j=1}^na_j\sum_{i=1}^nb_i$$
  From these two equations, we get
  $$n\sum_{j=1}^na_jb_j - \sum_{j=1}^na_j\sum_{i=1}^nb_i = \frac{1}{2}\left[\sum_{i=1}^n\sum_{j=1}^n\left(a_ib_i - a_ib_j + a_jb_j
    - a_jb_i\right)\right]$$
  $$= \frac{1}{2}\sum_{i=1}^n\sum_{j=1}^n(a_i - a_j)(b_i - b_j)$$
  Since both the sequences are either decreasing or increasing, we will have $(a_i - a_j)(b_i - b_j)\geq 0$. Thus, we have
  $$n\sum_{j=1}^na_jb_j - \sum_{j=1}^na_j\sum_{i=1}^nb_i \geq 0.$$
  Here equality holds if and only if for each of the indexes $i, j$ either $a_i = a_j$ or $b_i = b_j$.
\end{proof}

\begin{remark}
  If the the order of sequences $\langle a_i\rangle$ and $\langle b_i\rangle$ in the orevious theorem are reverses then the
  inequlaity reverses as well.

  {\rm The proof is similar to the proof of the theorem.}
\end{remark}

\begin{remark}
  Chebyshev's inequality can be generalized to three or more sets of real numbers, with the constraint that sets are in increasing
  or decreasing order.
\end{remark}

\begin{remark}
  If the two sequeqnces are non-increasing or non-decreasing, and let $p_1, p_2, \ldots, p_b$ be a sequence of non=negative real
  numbers such that $\sum_{i=1}^np_i$ is positive. Then the following inequality holds
  $$\left(\frac{\sum_{i=1}^np_ia_ib_i}{\sum_{i=1}^np_i}\right)\geq\left(\frac{\sum_{i=1}^np_ia_i}{\sum_{i=1}^np_i}\right)\left(\frac{\sum_{i=1}^np_ib_i}{\sum_{i=1}^np_i}\right).$$

  {\rm The proof is similar to the theorem. This is called Chebyshev's inequality with weights.}
\end{remark}

\section{Sur\'anyi's Inequality}
\begin{theorem}
  Let $a_1, a_2, \ldots, a_n$ be non-negative real numbers, and let $n\in P$. Then
  \begin{equation}
    (n - 1)(a_1^n + a_2^n + \cdots + a_n^n) + na_1a_2 \cdots a_n\geq (a_1 + a_2 + \cdots + a_n)(a_1^{n-1} + a_2^{n - 1} + \cdots + a_n^{n-1}).
  \end{equation}
\end{theorem}

\begin{proof}
  We will prove this by mathematical induction. Due to symmetry and homegeneity of the inequality we may assume $a_1\geq a_2\geq
  \cdots \geq a_n$ and $a_1 + a_2 + \cdots + a_n = 1$. For $n =1$ equality occurs. Let us assume that for $n = 1$ the inequality
  holds i.e. $$(k - 1)(a_1^k + a_2^k + \cdots + a_k^k) + ka_1a_2\cdots a_k\geq a_1^{k-1} + a_2^{k - 1} + \cdots + a_k^{k - 1}.$$
  We need to prove that:
  $$k\sum_{i=1}^{k+1}a_i^{k+1} + (k+1)\prod_{i=1}^{k+1}a_i - (1 + a_{k+1})\sum_{i=1}^{k+1}a_i^k\geq 0.$$
  Hence
  $$ka_{k+1}\prod_{i=1}^ka_i\geq a_{k+1}\sum_{i=1}^ka_i^{k-1} - (k-1)a_{k+1}\sum_{i=1}^ka_i^k.$$
  Using this last inequality, it remains to prove that:
  $$\left(k\sum_{i=1}^{k+1}a_i^{k+1} - \sum_{i=1}^ka_i^k\right) - a_{k+1}\left(k\sum_{i=1}^ka_i^k - \sum_{i=1}^ka_i^{k - 1}\right)
  + a_{k+1}\left(\prod_{i=1}^ka_i + (k - 1)a_{k+1}^k - a_{k+1}^{k-1}\right)\geq 0.$$
  We have $$\prod_{i=1}^k a_i + (k - 1)a_{k+1}^k - a_{k+1}^{k-1} = \prod_{i=1}^k(a_i - a_{k +1} + a_{k + 1}) + (k - 1)a_{k+1}^k - a_{k+1}^{k
    - 1}$$
  $$\geq a_{k+1}^k + a_{k+1}^{k-1}\sum_{i=1}^k(a_i - a_{k+1}) + (k - 1)a_{k+1}^k - a_{k+1}^{k-1} = 0.$$
  Also
  $$\left(k\sum_{i=1}^{k+1}a_i^{k+1} - \sum_{i=1}^ka_i^k\right) - a_{k+1}\left(k\sum_{i=1}^ka_i^k - \sum_{i=1}^ka_i^{k - 1}\right)
  \geq 0$$
  $$\Rightarrow k\sum_{i=1}^ka_i^{k+1} - \sum_{i=1}^ka_i^k\geq a_{k+1}\left(k\sum_{i=1}^ka_i^k - \sum_{i=1}^ka_i^{k - 1}\right)$$
  By Chebyshev's inequality, we have
  $$k\sum_{i=1}^ka_i^k\geq \sum_{i=1}^ka_i\sum_{i=1}^ka_i^{k-1} = \sum_{i=1}^ka_i^{k-1}$$
  $$\Rightarrow k\sum_{i=1}^ka_i^k - \sum_{i=1}^ka_i^{k-1}\geq 0.$$
  and since $a_1 + a_2 + \cdots + a_{k + 1} = 1$, by the assumption $a_1\geq a_2 \geq\cdots\geq a_{k +1}$, we deduce that $$a_{k +
    1}\leq \frac{1}{k}$$
  So it is enough to prove that
  $$k\sum_{i=1}^ka_i^{k+1} - \sum_{i=1}^k a_i^k\geq \frac{1}{k}\left(k\sum_{i=1}^ka_i^k - \sum_{i=1}^ka_i^{k - 1}\right).$$
  which is equivalent to
  $$k\sum_{i=1}^ka_i^{k+1} + \frac{1}{k}\sum_{i=1}^ka_i^{k-1}\geq 2\sum_{i=1}^ka_i^k$$
  Since AM $\geq$ GM we have that
  $$ka_i^{k+1} + \frac{1}{k}a_i^{k - 1}\geq 2a_i^k~\forall~i$$
  Adding this inequality for $i=1, 2, \ldots, k$ we obtain the required inequality.
\end{proof}

\section{Rearrangement Inequality}
\begin{theorem}
  Let $a_1\leq a_2\leq\cdots\leq a_n$ and $b_1\leq b_2\leq\cdots\leq b_n$ (or $a_1\geq a_2\geq\cdots\geq a_n$ and $b_1\geq
  b_2\geq\cdots\geq b_n$) be real numbers. If $a_1', a_2', \ldots, a_n'$ is any permutation of $a_1, a_2, \ldots, a_n$ then the
  equality
  \begin{equation}
    \sum_{i=1}^na_ib_{n+1-i}\leq \sum_{i=1}^na_i;b_i\leq\sum_{i=1}^na_ib_i,
  \end{equation}
  holds. Thus the sum $\displaystyle\sum_{i=1}^na_ib_i$ is maximum when the two sequences $\langle a_i\rangle$ and $\langle
  b_i\rangle$ are oredered similarly. And the sum is minimum when these are ordered in opposite manner.
\end{theorem}

\begin{proof}
  We start by assuming that both $a_i$'s and $b_i$'s are non-decreasing. Suppose $\langle a_i'\rangle\neq\langle a_i\rangle$. Let
  $r$ be the largest index such that $a_r'\neq a_r$ i.e. $a_r'\neq a_r$ and $a_i'=a_i$ for $r<i\leq n$. This implies that $a_r'$ is
  from the set $\{a_1, a_2, \ldots, a_{r - 1}\}$ and $a_r'<a_r$. Further this also shows that $a_1', a_2', \ldots, a_r'$ is a
  permutation of $a_1, a_2, \ldots, a_r$. Thus we can find indices $k<r$ and $l<r$ such that $a_k' = a_r$ and $a_r' = a_l$. It
  follows that $$a_k' - a_r' = a_r - a_l\geq 0,~~b_r - b_k\geq 0$$
  We now interchange $a_r'$ and $a_k'$ to get a permutation of $a_1'', a_2'', \ldots, a_n''$ of $a_1', a_2', \ldots, a_n'$; thus
  $$\begin{cases}a_i''=a_i', & \text{if}i\neq r, k\\a_r''=a_k'=a_r,\\a_k''=a_r'=a_l.\end{cases}$$
  Consider the sums
  $$S'' = a_1''b_1 + a_2''b_2 + \cdots + a_n''b_n, ~~S' = a_1'b_1 + a_2'b_2 + \cdots + a_n'b_n,$$
  and the difference $S'' - S':$
  $$\begin{aligned}S'' - S' & = \sum_{i=1}^n(a_i'' - a_i')b_i\\& = (a_k'' - a_k') + (a_r'' - a_r')b_r\\&=(a_r' - a_k')b_k + (a_k' -
    a_r')b_r\\& = (a_k' - a_r')(b_r - b_k).\end{aligned}$$
  $\because a_k' - a_r' \geq 0$ and $b_r - b_k \geq 0$, we can say that $S''\geq S'$. We observe that the permutations $a_1'', a_2'',
  \ldots, a_n''$ of $a_1', a_2', \ldots, a_n'$ has th eproperty that $a_i'' = a_i = a_i$ for $r< i\leq n$ and $a_r'' = a_k' =
  a_r$. Hence the permutation $\langle a_i''\rangle$ in place of $\langle a_i'\rangle$ may be considered and the steps can be
  continued like above. After at most $n - 1$ such steps, we will arrive at the original permutation $\langle a_i\rangle$ from
  $\langle a_i'\rangle$. At each step the corresponding sum has the same order as $a_i$'s i.e. non-decreasing. Thus,
  \begin{equation}
    \label{rearrangement:2}
    a_1'b_1 + a_2'b_2 + \cdots + a_n'b_n\leq a_1b_1 + a_2b_2 + \cdots + a_nb_n
  \end{equation}
  For the other part, let us put $c_i = a_{n + 1 - i}', d_i = -b_{n + 1 - i}$. Then $c_1, c_2, \ldots, c_n$ is a permutation of
  $a_1, a_2, \ldots, a_n$ and $d_1\leq d_2\leq \cdots\leq d_n$. Using the inequality (\ref{rearrangement:2}) for the sequences
  $\langle c_i\rangle$ and $\langle d_i\rangle$, we get
  $$c_1d_1 + c_2d_2 + \cdots + c_nd_n\leq a_1d_2 + a_2d_2 + \cdots + a_nd_n.$$
  Thus,
  $$-\sum_{i=1}^na_{n+1 - i}'b_{n+1-i}\leq -\sum_{n=1}^na_ib_{n+1-i}.$$
  Thus, \begin{equation}\label{rearrangement:3}a_1'b_1 + a_2'b_2 + \cdots + a_n'b_n\geq a_1b_1 + a_2b_{n - 1} + \cdots +
    a_nb_1,\end{equation}
  which is the other part of the inequality.

  For the equality, we consider pairs $k, l$ with $1\leq k < l\leq n$, either $a+k' = a_l'$ or $a_k'>a_l'$ and $b_k = b_l$, then
  the equality holds for (\ref{rearrangement:2}). For (\ref{rearrangement:3}), for each $k, l$ with $1\leq k < l \leq n$, either
  $a_{n + 1 - k}' \geq a_{n + 1 - l}'$ and $b_{n + 1 - k} = b_{n  + 1 - l}$.
\end{proof}

\begin{corollary}
  Let $\alpha_1, \alpha_2, \ldots, \alpha_n$ be real numbers and $\beta_1, \beta_2, \ldots, \beta_n$ be a permutation of $\alpha_1,
  \alpha_2, \ldots, \alpha_n$. Then $$\sum_{i=1}^n\alpha_i\beta_1\leq \sum_{i=1}^n\alpha_i^2.$$ The equality holds if and only if
  $\langle\alpha_i\rangle = \langle\beta_i\rangle$.
\end{corollary}

\begin{proof}
  Let $\alpha_1', \alpha_2', \ldots, \alpha_n'$ be a permutation of $\alpha_1, \alpha_2, \ldots, \alpha_n$ such that
  $\alpha_1'\leq\alpha_2'\leq\ldots\leq\alpha_n'$. Then we can find a bijections $\sigma$ of $\{1,2,\ldots,n\}$ onto itself such
  that $\alpha_i' = \alpha_{\sigma(i)}, 1\leq j\leq n$; i.e. $\sigma$ is a permutation on the set $\{1,2,\ldots, n\}$. Let
  $\beta_i' = \beta_{\sigma(i)}$. Then $\beta_1', \beta_2', \ldots, \beta_n'$ is a permutation of
  $\alpha_1'\leq\alpha_2'\leq\ldots\leq\alpha_n'$. Applying the rearrangement inequality to
  $\alpha_1'\leq\alpha_2'\leq\ldots\leq\alpha_n'$ and $\beta_1', \beta_2', \ldots, \beta_n'$, we get
  $$\sum_{i=1}^n\alpha_i'\beta_i'\leq\sum_{i=1}^n(\alpha_i')^2 = \sum_{i=1}^n\alpha_i^2.$$
  We also have $$\sum_{i=1}^n\alpha_i'\beta_i' = \sum_{i=1}^n\alpha_{\sigma(i)}\beta_{\sigma(i)} = \sum_{i=1}^n\alpha_i\beta_i,$$
  because $\sigma$ is a bijection on $\{1,2,\ldots,n\}$. Thus,
  $$\sum_{i=1}^n\alpha_i\beta_i\leq\sum_{i=1}^n\alpha_i^2.$$

  Say that equality holds and $\langle\alpha_i\rangle\neq\langle\beta_i\rangle$. Then
  $\langle\alpha_i'\rangle\neq\langle\beta_i'\rangle$. Let $k$ be the largest index such that $\alpha_k'\neq\beta_k'$ for $k <
  i\neq n$. Let $m$ be the least integer such that $\alpha_k' = \beta_m'$. If $m>k$, then $\beta_m' = \alpha_k'$ and hence
  $\alpha_k' = \alpha_m'$. This implies that $\alpha_k' = \alpha_{k+1}' = \cdots = \alpha_m'$ and hence $\beta_{k+1}' = \cdots =
  \beta_m'$. We now have an $m_1 > m$ such that $\alpha_k' = \beta_{m_1}'$. Using $m_1$ as pivot, we get $\alpha_k' = \alpha_{k+1}'
  = \cdots = \alpha_m' = \cdots = \alpha_m'$ and $\beta_{k+1}' = \cdots = \beta_m' = \cdots = \beta_{m_1}'$. It can be concluded
  that $\alpha_k' = \beta_l'$ for some $l<k$, thus forcing $m < k$.

  Clearly $\beta_m'\neq\beta_k'$ by our choice of $k$. We know that equality holds if and only if for any two indexes $r\neq s$,
  either $\alpha_r' = \alpha_s'$ or $\beta_r' = \beta_s'$. Since $\beta_m'\neq\beta_k'$, we must have $\alpha_m' = \alpha_k'$. But
  then we have $\alpha_m' = \alpha_{m+1}' = \cdots = \alpha_k'$. From the minimality of $m$, we see that $k - m + 1$ equal elements
  $\alpha_m', \alpha_{m+1}', \ldots, \alpha_k'$ must be among $\beta_m', \beta_{m+1}', \ldots, \beta_n'$ and since
  $\beta_k'\neq\alpha_k'$, we must have $\alpha_k' = \beta_l'$ for some $l > k$. But then using $\beta_l' = \alpha_l'$, we have
  $$\alpha_m' = \alpha_{m+1}' = \cdots = \alpha_k' = \cdots = \alpha_l'.$$
  Thus the number of equal elements gets enlarged to $l - m + 1 > k -m + 1$. Since this process cannot be continues indefinitely,
  we conclude that $\langle\alpha_i'\rangle = \langle\beta_i'\rangle$ which will be followed
  by $\langle\alpha_i\rangle\neq\langle\beta_i\rangle$.
\end{proof}

\begin{corollary}
  Let $\alpha_1, \alpha_2, \ldots, \alpha_n$ be positive real numbers and let $\beta_1, \beta_2, \ldots, \beta_n$ be a permutation
  of $\alpha_1, \alpha_2, \ldots, \alpha_n$. Then $$\sum_{i=1}^n\frac{\beta_i}{\alpha_i}\geq n.$$
  Equality holds if and only if $\langle\alpha_i\rangle\neq\langle\beta_i\rangle$.
\end{corollary}
\begin{proof}
  Let $\alpha_1', \alpha_2', \ldots, \alpha_n'$ be a permutation of $\alpha_1, \alpha_2, \ldots, \alpha_n$ suhc that
  $\alpha_1'\leq\alpha_2'\leq\ldots\alpha_n'$. Like in previous corollary, we can find a permutation $\sigma$ of $\{1,2,\ldots,
  n\}$ such that $\alpha_i' = \alpha_{\sigma(i)}$ for $1\leq i\leq n$. We defien $\beta_i' = \beta_{\sigma(i)}$. Then
  $\langle\beta_i'\rangle$ is a permutation of $\langle\alpha_i'\rangle$. Using the rearrangement theorem, we get
  $$\sum_{i=1}^n\beta_i'\left(-\frac{1}{\alpha_i'}\right)\leq\sum_{i=1}^n\alpha_i'\left(-\frac{1}{\alpha_i'}\right) = -n.$$
  Thus, we have the desired inequality. Like previous case we camn derive the equality.
\end{proof}

\section{Young's Inequality}
\begin{theorem}
  If $p\in [1,\infty)$ and $q = p/(p - 1)$. $q\in [1, \infty]$ and $\dfrac{1}{p} + \frac{1}{q} = 1$. If $a, b > 0$,
    then
    \begin{equation}
      \frac{a^p}{p} + \frac{b^q}{q}\geq ab
    \end{equation}
\end{theorem}

\begin{proof}
  Taking $\log$ of L.H.S. $\log\left(\frac{a^p}{p} + \frac{b^q}{q}\right)$

  Notice that, since $\dfrac{1}{p} + \frac{1}{q} = 1$, so the L.H.S. is just a convext combination of $a^p$ and $b^q$. Since $\log
  x$ is a concave function, we have
  $$\log\left(\frac{a^p}{p} + \frac{b^q}{q}\right)\geq \dfrac{\log a^p}{p} + \dfrac{b^q}{q} = \log a + \log b = \log(ab).$$
  Hence, the inequality is proved(since $\log x$ is strictly increasing).

  {\it Alternative Proof.}

  Using generalized AM-GM inequality, $$\dfrac{x^p}{p} + \frac{b^q}{q}\geq \left[(x^p)^{1/p}(y^q)^{1/q}\right] = xy.$$
\end{proof}

\section{H\"{o}lder's Inequality}
\begin{theorem}
  Let $a_1, a_2, \ldots, a_n, b_1, b_2, \ldots, b_n$ be real numbers and $p, q$ be two positive real numbers such that $\frac{1}{p}
  + \frac{1}{q} = 1.$ (Such a pair of indices is called a pair of conjugate indices.) Then the inequality holds
  \begin{equation}
    \left|\sum_{i=1}^na_ib_i\right|\leq\left(\sum_{i=1}^n|a_i|^p\right)^{1/p}\left(\sum_{i=1}^n|b_i|^q\right)^{1/q}
  \end{equation}
holds. Equality holds if and only if $|a_i|^p = c|b_i|^q, 1\leq i\leq n$, for some real constant $c$.
\end{theorem}

\begin{proof}
  Following Young's inequality, consider $$x = \frac{|a_k|}{\left(\sum_{i=1}^n|a_i|^p\right)^{1/p}}, y =
  \frac{|b_k|}{\left(\sum_{i=1}^n|b_i|^q\right)^{1/q}}$$
  so we get $$\frac{|a_k|^p}{p\left(\sum_{i=1}^n|a_i|^p\right)} +
  \frac{|b_k|^q}{q\left(\sum_{i=1}^n|b_i|^q\right)}\geq
  \frac{|a_k||b_k|}{\left(\sum_{i=1}^n|a_i|^p\right)^{1/p}\left(\sum_{i=1}^n|b_i|^q\right)^{1/q}}$$
  Now summing over $k$, we obtain
  $$\frac{1}{p} + \frac{1}{q}\geq
  \frac{\sum_{i=1}^n|a_kb_k|}{\left(\sum_{i=1}^n|a_i|^p\right)^{1/p}\left(\sum_{i=1}^n|b_i|^q\right)^{1/q}}$$
  Thus, we have
  $$\sum_{i=1}^n|a_kb_k|\leq \left(\sum_{i=1}^n|a_i|^p\right)^{1/p}\left(\sum_{i=1}^n|b_i|^q\right)^{1/q}.$$
  It is now trivial to prove the condition for equality.
\end{proof}

\begin{remark}
  If we take $p = q = 3$, H\"{o}lder's inequality reduces to the Cauchy-Schwarz inequality.
\end{remark}
\begin{remark}
  If either of $p$ and $q$ is negativem H\"{o}lder's inequality is reversed.
\end{remark}
\begin{remark}
  H\"{o}lder's inequality can have a version with weights. In addition to what we have, we also consider consider weights $w_1,
  w_2, \ldots, w_n$ then following equality holds
  $$\sum_{i=1}^nw_i|a_ib_i|\leq\left(\sum_{i=1}^nw_i|a_i|^p\right)^{1/p}\left(\sum_{i=1}^nw_i|b_i|^q\right)^{1/q}$$
\end{remark}

Given below is generalized H\"{o}lder's inequaltiy and the proof is similar like above.
\begin{theorem}
  Let $a_{ij}, i=1, 2, \ldots, m;j = 1, 2, \ldots, n$, be positive humbers and $\alpha_1, \alpha_2, \ldots, \alpha_n$ be positive
  real numbers such that $\alpha_1 + \alpha_2 + \cdots + \alpha_n = 1$. Then
  \begin{equation}
    \sum_{i=1}^m\left(\prod_{j=1}^na_{ij}a_{ij}^{\alpha_j}\right)\leq \prod_{j=1}^n\left(\sum_{i=1}^ma_{ij}\right)^{\alpha_j}.
  \end{equation}
\end{theorem}

\section{Minkowski's Inequality}
\begin{theorem}
  Let $p\geq 1$ be a real number and $a_1, a_2, \ldots, a_n, b_1, b_2, \ldots, b_n$ be real numbers. Then
  \begin{equation}
    \label{eq:10.7}
    \left(\sum_{i=1}^n|ai + b_i|^p\right)^{1/p}\leq\left(\sum_{i=1}^n|a_i|^p\right)^{1/p} + \left(\sum_{i=1}^n|b_i|^p\right)^{1/p}
  \end{equation}
  Here equality holds if and only if $a_i=\lambda b_i$ for some constant $\lambda, 1\leq i\leq n$.
\end{theorem}

\begin{proof}
  We assume that $p > 1$, because the result is clear for $p = 1$. Observe the following:
  $$\sum_{i=1}^n|ai + b_i|^p = \sum_{i=1}^n|a_i + b_i|^{p - 1}|a_i + b_i|\leq \sum_{i=1}^n|a_i + b_i|^{p - 1}|a_i| +
  \sum_{i=1}^n|a_i + b_i|^{p - 1}|b_i|.$$
  Let $q$ be the conjugate index of $p$. Using H\"{o}lder's inequaity to each sum on the right hand side, we have
  $$\sum_{i=1}^n|a_i + b_i|^{p - 1}|a_i|\leq \left(\sum_{i=1}^n|a_i|^p\right)^{1/p}\left(\sum_{i=1}^n|a_i + b_i|^{(p -
    1)q}\right)^{1/q}.$$
  Since $p, q$ are conjugate indexes, we get $(p - 1)q = p$. It follows that
  $$\sum_{i=1}^n|a_i + b_i|^{p - 1}|a_i|\leq\left(\sum_{i=1}^n|a_i|^p\right)^{1/p}\left(\sum_{i=1}^n|a_i + b_i|^p\right)^{1/q}.$$
  Similarly,
  $$\sum_{i=1}^n|a_i + b_i|^{p - 1}|b_i|\leq\left(\sum_{i=1}^n|b_i|^p\right)^{1/p}\left(\sum_{i=1}^n|a_i + b_i|^p\right)^{1/q}.$$
  It now follows that
  $$\sum_{i=1}^n|a_i + b_i|^p\leq\left[\left(\sum_{i=1}^n|a_i|^p\right)^{1/p} +
    \left(\sum_{i=1}^n|b_i|^p\right)^{1/p}\right]\left(\sum_{i=1}^n|a_i + b_i|^p\right)^{1/q}.$$
  If we use $1 - (1/1) = 1/p$, we finally get the required inequaity.

  Like H\"{o}lder's inequality the equality can be proven for this using the same conditions.
\end{proof}

\begin{remark}
\end{remark} For $0<p<1$, the inequality (\ref{eq:10.7}) gets reversed.

\section{Convex and Concave Functions}
Most of the inequalities discussed so far are consequencce of inequalities for a special class of functions, known as
\textit{convex} and \textit{concave} functions. Consider the function $f(x) = x^n~\forall~n>1$ defined on $\mathbb{R}$. Consider
the case of $n=2$, then on the graphs of this function, the chord joining any two points always lies above the graph. In fact
taking $a<b$, and the point $ka + (1 - k)b$ between $a$ and $b$, we see that
$$[ka + (1 - k)b]^2 - ka^2 - (1 - k)b^2 = -k(1 - k)(a - b)^2\leq 0.$$
Thus,
$$f(ka + (1 - k)b)\leq kf(a) + (1 - k)f(b).$$
This property is the defining property of a convex function. The family of convex functions obey a class of inequalities known as
Jensen's inequality.

Let $I$ be an interval in $\mathbb{R}$. A function $f:I\rightarrow\mathbb{R}$ is said to be convext if for all $x, y$ in $I$ and
$k$ in the interval $[0, 1]$, the following inequality holds:
\begin{equation}
  \label{eq:convex}
  f(kx + (1 - k)y)\leq kf(x) + (1 - k)f(y).
\end{equation}

If the inequality is strict for all $x\neq y$, $f$ is said to be strictly convex on $I$. If the inequality is reverse for same
conditions then $f$ is said to be concave and similarly for strictly concave $f$.

There are other equivalent properties of a convex function. Let $x_1, x_2, x_3$ are in $I$ such that $x_1<x_2<x_3$ and we take $k =
\frac{x_3 - x_2}{x_3 - x_1}$ which gives us
$$1 - k = \frac{x_2 - x_1}{x_3 - x_1},\text{~and~}x_2 = kx_1 + (1 - k)x_3.$$
We have $$\begin{aligned}f(x_2) &= f(kx_1 + (1 - k)x_3)\\&\leq kf(x_1) + (1 - k)f(x_3)\\&=\frac{x_3 - x_2}{x_3 - x_1}f(x_1) +
  \frac{x_2 - x_1}{x_3 - x_1}f(x_3).\end{aligned}$$
We can write this as $$f\frac{f(x_1) - f(x_2)}{x_1 - x_2}\leq\frac{f(x_2) - f(x_3)}{x_2 - x_3},$$
for all $x_1<x_2<x_3$ in $I$. We can also write this as:
$$\frac{f(x_1)}{(x_1 - x_2)(x_1 - x_3)} + \frac{f(x_2)}{(x_2 - x_1)(x_2 - x_3)} + \frac{f(x_3)}{(x_3 - x_1)(x_3 - x_2)}\geq 0.$$
Consider $z_1 = (a, f(a))$ and $z_2 = (b, f(b))$ as two points on $f$. The equation of line joining these two points is given by
$$g(x) = f(a) + \frac{f(b) - f(a)}{b - z}(x - a).$$
Any point between $a$ and $b$ is of the form $x = ka + (1 - k)b$. Thus,
$$\begin{aligned}g(x) & = g(ka + (1 - k)b)\\& = f(a) + \frac{f(b) - f(a)}{b - a}(ka + (1 - k)b - a)\\ & = f(a) + (1 - k)[f(b) -
    f(a)]\\ & = kf(a) + (1 - k)f(b)\\ & \geq f(ka + (1 - k)b) = f(x)\end{aligned}$$
Thus, $(x, g(x))$ lies above $(x, f(x))$, a point on $f$.

We can look at this in another way. A subset $E$ of the plane $\mathbb{R}^2$ is said to be convex if for every pair of points $z_1$
and $z_2$ in $E$, the line joining $z_1$ and $z_2$ lies entirely in $E$. With every function $f: I\rightarrow\mathbb{R}$, we
associate a subset of $\mathbb{R}^2$ by $$E(f) = \{(x, y):a\leq x\leq b, f(x)\leq y\}.$$

\begin{theorem}
  The function $f: I\rightarrow\mathbb{R}$ is convex if and only if $E(f)$ is a convex subset of $\mathbb{R}^2$.
\end{theorem}

\begin{proof}
  Let $f$ be convex. Let $z_1 = (x_1, y_1)$ and $z_2 = (x_2, y_2)$ be two points of $E(f)$. Consider any point on the line zoining
  $z_1$ and $z_2$. Then,
  $$\begin{aligned}z & = kz_1 + (1 - k)z_2\\& = (kx_1 + (1 - k)x_2, ky_1 + (1 - k)y_2)\end{aligned}$$ for some $k\in[0, 1]$. We see
  that $a\leq kx_1 + (1 - k)x_2\leq b$. Moreover,
  $$\begin{aligned}f(kx_1 + (1 - k)x_2)&\leq kf(x_1) + (1 - k)f(x_2)\\&\leq ky_1 + (1 - k)y_2.\end{aligned}$$
  Thus it follows that $z\in E(f)$, provingg that $E(f)$ is convex.

  Conversely let $E(f)$ be convex. Let $x_1, x_2$ be two points in $I$ and let $z_1 = (x_1,f(x_1))$ and $z_2 = (x_2, f(x_2))$. Then
  $z_1$ and $z_2$ are in $E(f)$. By conexity of $E(f)$, the point $kz_1 + (1 - k)z_2$ also lies in $E(f)$ for each $k\in[0,
    1]$. Thus,
  $$(kx_1 + (1 - k)x_2, kf(x_1) + (1 - k)f(x_1))\in E(f)$$
  The definition of $E(f)$ shows that $$f(kx_1 + (1 - k)x_2)\leq kf(x_1) + (1 - k)f(x_2).$$

  This shows that $f$ is convex on the interval $I$.
\end{proof}

Following theorem gives description about slope of a function's graph.

\begin{theorem}
  Let $f:I\rightarrow\mathbb{R}$ be a convex function and $a\in I$ be a fixed point. Define a function $P:I\setminus\{a\}\rightarrow
  \mathbb{R}$ by $$P(x) = \frac{f(x) - f(a)}{x - a}.$$ Then $P$ is a non-decreasing function on $I\setminus\{a\}$.
\end{theorem}

\begin{proof}
  Let $f$ is convex on $I$ and let $x, y$ be two points in $I, x\neq a, x\neq b$ such that $x<y$. Then exactly one of the three
  possibilities will be possible:
  $$a<x<y;~~x<a<y~~x<y<a.$$
  Consider the case $a<x<y$; other cases can be handled similarly. We can write
  $$x = \frac{x - a}{y - a}y + \frac{y - x}{y - a}a.$$
  The convexity of $f$ shows that
  $$f\left(\frac{x - a}{y - a}y + \frac{y - x}{y - a}a\right)\leq \frac{x - a}{y - a}f(y) + \frac{y - x}{y - a}f(a).$$
  This is equivalent to
  $$\frac{f(x) - f(a)}{x - a}\leq \frac{f(y) - f(a)}{y - a}.$$
  Thus $P(x)\leq P(y).$ This shows that $P(x)$ is a non-decreasing function for $x\neq a$.
\end{proof}

Interestingly, the converse is also true; if $P(x)$ is a non-decreasing function on $I\setminus\{a\}$ for every $a\in I$, then
$f(x)$ is convex. We fix $x<y$ in $I$ and let $a = kx + (1 - k)y$ where $k\in(0,1)$. (The cases $k = 0$ or $1$ are obvious.) In
this case
$$P(x) = \frac{f(x) - f(a)}{x - a} = \frac{f(x) - f(a)}{(1 - k)(x - y)}$$
$$P(y) = \frac{f(y) - f(a)}{y - a} = \frac{f(y) - f(a)}{k(y - x)}.$$
The condition $P(x)\leq P(y)$ implies that $f(a)\leq kf(x) + (1 - k)f(y).$ Hence convexity of $f$ is proven.

There is another easy way of deciding wherther a function is convex or concave for twice differentiable functions. If $f$ is convex
on an interval $I$ and if its second derivative exists on $I$, then $f$ is convex(strictly convex) on $I$ if $f''(x)\geq 0(> 0)$ for
all $x\in I$. Similarly $f$ is concave(striclty concave) on $I$ if $f''(x)\leq0(<0)$ for all $x\in I$.

When we defined conex function the inequality involved two points $x, y$; refer to (\ref{eq:convex}). Jensen's inequaity extends
this to any finite number of points.

\section{Jensen's Inequality}
\begin{theorem}
  Let $f:I\rightarrow\mathbb{R}$be a convex function. Let $x_1, x_2, \ldots, x_n$ are points in $I$ and $k_1, k_2, \ldots, k_n$ are
  real numbers in the interval $[0, 1]$ such that $k_1 + k_2 + \cdots + k_n = 1$. Then
  \begin{equation}
    \label{eq:jensen}
    f\left(\sum_{i=1}^nk_ix_i\right)\leq\sum_{i=1}^nk_if(x_i)
  \end{equation}
\end{theorem}

\begin{proof}
  We will use induction to prove this. For $n = 2$, this is the definition of a convex function. Suppose the inequality
  (\ref{eq:jensen}) is true for all $p<n$; i.e. for $p<n$ if $x_1, x-2, \ldots, x_p$ are $p$ points in $I$ and $k_1, k_2, \ldots,
  k_p$ are real numbers in $[0,1]$ such that $\sum_{i=1}^nk-i = 1$, then
  $$f\left(\sum_{i=1}^pk_ix_i\right)\leq \sum_{i=1}^pk+if(x_i).$$
  Now considering the conditions of the theorem,
  $$y_1 = \frac{\sum_{i=1}^{n-1}k_ix_i}{\sum_{i=1}^{n-1}k_i},~~y_2= x_n,~~\alpha_1 = \sum_{j=1}^{n- 1}k_i,~~\alpha_2 = k_n.$$
  We observe that $\alpha_2 = 1 - \alpha_1$, and $y_1, y_2$ are in $I$. Using the conexity of $f$, we get
  $$\begin{aligned}f(\alpha_1y_1 + \alpha_2y_2)& = f(\alpha_1y_1 + (1 - \alpha_1)y_2)\\& \leq\alpha_1f(y_1) + (1 -
    \alpha)1)f(y_2)\\& = \alpha_1f(y_1) + \alpha_2f(y_2).\end{aligned}$$
  However, we have $$\alpha_1y_1 + \alpha_2y_2 = \sum_{i=1}^nk_ix_i.$$
  Now we consider $f(y_1)$. If $$\mu_i = \frac{k_i}{\sum_{i=1}^{n-1}k_i}, 1\leq l\leq n - 1$$
  then it can be easily verifief that $\sum_{l=1}^{n-1}\mu_l = 1$. Using the induction hypothesis, we get
  $$f\left(\sum_{l=1}^{n-1}\mu_lx_l\right)\leq \sum_{l=1}^{n-1}\mu_lf(x_l)$$
  Since $$\sum_{l=1}^{n-1}\mu_lx_l = y_1,$$
  we get
  $$f(y_1)\leq \frac{\sum_{l=1}^{m-1}k_lf(x_l)}{\sum_{i=1}^{n - 1}k_i} = \frac{\sum_{i=1}^{n-1}f(x_i)}{\alpha_1}$$
  Thus we obtain
  $$f\begin{aligned}\left(\sum_{i=1}^nk_if(x_i)\right)&\leq\alpha_1\left(\frac{\sum_{i=1}^{n-1}k_if(x_i)}{\sum_{i=1}^{n-1}k_i}\right) + k_nf(x_n)\\& = \sum_{i=1}^nk_if(x_i).\end{aligned}$$
  Thus, the theorem is proved by induction.
\end{proof}

\begin{remark}
  If $f:I\rightarrow\mathbb{R}$ is concave, then the inequality (\ref{eq:jensen}) gets reversed. If $x_1, x_2, \ldots, x_n$ are
  points in $I$ and $k_1, k_2, \ldots, k_n$ are real numbers in the interval $[0, 1]$, such that $k_1 + k_2 + \cdots + k_n = 1$,
  then following inequality holds:
  \begin{equation}
    f\left(\sum_{i=1}^nk_ix_i\right)\geq\sum_{i=1}^nk_if(x_i)
  \end{equation}
\end{remark}

\begin{remark}
  Using the concavity of $f(x) = \ln x$ on $(0, \infty)$, the AM-GM inequality can be proved. If $x_1, x_2, \ldots, x_n$ are points
  in $(0, \infty)$ and $k_1, k_2, \ldots, k_n$ are real numbers in the interval $[0,1]$ such that $k_1 + k_2 + \ldots + k_n = 1$,
  then we have
  $$\ln\left(\sum_{i=1}^nk_ix_i\right)\geq\sum_{i=1}^nk_i\ln(x_i)$$
\end{remark}

\begin{proof}
  Taking $k_i = \frac{1}{n}$ for all $i$,
  $$\ln\left(\sum_{i=1}^n\frac{x_i}{n}\right)\geq\frac{1}{n}\sum_{i=1}^n\ln x_i = \sum_{i=1}^n\ln\left(x_i^{1/n}\right).$$
  Using the fact that $g(x) = e^x = {\mathrm{exp}}(x)$ is strictly increasing on the interval $(-\infty, \infty)$, this leads to
  $$\begin{aligned}\frac{1}{n}\sum_{i=1}^nx_i & \geq{\mathrm{exp}}\left(\sum_{i=1}^n\ln\left(x_i^{1/n}\right)\right)\\&
    =\prod_{i=1}^n\mathrm{exp}\left(\ln\left(x_i^{1/n}\right)\right)\\& = \left(x_1x_2\ldots x_n\right)^{1/n}.\end{aligned}$$

  We can also prove generalized AM-GM inequality with this method.
  $$\ln\left(\sum_{i=1}^nk_ix_i\geq\sum_{i=1}^nk_i\ln(x_i)\right) = \sum_{i=1}^n\ln x_i^{k_i},$$
  Taking antilog
  $$\sum_{i=1}^nk_ix_i\geq\prod_{i=1}^nx_i^{k_i}.$$
  Nowfor any $n$ positive real numbers $\alpha_1, \alpha_2, \ldots, \alpha_n$, consider
  $$k_i = \frac{\alpha_i}{\sum_{j=1}^n\alpha_j}$$
  Observe that $k_i$ are in $[0, 1]$ and $\sum_{i=1}^nk_i = 1$. These choices of $k_i$ give
  $$\frac{\alpha_1x_1 + \alpha_2x_2 + \cdots + \alpha_nx_n}{\alpha_1 + \alpha_2 + \cdots +
    \alpha_n}\geq\left(x_1^{\alpha_1}x_2^{\alpha_2}\ldots x_n^{\alpha)2}\right)^{1/(\alpha_1 + \alpha_2 + \cdots + \alpha_n)},$$
  which is our generalized AM-GM inequality.
\end{proof}

\begin{remark}
  Function $f(x) = x^p$ can be used to prove H\"{o}lder's inequality. We know that $f(x) = x^p$ is convex for $p\geq 1$ and concave
  for $0,p<1$ for $p\in(0, \infty)$. Let $x_1, x_2, \ldots, x_n$ be real numbers and $k_1, k_2, \ldots, k_n$ in $[0, 1]$, then we
  have
  $$\left(\sum_{i=1}^nk_ix_i\right)^p\leq \sum_{i=1}^nk_ix_i^p\text{~for~}p\geq 1$$
  and
  $$\left(\sum_{i=1}^nk_ix_i\right)^p\geq \sum_{i=1}^nk_ix_i^p\text{~for~}0<p<1.$$
\end{remark}

\begin{proof}
  Let $a_1, a_2, \ldots, a_n, b_1, b_2, \ldots, b_n$ be real numbers and $p>1$ and $q$ be conjugate numbers. Thus, $\frac{1}{p} +
  \frac{1}{q} = 1.$ We need to assume that $b_i\neq 0$ for all $i$; else we may delete all those $b_i$ which are zero without
  having an effect on the equality. Let
  $$t = \sum_{i=1}^n|b_i|^q,~~k_j = \frac{|b_j|^q}{t},~~x_j = \frac{|a_j|}{|b_j|^{q - 1}}$$
  We have $k_j\in[0, 1]$ and $k_1 + k_2 + \ldots + k_n = 1$. Using the conexity of $x^p$, we have
  $$\left(\sum_{i=1}^nk_ix_i\right)^p\leq\sum_{i=1}^nk_ix_i^p,$$
  which implies that
  $$\left(\sum_{j=1}^n\frac{|b_j|^q}{t}\frac{|a_j|}{|b_j|^{q - 1}}\right)^p \leq
  \sum_{j=1}^n\frac{|b_j|^1}{t}\frac{|a_j|^p}{|b_j|^{(q - 1)p}} = \frac{1}{t}\sum_{j=1}^n|a_j|^p.$$
  Futher simplification yields
  $$\sum_{j=1}^n|a_jb_j|\leq\left(\sum_{j=1}^n|a_j|^p\right)^{1/p}t^{1 - (1/p)} =
  \left(\sum_{j=1}^n|a_j|^p\right)^{1/p}\left(\sum_{j=1}^n|b_j|^q\right)^{1/q}$$
  For concave case the inequality is simply reversed.
\end{proof}

\begin{theorem}
  Let $f:I\rightarrow\mathbb{R}$ be a convex function; $a_1\leq a_2\leq\cdots\leq a_n, b_1, b_2, \ldots, b_n$ are real numbers in
  $I$ such that $a_1 + b_1\in I$ and $a_n + b_n\in I$. Let $a_1', a_2', \ldots, a_n'$ be a permutation of $a_1, a_2, \ldots,
  a_n$. Then the follwoing inequality is true: $$\sum_{i=1}^nf(a_i + b_{n+1-i})\leq\sum_{i=1}^nf(a_i' + b_i)\leq\sum_{i=1}^nf(a_i +
  b_i).$$
\end{theorem}

\begin{proof}
  We will use the proof of rearrangement inequality. Assume $\langle a_i'\rangle \neq\langle a_i\rangle$ and $r$ be the largest
  index such that $a_r'\neq a_r$. Since $a_i=a_i'$ for $r<i\leq n$, we see that $a_1', a_2', \ldots, a_r'$ is a permutation of
  $(a_1, a_2, \ldots, a_r)$. Thus we can find $k<r, l<r$ such that $a_k' = a_r$ and $a_r' = a_l$. We deduce that $a_k' - a_r' = a_r
  - a_l\geq 0$ and $b_r - b_k\geq 0$. Interchanging $a_r'$ and $a_k'$ to get a permutation $(a_1'', a_2'', \ldots, a_n'')$ of $(a_1',
  a_2', \ldots, a_n')$. Thus
  $$a_i''= a_i'\text{~{\rm{for}}~} j\neq r,k,~~a_r''=a_k'=ar,~~a_k''=a_r'=a_l.$$
  Let $$S'' = \sum_{i=1}^nf(a_i'' + b_i),~~S' = \sum_{i=1}^nf(a_i' + b_i).$$
  Then, $$\begin{aligned}S'' - S' & = f(a_r'' + b_r) + f(a_k'' + b_k) - f(a_r' + b_r) - f(a_k' + b_k)\\& = f(a_r + b_r) + f(a_l +
    b_k) - f(a_l + b_r) - f(a_r + b_k).\end{aligned}$$
  We notice that $$a_l + b_k<a_r + b_k\text{~{\rm{and}}~}a_l + b_r < a_r + b_r.$$
  These give $$a_l + b_k < a_r + b_k \leq a_r + br,~~a_l + b_k\leq a_l + b_r < a_r + b_r.$$
  If $x_1, x_2, x_3$ are in $I$, then the convexity of $f$ implies that $$(x_3 - x_1)f(x_2)\leq(x_3 - x_2)f(x_1) + (x_2 -
  x_1)f(x_3).$$
  Putting $x_1 = a_l + b_k, x_2 = a_r + b_k$ and $x_3 = a_r + b_r$, we get
  $$(a_r + b_r - a_l - b_k)f(a_r + b_k)\leq (b_r - b_k)f(a_l + b_k) + (a_r - a_l)f(a_r + b_r).$$
  Similarly putting $x_1 = a_l + b_k, x_2= a_l + b_r$ and $x_3 = a_r + b_r$, we get
  $$(a_r + b_r - a_l -b_k)f(a_l + b_r)\leq(a_r - a_l)f(a_l + b_k) + (b_r - b_k)f(a_r + b_r).$$
  Adding, we get
  $$(a_r + b_r - a_l - b_k)\{f(a_r + b_k) + f(a_l + b_r)\}\leq (a_r + b_r - a_l - b_k)\{f(a_l + b_k) + f(a_r + b_r)\}.$$
  Since $a_l + b_k < a_r + b_r$, we arrive at
  $$f(a_r + b_k) + f(a_l + b_r)\leq f(a_l + b_k) + f(a_r + b_r).$$
  This proves that $S'' - S' \geq 0$.

  Now we observe that the permutation $(a_1'', a_2'', \ldots, a_n'')$ has the property $a_r''=a_r$ and $a_i'' = a_i$, for $r <
  j\leq n$. We may consider the $(a_1'', a_2'', \ldots, a_n'')$ in place $(a_1', a_2', \ldots, a_n')$ and proceed as above. After
  at most $n-1$ steps we arrive at the original numbers $\langle a_i\rangle$ from $\langle a_i'\rangle$ and at each stage the
  corresponding sum in non-decreasing. Thus, finally we arrive at
  $$\sum_{i=1}^nf(a_i' + n_i)\leq \sum_{i=1}^nf(a_i + b_i).$$
  For the other inequality we define $c_i = a_{n+1-i}$ so that $c_1\geq c_2\geq\cdots\geq c_n$. We have to show that
  $$\sum_{i=1}^nf(a_{n+1-i} + b_i)\leq\sum_{i=1}^nf(a_i' + b_i).$$
  Setting $c_i' = a_i'$, we have
  $$\sum_{i=1}^nf(c_i + b_i)\leq\sum_{i=1}^nf(c_i' + b_i),$$
  where $(c_1', c_2', \ldots, c_n')$ is a permutation of $(c_1, c_2, \ldots, c_n)$. We take $\langle c_i'\rangle\neq\langle
  c_i\rangle$ and let $r$ be the smallest index such that $c_r'\neq c_r$. This forces that $c_r'\in\{c_{r+1}, c_{r+2},\ldots,
  c_n\}$ and $c_r'<c_r$. We see that $(c_r', c_{r+1}', \ldots, c_n')$ is a permutation of $(c_r, c_{r+1}, \ldots, c_n)$. We can
  find $k>r, l>r$ such that $c_k' = c_r$ and $c_r' = c_l$. This implies that $c_k' - c_r' = c_r - c_l\geq 0$ and $b_k - b_r\geq
  0$. Now we can interchange $c_r'$ and $c_k'$ to get a permutation $(c_1'', c_2'', \ldots, c_n'')$ of $(c_1', c_2', \ldots,
  c_n')$; thus
  $$c_i''=c_i'\text{~{\rm{for}}~}i\neq r, k,~~c_r''=c_k' = c_r,~~c_k''=c_r'=c_l.$$
  We compute the difference between
  $$S'' = \sum_{i=1}^nf(c_i'' + b_i),~~S'=\sum_{i=1}^nf(c_i' + b_i),$$
  and obtain
  $$\begin{aligned}S'' - S'& = f(c_r'' + b_r) + f(c_k'' + b_k) - f(c_r' + b_r) - f(c_k' + b_k)\\& = f(c_r + b_r) + f(c_l + b_k) -
    f(c_l + b_r) - f(c_r + b_k).\end{aligned}$$
  We see that
  $$c_l + b_r \leq c_l + b_k < c_r + b_k, c_l + b_r\leq c_r + b_r < c_r + b_k.$$
  From the convexity of $f$
  $$(c_r + b_k - c_l - b_r)f(c_l + b_k)\leq(c_r - c_l)f(c_l + b_r) + (b_k - b_r)f(c_r + b_k),$$
  and
  $$(c_r + b_k - c_l - b_r)f(c_r + b_r)\leq(b_k - b_r)f(c_l + b_r) + (c_r - c_l)f(c_r + b_k).$$
  Adding, we get
  $$(c_r + b_k - c_l - b_r)\{f(c_l + b_k) + f(c_r + b_r)\}\leq(c_r + b_k - c_l - b_r)\{f(c_l + b_r) + f(c_r + b_k)\}.$$
  We know that $c_r + b_k - c_l - n_r \neq 0$, so we have
  $$f(c_l + b_k) + f(c_r + b_r)\leq f(c_l + b_r) + f(c_r + b_k).$$
  Thus, we see that $S''\leq S'$. We also see that the new sequence $\langle c_i''\rangle$ has the property: $c_r'' = c_r$ and
  $C_i'' = c_i$ for $1\leq i< r$. Now we repeat the above argument by replacing $\langle c_i'\rangle$ with $\langle
  c_i''\rangle$. At each step the sum will never increase. After at most $n-1$ steps we arrive at the sequence $\langle
  c_i\rangle$. Thus, we find that the corresponding sum does not exceed to that of $S'$. Thus we get
  $$\sum_{i=1}^nf(c_i + b_i)\leq\sum_{i=1}^nf(c_i' + b_i),$$ which was to be proved.
\end{proof}

\section{Bernoulli's Inequality}
\begin{theorem}
  For every real number $r\geq 1$ and real number $x\geq -1$, we have $$(1 + x)^r\geq 1 + rx$$ while for $0\leq r\leq 1$ and real
  number $x\geq -1$ we have $$(1 + x)^r\leq 1 + rx.$$
\end{theorem}

\begin{proof}
  Using the convexity of $f(x) = \ln(x)$ on $(0, \infty)$. Since $x\geq -1$, we have $1 + x\geq 0$. If $0 \leq r \leq 1$, we have
  $$\ln(1 + rx) = \ln(r(1 + x) + 1 - r)\geq r\ln(1 + x) + (1 - r)\ln(1) = r\ln(1 + x).$$
  Taking antilog gives $(1 + x)^r\leq 1 + rx.$ When $1\leq r < \infty$,
  $$\ln(1 + x) = \ln\left(\frac{r - 1}{r} + \frac{1}{r}(1 + rx)\right)\geq\frac{r - 1}{r}\ln(1) + \frac{1}{r}\ln(1 + rx) =
  \frac{1}{r}\ln(1 + rx).$$
  This gives $(1 + x)^r\geq 1 + rx.$
\end{proof}

\section{Popoviciu's Inequality}
\begin{theorem}
  Let $f: I\rightarrow\mathbb{R}$. If $f$ is convex, then for any three p;oints $x, y, z$ in $I$:
  \begin{equation}
    \frac{f(x) + f(y) + f(z)}{3} + f\left(\frac{x + y + z}{3}\right)\geq \frac{2}{3}\left[f\left(\frac{x + y}{2}\right) +
      f\left(\frac{y + z}{2}\right) + f\left(\frac{z + x}{2}\right)\right]
  \end{equation}
\end{theorem}

\begin{proof}
  Without loss of generality, we can assume that $x\leq y\leq z$. If $x\leq y\leq \frac{x + y + z}{3}$, then
  $$\frac{x + y + z}{3}\leq \frac{x + z}{2}\leq z\text{~{\rm{and}}~}\frac{x + y + z}{3}\leq \frac{y + z}{2}\leq z.$$
  Therefore, there exists $s, t\in[0,1]$ such that
  $$\frac{x + z}{2} = \left(\frac{x + y + z}{3}\right)s + z(1 - s)$$
  $$\frac{y + z}{2} = \left(\frac{x + y + z}{3}\right)t + z(1 - t)$$
  Adding, we get
  $$\frac{x + y - 2z}{2} = \frac{x + y - 2z}{3}(s + t)\Rightarrow s + t = \frac{3}{2}.$$
  As $f$ is a convex function
  $$f\left(\frac{x + z}{2}\right)\leq s.f\left(\frac{x + y + z}{3}\right) + (1 - s).f(z)$$
  $$f\left(\frac{y + z}{2}\right)\leq t.f\left(\frac{x + y + z}{3}\right) + (1 - t).f(z)$$
  and
  $$f\left(\frac{x + y}{2}\right)\leq \frac{1}{2}f(x) + \frac{1}{2}f(y).$$
  Adding together last three inequalities we get the required inequality.
  The case when $\frac{x + y + z}{3}\leq y$ is considered similarly, bearing in mind that $x\leq \frac{x + z}{2}\leq \frac{x + y +
    z}{3}$ and $x\leq\frac{y + z}{2}\leq\frac{x + y + z}{3}$.

  When $f$ is a concave function, the inequality gets reversed.
\end{proof}

\section{Majorization}
\begin{definition}[Majorization]
  Given two seuquences $\langle a\rangle = (a_1, a_2, \ldots, a_n)$ and $\langle b\rangle = (b_1, b_2, \ldots, b_n)$ where $a_i,
  b_i\in\mathbb{R}~~\forall i\in\{1,2,\ldots, n\}$. We say that the sequence $\langle a\rangle$ majorizes the seuqnece $\langle
  b\rangle$, and write $\langle a\rangle\succ \langle b\rangle$, if the following conditions are fulfilled:
  $$a_1\geq a_2\geq \cdots\geq a_n;$$
  $$b_1\geq b_2\geq \cdots\geq b_n;$$
  $$a_1 + a_2 + \cdots + a_n = b_1 + b_2 + \cdots + b_n;$$
  $$a_1 + a_2 + \cdots + a_k\geq b_1 + b_2 + \cdots + b_k~~~\forall k\in\{1, 2, \ldots, n - 1\}.$$
\end{definition}

\section{Karamata's Inequality}
\begin{theorem}
  Let $f: [a,b]\rightarrow\mathbb{R}$ be a convex function. Suppose that $(x_1, \ldots, x_n)\succ(y_1, \ldots, y_n)$ where $x_1,
  \ldots, x_n, y_1, \ldots, y_n\in[a, b]$. Then we have:
  \begin{equation}
    \sum_{i=1}^nf(x_i)\geq\sum_{i=1}^nf(y_i).
  \end{equation}
\end{theorem}

\begin{proof}
  If $f(x)$ is a convex function over the interval $(a, b)$, then $\forall a\leq x_1\leq x_2\leq b$ and $g(x, y)=\frac{f(y) -
    f(x)}{y - x}, f(x_1,x)\leq g(x_2, x)$. If $x<x_1$, then
  $$g(x_1, x) = \frac{f(x_1) - f(x)}{x_1 - x}\leq\frac{f(x_1) - f(x)}{x_1 - x} = g(x_2- x).$$
  We can argue similarly for other values of $x$.

  \noindent We define a sequence $\langle C\rangle$ such that $c_i = g(a_i, b_i)$

  \noindent We also define sequences $\langle A\rangle$ and $\langle B\rangle$ such that $$A_i = \sum_{j=1}^ia_j, A_0 =
  0\text{~{\rm{and}}~} B_i = \sum_{j=1}^ib_j, B_0 =0$$
  If we assume that $a_i\geq a_{i+1}$ and similarly $b_i\geq b_{i+1}$, then we get that $c_i\geq c_{i+1}$. Now, we know that
  $$\sum_{i=1}^nf(a_i) - \sum_{i=1}^nf(b_I) = \sum_{i=1}^nc_i(a_i - b_i) = \sum_{i=1}^nc_i(A_i - A_{i-1} - B_i + B_{i + 1})$$
  $$=\sum_{i=1}^nc_i(A_i - B_i) - \sum_{i=0}^{n-1}c_{i+1}(A_i - B_i) = \sum_{i=1}^n(c_i - c_{i+1})(A_i - B_i)\geq 0$$
  Therefore,
  $$\sum_{i=1}^nf(x_i)\geq\sum_{i=1}^nf(y_i).$$
\end{proof}

\section{Muirhead's Inequality}
\begin{theorem}
  If a sequence $\langle a\rangle$ majorises a sequence $\langle b\rangle$, and $x_1, x_2, \ldots, x_n$ be a set of postiive real
  numbers then
  \begin{equation}
    \sum_{sym}x_1^{a_1}x_2^{a_2}\ldots x_n^{a_n}\geq \sum_{sym}x_1^{b_1}x_2^{b_2}\ldots x_n^{b_n}
  \end{equation}
\end{theorem}

\begin{proof}
  We define a sequence $\langle c\rangle$ such that $\sum_{i=1}^nc_i = 0$, the we observe
  $$\sum_{sym}x_1^{c_1}x_2^{c_2}\ldots x_n^{c_n}\geq n!$$
  for real $x_1, x_2, \ldots x_n$.
  By AM-GM we know that
  $$\frac{\sum_{sym}x_1^{c_1}x_2^{c_2}\ldots x_n^{c_n}}{n!}\geq n!\sqrt{\prod_{sym}x_1^{c_1}x_2^{c_2}\ldots x_n^{c_n}}$$
  $$\Rightarrow n!\sqrt{\prod_{sym}x_1^{c_1}x_2^{c_2}\ldots x_n^{c_n}} = n!\sqrt{\prod_{i=1}^nx_i^{(n-1)!(c_1 + c_2 + \cdots
      c_n)}} = 1$$
  $$\Rightarrow \sum_{sym}x_1^{c_1}x_2^{c_2}\ldots x_n^{c_n} \geq n!$$
  We defined out sequence $\langle c\rangle$ such that $c_i = a_ i - b_i$ which gives us $\sum c_i = \sum a_i - \sum b_i = 0$

  \noindent Thus, $\sum_{sym}x_1^{c_1}x_2^{c_2}\ldots x_n^{c_n} - n! \geq 0.$
  Multiplying with $\sum_{sym}\prod_{i=1}^nx_i^{b_i}$, we get
  $$\left(\sum_{sym}\prod_{i=1}^nx_i^{b_i}\right)\left(\sum_{sym}x_1^{c_1}x_2^{c_2}\ldots x_n^{c_n} - 1\right)$$
  $$= \sum_{sym}\prod_{i=1}^nx_i^{b_i + c_i} - \prod_{i=1}^nx_i^{b_1}\geq 0$$
  $$\Rightarrow \sum_{sym}\prod_{i=1}^nx_i^{a_i} - \prod_{i=1}^nx_i^{b_1}\geq 0$$
  Hence, it is proved.
\end{proof}

\section{Schur's Inequality}
\begin{theorem}
  Let $x, y, z$ be non-negative real numbers. For any $r > 0$, we have
  \begin{equation}
    \sum_{cyc}x^r(x - y)(x - z) \geq 0
  \end{equation}
  with equality if and only if $x = y = z$, or if two of $x, y, z$ are equal and the third is $0$.
\end{theorem}

\begin{proof}
  When $r = 1$, the following case arises:
  $$x^3 + y^3 + z^3 + 3xyz \geq xy(x + y) + yz(y + z) + zx(z + x).$$
  Because L.H.S. is cyclic in $x, y, z$ without loss of generality we can assume $x\geq y \geq z$. Rewriting L.H.S., we have
  $$(x - y)[x^r(x - z) - y^r(y - z)] + z^r(z - x)(z - y).$$
  We see that $x^r \geq y^r$ and $x - z\geq y - z$. Thus the expression inside brackets is non-negative. $(x - y)$ is also
  non-negative. $z^r$ and $(z - x)(z - y)$ are also non-negtive. Thus entire expression is non-negative and hence the inequality is
  proven.
\end{proof}

\noindent Velentin Vornicu has given a general form of Schur's inequality. Consider $a, b, c, x, y, z\in\mathbb{R}$, where $a\geq
b\geq c$, and either $z\geq y\geq z$ or $z\geq y\geq x$. Let $k\in\mathbb{Z}^+$, and let $f:\mathbb{R}\rightarrow\mathbb{R}_0^+$ be
either convex or monotonic, then
\begin{equation}
  f(x)(a - b)^k(a - c)^k + f(y)(b - a)^k(b - c)^k + f(z)(c - a)^k(c - b)^k\geq 0.
\end{equation}

\section{Symmetric Functions}
Let $a_1, a_2, \ldots, a_n$ be arbitrary real numbers. Considering the polynomial $P(x) = (x + a_1)(x + a_2)\cdots(x + a_n) =
c_ox^n + c_1x^{n - 1} + \cdots + c_{n-1}x + c_n$. The the coefficients $c_0, c_1, \ldots, c_n$ can be expressed as functions of
$a_1, a_2, a_n$ like $c_0 = 1, c_1 = a_1 + a_2 + \cdots + a_n, c_2 = a_1a_2 + a_2a_3 + \cdots, a_{n-1}a_n, c_3 = a_1a_2a_3 +
a_2a_3a_4 + \cdots + a_{n-2}a_{n-1}a_n, \ldots, c_n = a_1a_2\ldots a_n$.

These are also called \textit{elementary symmetric sum} and the first elementary symmetric sum of $f(x)$ is often written as
$\sum_{sym}f(x)$ while the $n$th can be written as $\sum_{sym}^nf(x)$.

The \textit{symmetric sum} $\sum_{sym}f(x_1, x_2, \ldots, x_n)$ of a function $f(x_1, x_2, \ldots, x_n)$ of $n$ variables is
defined to be $\sum_{\sigma}f(x_{\sigma(1)}, x_{\sigma(2)}, \ldots, x_{\sigma(n)})$, where $\sigma$ ranges over over all
permutations of $(1, 2, \ldots, n)$. More generally symmetric sum of $n$ variables is a sum that is unchanged by any permutatoin of
its variables. Any symmetric sum can be written as a polynomial of elementary symmetric sums.

A \textit{symmetric function} of $n$ variables is a function that does not change by any permutation of its variables. Therefore,

$$\sum_{sym}f(x_1, x_2, \ldots, x_n) = n!f(x_1, x_2, \ldots, x_n)$$

We define \textit{symmetric average} $p_k$ as $\frac{c_k}{\binom{n}{k}}$.

\section{Newton's Inequality}
\begin{theorem}
  For non-negative $x_1, x_2, \ldots, x_n$ and $0<k< n$m
  \begin{equation}
    d_k^2\geq d_{k-1}d_{k + 1},
  \end{equation}
  equality holds when all $x_i$'s are equal.
\end{theorem}

\begin{proof}
  We will prove this by mathematical induction. A proof by calculus is also possible but we will not prove by that method.

  For $n = 2$, the inequality becomes AM-GM inequaltiy. Let the inequality hold for $n = m - 1$ for some positive integer $m\geq
  3$.

  Let $d_k'$ be the symmetric averages of $x_1, x_2, \ldots, x_{m-1}$. Note that $d_k = \frac{n-k}{n}{d'}_k +
  \frac{k}{n}{d'}_{k-1}x_m$.
  \[d_{k-1}d_{k+1} = \left(\frac{n-k+1}{n} {d'}_{k-1} + \frac{k-1}{n} {d'}_{k-2} x_m \right)\left(\frac{n-k-1}{n} {d'}_{k+1} +
  \frac{k+1}{n} {d'}_k x_m \right)\]
  \[= \frac{(n-k+1)(n-k-1)}{n^2} {d'}_{k-1}{d'}_{k+1} + \frac{(k-1)(n-k-1)}{n^2} {d'}_{k-2} {d'}_{k+1} x_m\]
  \[+ \frac{(n-k+1)(k+1)}{n^2} {d'}_{k-1}{d'}_k x_m + \frac{(k-1)(k+1)}{n^2} {d'}_{k-2}{d'}_k x_m^2\]
  \[\le  \frac{(n-k+1)(n-k-1)}{n^2} {d'}_k^2 + \frac{(k-1)(n-k-1)}{n^2} {d'}_{k-2} {d'}_{k+1} x_m\]
  \[+ \frac{(n-k+1)(k+1)}{n^2} {d'}_{k-1}{d'}_k x_m + \frac{(k-1)(k+1)}{n^2} {d'}_{k-1}^2 x_m^2\]
  \[\le  \frac{(n-k+1)(n-k-1)}{n^2} {d'}_k^2 + \frac{(k-1)(n-k-1)}{n^2} {d'}_{k-1} {d'}_{k} x_m\]
  \[+ \frac{(n-k+1)(k+1)}{n^2} {d'}_{k-1}{d'}_k x_m + \frac{(k-1)(k+1)}{n^2} {d'}_{k-1}^2 x_m^2\]
  \[= \frac{(n-k)^2}{n^2} {d'}_k^2 + \frac{2(n-k)k}{n^2} {d'}_k {d'}_{k-1} x_m +\frac{k^2}{n^2} {d'}_{k-1}^2 x_m^2 -
  \left(\frac{d_k}{n} - \frac{d_{k-1}x_m}{n}\right)^2\]
  \[\le  \left(\frac{n-k}{n} {d'}_k + \frac{k}{n} {d'}_{k-1} x_m \right)^2        = d_k^2\]
  Hence, it is proven by induction.
\end{proof}

\section{Maclaurin's Inequality}
\begin{theorem}
  For non-negative $x_1, x_2, \ldots, x_n$ and $0<k< n$m
  \begin{equation}
    d_1\geq d_2^{1/2}\geq\cdots\geq d_n^{1/n},
  \end{equation}
  equality holds when all $x_i$'s are equal.
\end{theorem}

\begin{proof}
  Following Newton's inequality it is enough to show that $d_{n-1}^{1/(n-1)}\geq d_n^{1/n}$.

  Since this is a homogeneous inequaqlity, it can be normalized. Thus, $d_n = \prod x_i = 1$ We then transform the inequality to(by
  exponentiating both sides by $n-1$)
  $$\frac{\sum 1/x_i}{n}\geq 1^{(n-1)/n} = 1.$$ We know that the G.M. of $\frac{1}{x_1}, \frac{1}{x_2}, \cdots,
  \frac{1}{x_n}$ is $1$ and hence the inequality is true by AM-GM.
\end{proof}

\section{Aczel's Inequality}
\begin{theorem}
  If $a_1^2>a_2^2 + \cdots + a_n^2$ or $b_1^2 > b_2^2 + \cdots + b_n^2$, then
  \begin{equation}
    (a_1b_1 - a_2b_2 - \cdots - a_nb_n)^2 \geq (a_1^2 - a_2^2 - \cdots - a_n^2)(b_1^2 - b_2^2 - \cdots - b_n^2)
  \end{equation}
\end{theorem}

\begin{proof}
  Consider the function $$f(x) = (a_1x  - b_1)^2 - \sum_{i=2}^n(a_ix - b_i)^2$$
  $$= (a_1^2 - a_2^2 - \cdots - a_n^2)x^2 - 2(a_1b_2 - a_2b_2 - \cdots - a_nb_n)x + (b_1^2 - b_2^2 - \cdots - b_n^2).$$
  We have $f\left(\frac{b_1}{a_1}\right) = -\sum_{i=2}^n\left(a_i\frac{b_1}{a_1} - b_i\right)^2 \leq 0$, and from $a_1^2>a_2^2 +
  \cdots + a_n^2$ we get $\displaystyle\lim_{x\to\infty}f(x)\rightarrow\infty$. Therefore, $f(x)$ must have at least one root,
  $\Leftrightarrow D = (a_1b_1 - a_2b_2 - \cdots - a_nb_n)^2 - (a_1^2 - a_2^2 - \cdots - a_n^2)(b_1^2 - b_2^2 - \cdots - b_n^2)\geq
  0.$
\end{proof}

\section{Carleman's Inequality}
\begin{theorem}
  Let $a_1, a_2, \ldots, a_n$ be $n$ non-negative real numbers, where $n\geq 1$ then
  \begin{equation}
    \sum_{i=1}^{\infty} (a_1 a_2 \dotsm a_i)^{1/i} < e \sum_{i=1}^{\infty} a_i ,
  \end{equation}
  unless all of $a_i$'s are equal to zero.
\end{theorem}

\begin{proof}
  Let us define $c_n = n\left(1 + \frac{1}{n}\right)^n = \frac{(n + 1)^n}{n^{n - 1}}$. Then for all positive integers $i$,
  $$(c_1\ldots c_i)^{1/i} = i + 1$$
  $$\Rightarrow \sum_{i=1}^\infty(a_1\ldots a_i)^{1/i} = \sum_{i=1}^\infty\frac{(c_1a_1\ldots c_ia_i)^{1/i}}{(c_1\ldots
    c_i)^{1/i}} = \sum_{i=1}^\infty\frac{(c_1a_1\ldots c_ia_i)^{1/i}}{i + 1}.$$
  Using AM-GM inequality, we get
  $$\sum_{i=1}^\infty\frac{(c_1a_1\ldots c_ia_i)^{1/i}}{i + 1} \leq \sum_{i=1}^\infty\sum_{j=1}^i\frac{c_ja_j}{i(i + 1)} =
  \sum_{j=1}^\infty\sum_{i=j}^\infty\frac{c_ja_j}{i(i + 1)}.$$
  Using the partial fraction for $\frac{1}{i(i + 1)}$
  $$\sum_{i=j}^\infty\frac{1}{i(i + 1)} = \sum_{i=j}^\infty\left(\frac{1}{i} - \frac{1}{i + 1}\right) = \frac{1}{j}.$$
  $$\Rightarrow \sum_{j=1}^\infty\sum_{i=j}^\infty\frac{c_ja_j}{i(i + 1)} = \sum_{j=1}^\infty\left(1 + \frac{1}{j}\right)^ja_i.$$
  Since $\left(1 + \frac{1}{j}\right)^j< e,~\forall~j\in I$ the inequality holds.
\end{proof}

\section{Sum of Squares(SOS Method)}
Sum of sqaures or S.O.S. method revolves around the basic fact that sum of
squares is a non-negative quantity. As you can see it requires knowledge only of
very basic inequalitites which makes it highly desirable. By using SOS method we
rewrite inequalitites as \textit{sum of squares} to prove them as non-negative
using only basic inequalities.

\begin{proposition}
  \label{pp:sos1}
  Let $a, b, c\in\mathbb{R}$. Then $(a - c)^2\leq 2(a - b)^2 + 2(b - c)^2$.
\end{proposition}

\begin{proof}
  We have $$(a - c)^2\leq 2(a - b)^2 + 2(b - c)^2$$
  $$\Leftrightarrow a^2 - 2ac + c^2\leq 2(a^2 - 2ab + b^2) + 2(b^2 - 2bc +
  c^2)$$
  $$\Leftrightarrow a^2 + 4b^2 + c^2 0 4ab - 4bc + 2ac\geq 0$$
  $$\Leftrightarrow (a + c - 2b)^2\geq 0,$$
  which clearly holds.
\end{proof}

\begin{proposition}
  \label{pp:sos2}
  Let $a\geq b\geq c$. Then $(a - c)^2\geq (a - b)^2 + (b - c)^2$.
\end{proposition}

\begin{proof}
  We have $$(a - c)^2\geq (a - b)^2 + (b - c)^2$$
  $$\Leftrightarrow a^2 - 2ac + c^2 \geq (a^2 - 2ab + b^2) + (b^2 - 2bc + c^2)$$
  $$\Leftrightarrow b^2 + ac - ab - b \leq 0$$
  $$\Leftrightarrow (b - a)(b - c)\leq 0,$$
  which is true for $a\geq b\geq c$.
\end{proof}

\begin{proposition}
  \label{pp:sos3}
  Let $a\geq b\geq c$. Then $\frac{a - c}{b - c}\geq \frac{a}{b}$.
\end{proposition}

\begin{proof}
  Given $\frac{a - c}{b - c}\geq \frac{a}{b}$
  $$\Leftrightarrow b(a - c)\geq a(b - c)\Leftrightarrow ac\geq
  bc\Leftrightarrow a\geq b.$$
\end{proof}

\begin{theorem}
  Consider the expression $S = S_a(b - c)^2 + S_b(c - a)^2 + S_c(a - b)^2$,
  where $S_a, S_b, S_c$ are functions of $a, b, c$.
  \begin{enumerate}
  \item If $S_a, S_b, S_c\geq 0$ then $S\geq 0$.
  \item If $a\geq b\geq c$ or $a\leq b\leq c$ and $S_b, S_b + S_a, S_b + S_c\geq
    0$ then $S\geq 0$.
  \item If $a\geq b\geq c$ or $a\leq b\leq c$ and $S_a, S_c, S_a + 2S_b, S_c +
    2S_b \geq 0$ then $S\geq 0$.
  \item If $a\geq b\geq c$ and $S_b, S_c, a^2S_b + b^2S_a\geq 0$ then $S\geq 0$.
  \item If $S_a + S_b\geq 0$ or $S_b + S_c\geq 0$ or $S_c + S_a \geq 0$ or $S_a
    + S_b + S_c \geq 0$ and $S_aS_b + S_bS_c + S_cS_a\geq 0$ then $S\geq 0$.
  \end{enumerate}
\end{theorem}

\begin{proof}
  \begin{enumerate}
  \item If $S_a, S_b, S_c\geq 0$ then clearly $S\geq 0$.
  \item Let us assume that $a\geq b\geq c$ or $a\leq b\leq c$ and $S_b, S_b +
    S_a, S_b + S_c\geq 0$.

    \noindent By Proposition (\ref{pp:sos2}), it follows that $(a - c)^2 \geq (a
    - b)^2 + (b - c)^2$, so we have
    $$\begin{aligned}S &= S_a(b - c)^2 + S_b(c - a)^2 + S_c(a - b)^2\\&\geq S_a(b
      - c)^2 + S_b[(a - b)^2 + (b - c)^2] + S_c(a - b)^2\\&=(b - c)^2(S_a + S_b)
      + (a - b)^2(S_b + S_c).\end{aligned}$$
    Thus, $S\geq 0$ because $S_a + S_b, S_b + S_c\geq 0$.
    \item Let us assume that $a\geq b\geq c$ or $a\leq b\leq c$ and $S_a, S_c,
      S_a + 2S_b, S_c + 2S_b \geq 0$.

      \noindent Then if $S_b\geq 0$ clearly $S\geq 0$.

      \noindent For case when $S_b\leq 0$, by Proposition (\ref{pp:sos1}), we
      have $(a - c)^2 \leq 2(a - b)^2 + 2(b - c)^2$. Therefore
      $$\begin{aligned}S &= S_b(b - c)^2 + S_b(a - c)^2 + S_c(a - b)^2\\&\geq
        S_a(b - c)^2 + S_b[2(a - b)^2 + 2(b - c)^2] + S_c(a - b)^2\\&= (b -
        c)^2(S_a + 2S_b) + (a - b)^2(S_c + 2S_b)\end{aligned}$$
      which is true for the given conditions.
    \item Given $a\geq b\geq c$ and $S_b, S_c, a^2S_b + b^2S_a\geq 0$

      \noindent By Proposition (\ref{pp:sos3}), we have $\frac{a - c}{b - c}\geq
      \frac{a}{b}$. Therefore
      $$\begin{aligned}S &= S_a(b - c)^2 + S_b(a - c)^2 + S_c(a - b)^2\geq S_a(b
        - c)^2 + S_b(a - c)^2\\&=(b - c)^2\left[S_a + S_b\left(\frac{a - c}{b -
            c}\right)^2\right]\geq (b - c)^2\left[S_a +
          S_b\left(\frac{a}{b}\right)^2\right]\\& = (b - c)^2\left(\frac{b^2S_a
          + a^2S_b}{b^2}\right),\end{aligned}$$
      which is true for given conditions.
    \item We assume that $S_b + S_c\geq 0$. Then
      $$\begin{aligned}S & = S_a(b - c)^2 + S_b(a - c)^2 + S_c(a - b)^2\\&=
      S_a(b - c)^2 +
      S_b[(c - b) + (b - a)]^2 + S_c(a - b)^2\\&= (S_b + S_c)(a - b)^2 + 2S_b(c
      - b)(b - a) + (S_a + S_b)(b - c)^2\\&=(S_b + S_c)\left(b - a +
      \frac{S_b}{S_b + S_c}(c - b)\right)^2 + \frac{S_aS_b + S_bS_c+ S_cS_a}{S_b
      + S_c}(c - b)^2&\geq 0.\end{aligned}$$
  \end{enumerate}
\end{proof}

Every difference $\sum_{cyc}x_1^{\alpha_1}x_2^{\alpha_2}\ldots x_n^{\alpha_n} -
\sum_{cyc}x_1^{\beta_1}x_2^{\beta_2}\ldots x_n^{\beta_n}$ where $\alpha_1 +
\alpha_2 + \cdots + \alpha_n = \beta_1 + \beta_2 + \cdots + \beta_n$ can be
written in SOS form.

Some special cases are given below:

\begin{enumerate}
  \item $a^2 + b^2 + c^2 - ab - bc - ca = \frac{(a - b)^2 + (b - c)^2 + (c -
    a)^2}{2}$
  \item $a^3 + b^3 + c^3 - 3abc = \frac{a + b + c}{2}\left[(a - b)^2 + (b - c)^2
    + (c - a)^2\right]$
  \item $a^b + b^2c + c^2a - ab^2 - bc^2 - ca^2 = \frac{(a - b)^3 + (b - c)^3 + (c -
    a)^3}{3}$
  \item $a^3 + b^3 + c^3 - a^2b - b^2c - c^2a = \frac{(2a + b)(a - b)^2 + (2b +
    c)(b - c)^2 + (2c + a)(c - a)^2}{3}$
  \item $a^4 + b^4 + c^4 - a^3b - b^3c - c^3b = \\\\\frac{(3a^2 + 2ab + b^2)(a -
    b)^2 + (3b^2 + 2bc + c^2)(b - c)^2 + (3c^2 + 2ca + a^2)(c - a)^2}{4}$
  \item $a^3b + b^3c + c^3a - ab^3 - bc^3 - ca^3 = \frac{a + b + c}{3}\left[(b -
    a^3) + (c - b)^3 + (a - c)^3\right]$
  \item $a^4 + b^4 + c^4 - a^2b^2 - b^2c^2 - c^2a^2 = \frac{(a^2 - b^2)^2 + (b^2
    - c^2)^2 + (c^2 - a^2)^2}{2}$
\end{enumerate}

\begin{theorem}
  Consider two polynomials having the same degree and same number of variables $A$ and $B$. The difference of these two polynomilas
  can be written in SOS form:
  $$\sum_{cyc}a_1^{\alpha_1}a_2^{\alpha_2}\ldots a_n^{\alpha_n} - \sum_{cyc}a_1^{\beta_1}a_2^{\beta_2}\ldots a_n^{\beta_n} = \sum
  P_{ij}(a)(a_i - a_j)^2,$$
  where $\alpha_1 + \alpha_2 + \cdots + \alpha_n = \beta_1 + \beta_2 + \cdots + beta_n = m$ and $a = (a_1, a_2, \ldots,
  a_n)$.
\end{theorem}

\begin{proof}
  We need to prove the following lemma first.
  \begin{lemma}
    If $a = (a_1, a_2, \ldots, a_n)$ and $\alpha_1 + \alpha_2 + \alpha_n = m$, then:
    $$\sum_{cyc}a_1^n - \sum_{cyc}a_1^{\alpha_1}a_2^{\alpha_2}\ldots a_n^{\alpha_n} = \sum P_{ij}(a)(a_i - a_j)^2$$
  \end{lemma}
  We prove this lemma by induction over $k$, which will be the number of elements except $0$ belonging to the set ${\alpha_1,
    \alpha_2, \ldots, \alpha_n}$.

  If $k = 1$, the theorem is obviously true.

  If $k = 2$, the expression becomes $\sum_{cyc}a_1^m - \sum_{a_1}^ta_2^{m - t} = \sum P_{ij}(a)(a_i - a_j)^2$

  We observe that $ta^m + (m - t)b^m - ma^tb^{m - t} = P(a, b)(a - b)^2$. We also observe that $f(x) = tx^n + (m - t) - mx^t = 0$
  has one repeated root which is $1$ because $f(1) = f'(1) = 0$. Therefore $f(x)$ can be written like $Q(x)(x - 1)^2$ where degree
  of $Q$ will be $m - 2$.

  Let $x = \frac{a}{b}$, then we have: $b^mf\left(\frac{a}{b}\right) = ta^m + (m - t)b^m - ma^tb^{m - 1} = b^{m -
    2}Q\left(\frac{a}{b}\right)(a - b)^2$.

  However, $b^{m - 2}$ is a polynomial having $2$ variables $a, b$ because $Q$ is a $m - 2$ degree polynomial. If our proposition
  is already true with $k$, the number of elements except for $0$ in the set of $\alpha$, with $k + 1$ we can transform this into
  the case of $k$ as given below:

  $a_1^{\alpha_1}a_2^{\alpha_2}\ldots a_{k+1}^{\alpha_{k +1}} = \frac{\alpha_1a_1^{\alpha_1 + \alpha_2} + \alpha_2a_2^{\alpha_1 +
      \alpha_2} - (\alpha_1 + \alpha_2)a_1^{\alpha_1}a_2^{\alpha_2}}{\alpha_1 + \alpha_2}.a_3^{\alpha_3}\ldots a_{k+1}^{\alpha_{k
    +1}}\frac{\alpha_1}{\alpha_1 + \alpha_2}a_1^{\alpha_1+\alpha_2}$

  $a_3^{\alpha_3}\ldots a_{k + 1}^{\alpha_{k + 1}} + \frac{\alpha_2}{\alpha_1 + \alpha_2}a_2^{\alpha_1 + \alpha_3}a_3^{\alpha_3}\ldots a_{k
    + 1}^{\alpha_{k +1}}$

  With $k = 2$: $\frac{\alpha_1a_1^{\alpha_1 + \alpha_2} + \alpha_2a_2^{\alpha_1 + \alpha_2} - (\alpha_1 +
    \alpha_2)a_1^{\alpha_1}a_2^{\alpha_2}}{\alpha_1 + \alpha_2} = H_{12}(a)(a_1 - a_2)^2$, we have:

  $a_1^{\alpha_1}a_2^{\alpha_2}\ldots a_{k+1}^{k + 1} = Q_{12}(a)(a_1 - a_2)^2 + \frac{\alpha_1}{\alpha_1 +
    \alpha_2}a_1^{\alpha_1+\alpha_2}a_3^{\alpha_3}\ldots a_{k + 1}^{\alpha_{k + 1}} + \frac{\alpha_2}{\alpha_1 + \alpha_2}a_2^{\alpha_1 +
    \alpha_3}a_3^{\alpha_3}\ldots a_{k + 1}^{\alpha_{k +1}}$

  $\therefore \sum_{cyc}a_1^m - \sum_{cyc}a_1^{\alpha_1}a_2^{\alpha_2}\ldots a_{k + 1}^{\alpha_{k + 1}} = -\sum_{cyc}Q_{12}(a)(a_1
  - a_2)^2 + \sum_{cyc}a_1^m - \frac{\alpha_1}{\alpha_1 +
    \alpha_2}$

  $\sum a_1^{\alpha_1+\alpha_2} a_3^{\alpha_3}\ldots a_{k + 1}^{\alpha_{k + 1}} + \sum\frac{\alpha_2}{\alpha_1 +
    \alpha_2}a_2^{\alpha_1 +\alpha_3}a_3^{\alpha_3}\ldots a_{k + 1}^{\alpha_{k +1}} \sum_{a_1}^m - \sum_{cyc}a_1^{\alpha_1}\ldots
  a_{k +1}^{\alpha_{k +1}}$

  $= -\sum Q_{12}(a)(a_1 - a_2)^2 + \frac{\alpha_1}{\alpha_1 + \alpha_2}\left(\sum_{cyc} a_1^m - \sum_{cyc}a_1^{\alpha_1 +
    \alpha_2}a_3^{\alpha_3}\ldots a_{k+1}^{\alpha_{k+1}}\right) + $

  $\frac{\alpha_2}{\alpha_1 + \alpha_2}\left(\sum_{cyc} a_1^m - \sum_{cyc}a_2^{\alpha_1 + \alpha_2}a_3^{\alpha_3}\ldots a_{k+1}^{\alpha_{k+1}}\right)$

  So we see that these can be written in SOS form recursively. Hence proved.
\end{proof}

\section{Problems}
Prove the following equalities:

\begin{enumerate}
\item $a^2 + b^2 \geq 2ab$.
\item $\sqrt{ab}\geq \frac{2}{\frac{1}{a} + \frac{1}{b}}$, where $a>0, b>0$.
\item $\sqrt{\frac{a^2 + b^2}{2}}\geq \frac{a + b}{2}$
\item $\frac{a + b}{2}\geq \frac{2}{\frac{1}{a} + \frac{1}{b}}$, where $a>0, b>0$.
\item $a + b > 1 + ab$, where $b < 1 < a$.
\item $a^2 + b^2 > c^2 + (a + b - c)^2$, where $b < c< a$.
\item $2\leq \frac{a}{b} + \frac{b}{a}$, where $ab > 0$.
\item $\frac{a}{b} + \frac{b}{a}\leq -2$, where $ab < 0$.
\item $x_1\leq \frac{x_1 + \cdots + x_n}{n}\leq x_n$, where $x_1\leq \cdots\leq x_n$.
\item $\frac{x_1}{y_1}\leq \frac{x_1 + \cdots + x_n}{y_1 + \cdots + y_n}\leq x_n$, where $\frac{x_1}{y_1}\leq\cdots\leq
  \frac{x_n}{y_n}$ and $y_i> 0, i=1, \ldots, n$.
\item $x_1\leq(x_1\ldots x_n)^{\tfrac{1}{n}}\leq x_n$, where $n\geq 2, 0\leq x_1\leq\ldots\leq x_n$.
\item $|a_1| + \cdots + |a_n|\geq |a_1 + a_2 + \cdots + a_n|$.
\item $\frac{a_1 + \cdots + a_n}{n}\geq \frac{n}{\frac{1}{a_1} + \cdots + \frac{1}{a_n}}$, where $n\geq 2, a_i> 0, i=1, \ldots, n$.
\item $a + b\sqrt{\frac{a + b}{2}} \geq a\sqrt{b} + b\sqrt{a}$, where $a > 0, b > 0$.
\item $\frac{1}{2}(a + b) + \frac{1}{4}\geq \sqrt{\frac{a + b}{2}}$, where $a > 0, b > 0$.
\item $a(x + y - a)\geq xy$, where $x\leq a\leq y$.
\item $\frac{1}{x - 1} + \frac{1}{x + 1} > \frac{2}{x}$, where $x > 1$.
\item $\frac{1}{3k + 1} + \frac{1}{3k + 2} + \frac{1}{3k + 3} > \frac{1}{2k + 1} + \frac{1}{2k + 2}$, where $k\in\mathbb{N}$.
\item $\frac{ab}{(a + b)^2}\leq \frac{(1 - a)(1 - b)}{[(1 - a) + (1 - b)]^2}$, where $o<\leq\frac{1}{2}, 0<b\leq\frac{1}{2}$.
\item $\frac{1}{\sqrt{3k + 1}}.\frac{2k + 1}{2k + 2} < \frac{1}{\sqrt{3k + 4}}$, where $k\in\mathbb{N}$.
\item $2^{n - 1}\geq n$, where $n\in\mathbb{N}$.
\item $\frac{1}{3} + \frac{2}{3}.\frac{1}{5} + \frac{2}{3}.\frac{4}{5} + \frac{1}{7} + \cdots +
  \frac{2}{3}.\frac{4}{5}.\frac{6}{7}\cdots \frac{100}{101}.\frac{1}{103} < 1$.
\item $\frac{1 - a}{1 - b} + \frac{1 - b}{1 - a}\leq \frac{a}{b} + \frac{b}{a}$, where $0 < a, b\leq \frac{1}{2}$.
\item $\displaystyle\sum_{i=1}^n\frac{1}{1 - a_i}\sum_{i=1}^m(1 - a_i)\leq \sum_{i=1}^n\frac{1}{a_i}\sum_{i=1}^na_i$, where $0 < a_1, \ldots,
  a_n\leq \frac{1}{2}$.
\item $1 + \frac{1}{2^3} + \cdots + \frac{1}{n^3} < \frac{5}{4}$, where $n\in\mathbb{N}$.
\item $\frac{1}{1 + a + b}\leq 1 - \frac{a + b}{2} + \frac{ab}{3}$, where $0\leq a\leq 1, 0\leq b\leq$.
\item $|x - y| < |1 - xy|$, where $|x| < 1, |y| < 1$.
\item $\frac{a}{bc} + \frac{b}{ca} + \frac{c}{ab}\geq \frac{2}{a} + \frac{2}{b} - \frac{2}{c}$, where $a > 0, b > 0, c > 0$.
\item $\frac{1}{a} + \frac{1}{b} - \frac{1}{c} < \frac{1}{abc}$, where $a^2 + b^2 + c^2 = \frac{5}{3}$ and $a > 0, b > 0, c > 0$.
\item $3(1 + a^2 + a^4)\geq (1 + a + a^2)^2$.
\item $(ac + bd)^2 + (ad - bc)^2\geq 144$, where $a + b = 4, c + d = 6$.
\item $x_1^2 + x_2^2 + \cdots + x_{2n}^2 + na^2 \geq a\sqrt{2}(x_1 + x_2 + \cdots + x_{2n})$.
\item $\frac{1}{a + b} + \frac{1}{b + c} + \frac{1}{a + c}\leq \frac{\sqrt{a} + \sqrt{b} + \sqrt{c}}{2\sqrt{abc}}$, where $a > 0, b
  > 0, c > 0$.
\item $a^3(b^2 - c^2) + b^3(c^2 - a^2) + c^3(a^2 - b^2) < 0$, where $0 < a < b < c$.
\item $\frac{y}{x} + \frac{y}{z} + \frac{x + z}{y}\leq \frac{(x + z)^2}{xz}$, where $0< x\leq y\leq z$.
\item $\sqrt{1 + \sqrt{a}} + \sqrt{1 + \sqrt{a + \sqrt{a^2}}} + \cdots + \sqrt{1 + \sqrt{a + \cdots + \sqrt{a^n}}} < na$ where $n
  \geq 2, a\geq 2, n\in\mathbb{N}$.
\item $[5x]\geq [x] + \frac{[2x]}{2} + \frac{[3x]}{3} + \frac{[4x]}{4} + \frac{[5x]}{5}$, where $[x]$ si the integer part of the
  real number $x$.
\item $(n!)^2 \geq n^n$, where $n\in\mathbb{N}$.
\item $x^6 + x^5 + 4x^4 - 12x^3 + 4x^2 + x + 1\geq 0$.
\item $\log^2\alpha\geq \log\beta\log\gamma$, where $\alpha>1, \beta>1, \gamma>1, \alpha^@\geq\beta\gamma$.
\item $\log_45 + \log_56 + \log_67 + \log_78> 4.4$.
\item $\frac{1}{3} + \frac{2}{3.5} + \cdots + \frac{n}{3.5\ldots(2n + 1)} < \frac{1}{2}$, where $n\in\mathbb{N}$.
\item $\frac{2^3 + 1}{2^3 - 1}\cdots\frac{n^3 + 1}{n^3 - 1}< \frac{3}{2}$, where $n\geq 2, n\in\mathbb{N}$.
\item $1.1! + 2.2! + \cdots + n.n! < (n + 1)!$, where $n\in\mathbb{N}$.
\item $\left(1 + \frac{1}{2^2}\right)\left(1 + \frac{1}{3^2}\right)\cdots\left(1 + \frac{1}{n^2}\right) < 2$, where $n\geq 2,
  n\in\mathbb{B}$.
\item $\left(1 - \frac{1}{p_1^2}\right)\left(1 - \frac{1}{p_2^2}\right)\cdots\left(1 - \frac{1}{p_n^2}\right) > \frac{1}{2}$, where
  $1 < p_1 < p_2 <\cdots < p_n, p_i\in\mathbb{N}, i = 1, 2, \ldots n$.
\item $\frac{1}{2} - \frac{1}{3} + \frac{1}{4} - \frac{1}{5} + \cdots - \frac{1}{999} + \frac{1}{1000}<\frac{2}{5}$.
\item $\frac{a + b}{1 + a + b}\leq \frac{a}{1 + a} + \frac{b}{1 + b}$, where $a\geq 0, b\geq 0$.
\item $\frac{a + b}{2 + a + b}\geq \frac{1}{2}\left(\frac{a}{1 + a} + \frac{b}{1 + b}\right)$, where $a\geq 0, b\geq 0$.
\item $\displaystyle\sum_{i=1}^n\frac{a_1 + 2a_2 + \cdots + ia_i}{i^2}\leq 2\sum_{i=1}^na_i$, where $a_i\geq 0, i=1, 2, \ldots, n$.
\item $\frac{1}{a} + \frac{1}{b} + \frac{1}{c}\leq \frac{41}{42}$, where $\frac{1}{a} + \frac{1}{b} + \frac{1}{c}< 1, a, b,
  c\in\mathbb{N}$.
\item $\frac{4x}{y + z} + \frac{y}{x + z} + \frac{z}{x + y} > 2$, where $x,y,z > 0$.
\item $1 < \frac{a}{a + b + d} + \frac{b}{a + b + c} + \frac{c}{b + c + d} + \frac{d}{a + c + d}< 2$, where $a,b,c,d > 0$.
\item $a + b > c + d$, where $a, b, c, d\geq \frac{1}{2}$ and $a^2 + b > c^2 + d, a + b^2 > c + d^2$.
\item $(b - a)(9 - a^2) + (c - a)(9 - b^2) + (c - b)(9 - c^2)\leq 24\sqrt{2}$, where $0\leq a\leq b\leq c\leq 3$.
\item If $0<a, b, c < 1$, then one of the numbers $(1 - a)b, (1 - b)c, (1 - c)a$ is not greater than $\frac{1}{4}$.
\item Let $a > 0, b > 0, c > 0$, and $a + b + c = 1$. Prove that $\sqrt{a + \frac{1}{4}(b - c)^2} + \sqrt{b + \frac{1}{4}(c - a)^2}
  + \sqrt{c + \frac{1}{4}(b - a)^2}\leq 2$.
\item Let $a > 0, b > 0, c > 0$, and $a + b + c = 1$. Prove that $\sqrt{a + \frac{1}{4}(b - c)^2} + \sqrt{b} + \sqrt{c}\leq
  \sqrt{3}$.
\item Find the smallest possible value of the expression: $\frac{a^4}{b^4} + \frac{b^4}{a^4} - \frac{a^2}{b^2} - \frac{b^2}{a^2} +
  \frac{a}{b} + \frac{b}{a}$, where $a, b > 0$.
\item $\frac{(1 - x_1)(1 - x_2)\ldots(1 - x_n)}{x_1x_2\ldots x_n}\geq (n - 1)^n$, where $n\geq 2, x_i>0,  i = 1, 2, \ldots, n$ and
  $x_1 + x_2 + \cdots + x_n = 1$.
\item $\frac{1}{1 + x_1} + \frac{1}{1 + x_2} + \cdots + \frac{1}{1 + x_n}$, where $n\geq 2, x_1\geq 1, x_2\geq 1, \ldots, x_n\geq
  1$.
\item $abc + bcd + cda + dab\leq \frac{1}{27} + \frac{176}{27}abcd$, where $a, b, c, d\geq 0$, and $a + b + c + d = 1$.
\item $0\leq xy + yz + zx - 2xyz\leq \frac{7}{27}$, where $x, y, z\geq 0$, and $x + y + z = 1$.
\item Suppose that for numbers $x_1, x_2, \ldots, x_{1997}$, the following conditions holds: (a) $-\frac{1}{\sqrt{3}}\leq x_i \leq
  \sqrt{3}, i = 1, 2, \ldots, 1997$, (b) $x_1 + x_2 + \cdots + x_{1997} = -318\sqrt{3}$. Find the greatest possible value of the
  expression $x_1^{12} + x_2^{12} + \cdots + x_{1997}^{12}$.
\item Prove that $\cos\alpha_1\cos\alpha_2\cdots\cos\alpha_n(\tan\alpha_1 + \tan\alpha_2 + \cdots + \tan\alpha_n)\leq \frac{(n-
  1)^{(n - 1)/2}}{n^{(n - 2)/2}}$, where $n\geq 2$ and $0\leq \alpha_i<\frac{\pi}{2}, i = 1, 2, \ldots, n$.
\item Prove that $\displaystyle\sum_{i=1}^nx_i^k(1 - x_i)\leq a_k$, where $k\geq 2, k\in\mathbb{N}$, and $a_k = {\rm max}[x^k(1 -
  x) + (1 - x)^kx], x_i\geq 0, i=1, 2, \ldots, n, x_1 + x_2 + \cdots + x)n = 1, n\geq 2$.
\item $2(n - 1)(x_2x_3 + x_1x_3 + \cdots + x_1x_n + x_2x_3 + \cdots + x_2x_n + \cdots + x_{n - 1}x_n) - n^{n - 1}x_1x_2\ldots
  x_n\leq n - 2$, where, $n\geq 2, x_1, x_2, \ldots x_n\geq 0$ and $x_1 + x_2 + \cdots + x_n = 1$.
\item $\frac{x_1 + x_2 + \cdots + x_n}{n} - \sqrt[n]{x_1x_2\ldots x_n}\leq$

  $\frac{(\sqrt{x_1} - \sqrt{x_2})^2 + (\sqrt{x_1} -
  \sqrt{x_3})^2 + \cdots + (\sqrt{x_1} - \sqrt{x_n})^2 + \cdots + (\sqrt{x_{n - 1}} - \sqrt{x_n})^2}{n}$, where $n\geq 2, x_1, x_2,
  \ldots, x_n\geq 0$.
\item \textit{Turkevici's Inequality}: $(n - 1)(x_1^2 + x_2^2 + \cdots + x_n^2) + \sqrt[n]{x_1^2x_2^2\ldots x_n^2}\geq (x_1 + x_2 +
  \cdots + x_n)^2$, where $n\geq 2, x_2, x_2, \ldots, x_n\geq 0$.
\item $(a + b)(b + c)(c + a)\geq 8abc$, where $a>0, b>0, c>0$.
\item $(a + b + c - d)(b + c + d - a)(c + d + a - b)(d + a + b - c)\leq (a + b)(b + c)(c + d)(d + a)$, where $a > 0, b > 0, c > 0,
  d > 0$.
\item (\textit{Schur's Inequality}) $a^3 + b^3 + c^3 + 3abc \geq a^2b + ab^2 + b^2c + bc^2 + ca^2 + c^2a$, where $a > 0, b > 0, c > 0$.
\item $\left(1 + \frac{4a}{b + c}\right)\left(1 + \frac{4b}{c + a}\right)\left(1 + \frac{4c}{a + b}\right) > 25$, where $a > 0, b >
  0, c > 0$.
\item $\frac{\log(a - 1)}{\log a} < \frac{\log a}{\log(a + 1)}$, where $a > 1$.
\item (\textit{Schur's Inequality}) $abc \geq (a + b - c)(c + a - b)(b + c - a)$, where $a>0, b>0, c>0$.
\item $x^8 + y^8 \geq \frac{1}{128}$, if $x + y = 1$.
\item $\left(a + \frac{1}{a}\right)^2 + \left(b + \frac{1}{b}\right)^2\geq 12.5$, if $a > 0, b > 0$ and $a + b = 1$.
\item $\left(x_1 + \frac{1}{x_1}\right)^2 + \cdots + \left(x_n + \frac{1}{x_2}\right)^2\geq \frac{(n^2 + 1)^2}{n}$, if $n\geq 2,
  x_1> 0, \ldots, x_n>0$ and $x_1 + \cdots + x_n = 1$.
\item $a^4 + b^4 + c^4 \geq abc(a + b + c)$.
\item $x^2 + y^2 \geq 2\sqrt{2}(x - y)$, if $xy = 1$.
\item $\sqrt{6a_1 + 1} + \sqrt{6a_2 + 1} + \sqrt{6a_3 + 1} + \sqrt{6a_4 + 1} + \sqrt{6a_5 + 1}\leq \sqrt{55}$, if $a_1 > 0, \ldots,
  a_5 > 0$ and $a_1 + \cdots + a_5 = 1$.
\item $6a + 4b + 5c\geq 5\sqrt{ab} + 3\sqrt{bc + 7\sqrt{ca}}$, where $a\geq 0, b\geq 0, c\geq 0$.
\item $2(a^4 + b^4) + 17 > 16 ab$.
\item $\left(\frac{1 + nb}{n + 1}\right)^{n + 1}\geq b^n$, where $n\in\mathbb{N}, b > 0$.
\item $\left(1 + \frac{1}{n}\right)^n < \left(1 + \frac{1}{n + 1}\right)^{n + 1}$, where $n\in\mathbb{N}$.
\item $\left(1 + \frac{1}{n}\right)^{n + 1} < \left(1 + \frac{1}{n + 1}\right)^{n + 2}$, where $n\in\mathbb{N}$.
\item $\left(1 + \frac{m}{n - 1}\right)^{(n - 1)/m} < \left(1 + \frac{m}{n}\right)^{n/m} < \left(1 + \frac{m - 1}{n}\right)^{n/(m
  - 1)}$, where $m > 1, n > 1$ and $m,n \in\mathbb{N}$.
\item $n!< \left(\frac{n + 1}{2}\right)^n$, where $n = 2, 3, 4, \ldots$.
\item $n(n + 1)^{1/n} < n + S_n$, where $S_n = \frac{1}{1} + \frac{1}{2} + \cdots + \frac{1}{n}, n = 2, 3, 4, \ldots$.
\item $n - S_n > (n - 1)^{1/(1 - n)}$, where $S_n = \frac{1}{1} + \frac{1}{2} + \cdots + \frac{1}{n}, n = 3, 4, \ldots$.
\item $(q^n - 1)(q^{n + 1} + 1)\geq 2nq^n(q - 1)$, where $q > 1, n \in\mathbb{N}$.
\item $a^2 + b^2 + c^2 + d^2 + ab + ac + ad + bc + bd + cd\geq 10$, where $a, b, c, d>0$, and $abcd = 1$.
\item $\left(a - 1 + \frac{1}{b}\right)\left(b - 1 + \frac{1}{c}\right)\left(c - 1 + \frac{1}{a}\right)\leq \left(\frac{1 +
  abc}{2\sqrt{abc}}\right)^3$, where $a, b, c> 0$.
\item $\left(a + \frac{1}{b} - t\right)\left(b + \frac{1}{c} - t\right)\left(c + \frac{1}{a} - t\right)\leq (a + b +
  c)\left(\frac{1}{a} + \frac{1}{b} + \frac{1}{c}\right)(1 - t)^2 + 4 - 3t$, where $a, b,c, t > 0$ and $abc = 1$.
\item $n\sqrt[n]{a_1a_2\ldots a_n} - (n - 1)\sqrt[n - 1]{a_1a_2\ldots a_{n - 1}} \leq a_n$, where $a_i > 0, i = 1, 2,
  \ldots, n, n = 3, 4, \ldots$.
\item $\sqrt[n]{a_1a_2\ldots a_n} + \sqrt[n]{b_1b_2\ldots b_n} + \cdots + \sqrt[n]{k_1k_2\ldots k_n}$

  $\leq\sqrt[n]{(a_1 + b_1 + \cdots + k_1)(a_2 + b_2 + \cdots + k_2)\cdots(a_n + b_n + \cdots + k_n)}$ where

  $a_1, a_2,\ldots, a_n, b_1, b_2, \ldots, b_n, \ldots, k_1, k_2, \ldots, k_n > 0$.
\item $a_1 + \sqrt{a_1a_2} + \cdots + \sqrt[n]{a_1a_2\ldots a_n}\leq e(a_1 + a_2 + \cdots + a_n)$, where $n\geq 2, a_1, a_2,
  \ldots, a_n\geq 0$.
\item $na^k - ka^n\leq n - 1$, where $n > k, n, k\in\mathbb{N}, a > 0$.
\item $\frac{x_1^2}{x_2} + \frac{x_2^3}{x_3^2} + \cdots + \frac{x_n^{n + 1}}{x_1^n}\geq x_1 + x_2 + \cdots + x_n$, where $n\geq 2,
  n\in\mathbb{N}, x_1 = {\rm min}(x_1, x_2, \ldots, x_n) > 0$.
\item $\frac{a^{x_1 - x_2}}{x_1 + x_2} + \frac{a^{x_2 - x_3}}{x_2 + x_3} + \cdots + \frac{a^{x_n - x_1}}{x_n +
  x_1}\geq \frac{n^2}{2\displaystyle\sum_{i=1}^nx_i}$, where $a > 0, x_i > 0, i = 1, 2, \ldots, n$.
\item $\sqrt[p]{x_1 + 1} + \sqrt[p]{x_2 + 1} + \cdots + \sqrt[p]{x_n + 1}\leq n + 1$, where $n\geq 2, x_1, x_2, x_n > 0, x_1 + x_2
  + \cdots + x_n = p, p\in\mathbb{N}, p\geq 2$.
\item $x^k(1 - x^m)\leq \frac{k^{k/m}.m}{(k + m)^{1 + k/m}}$, where $0\leq x\leq 1, k, m\in\mathbb{N}$.
\item $\frac{x}{1 - x^2} + \frac{y}{1 - y^2} + \frac{z}{1 - z^2}\geq \frac{3\sqrt{3}}{2}$, where $x, y, z > 0$ and $x^2 + y62 + z^2
  = 1$.
\item $\frac{1}{1 - x} + \frac{1}{1 - y} + \frac{1}{1 - z}\geq \frac{9 + 3\sqrt{3}}{2}$, where $x, y, z > 0$ and $x^2 + y62 + z^2
  = 1$.
\item Find the minimum value of the funciton $f(x) = \frac{1}{\sqrt[n]{1 + x}} + \frac{1}{\sqrt[n]{1 - x}}$ in $[0, 1)$, where
  $n\in\mathbb{N}, n > 1$.
\item Find the minimum value of the funciton $f(x) = ax^m + \frac{b}{x^n}$ in $(0, \infty)$, where $a, b > 0, m, n\in\mathbb{N}$.
\item Find in $[a, b] (0 < a < b)$ a point $x_0$ such that the function $f(x) = (x - a)^2(b^2 - x^2)$ attains its maximum value in
  $\lfloor a, b\rfloor$ at $x_0$.
\item Find the greatest possible value of the product $xyz$ given $x, y, z > 0$, and $2x + \sqrt{3}y + \pi z = 1$.
\item Find the maximum and minimum values of the function $y = \frac{x}{ax^2 + b}$, where $a, b > 0$.
\item Find the maximum value of the function $y = \frac{5\sqrt{x^2 + 6x + 8} + 12}{x + 3}$.
\item Find the maximum value of the function $y = \frac{\sqrt[3]{(x^2 + 1)^2(x^2 + 3)}}{3x^3 + 4}$.
\item Solve the system of equations: $x + y = 2, xy - z^2 = 1$.
\item Solve the system of equations: $x + y + z = 3, x^2 + y^2 + z^2 = 3$.
\item Given $a + b + c + d + e = 8, a^2 + b^2 + c^2 + d^2 + e^2 = 16$, find the greatest possible value of $e$.
\item Find the minimum value of the expression $\frac{x_1}{x_2} + \frac{x_3}{x_4} + \frac{x_5}{x_6}$ if $1\leq x_1\leq x_2\leq
  x_3\leq x_4\leq x_5\leq x_6\leq 1000$.
\item Solve the equation $x^4 + y^4 + 2 = 4xy$.
\item Find all integer solutions of the equation $\frac{xy}{z} + \frac{zx}{y} + \frac{yz}{x} = 3$.
\item Prove that $x_1^\alpha + \cdots + x_n^\alpha\geq x_1^\beta + \cdots + x_n^\beta$, where $n\geq 2, x_1 > 0, \ldots, x_n > 0,
  \alpha >\beta\geq 0$, and $x_1\ldots x_n = 1$.
\item Prove that $x_1^\alpha + \cdots + x_n^\alpha\geq x_1^\beta + \cdots + x_n^\beta$, where $n\geq 2, x_1 > 0, \ldots, x_n > 0,
  \alpha >(n - 1)|\beta|$, and $x_1\ldots x_n = 1$.
\item Prove that $2ST > \sqrt{3(S + T)[S(bd + df + fb) + T(ac + ce + ea)]}$, where $0 < a < b < c < d < e < f$ and $a + c + e = S,
  b + d + f = T$.
\item Prove that $\frac{a + \sqrt{ab} + \sqrt[3]{abc} + \sqrt[4]{abcd}}{4} \leq \sqrt[4]{a . \frac{a + b}{2} . \frac{a + b + c}{3}
  . \frac{a + b + c + d}{4}}$, where $a > 0, b > 0, c > 0, d > 0$.
\item Prove that $ab \leq \frac{a^p}{p} + \frac{b^q}{q}$, if $\frac{1}{p} + \frac{1}{q} = 1, a > 0, b > 0, p > 0, q > 0$ where $p$
  and $q$ are rational numbers.
\item Prove that $\left(1 + \frac{1}{n}\right)^n > 2$, where $n\in \mathbb{N}$.
\item Prove that $(1 + a_1)\cdots(1 + a_n)\leq 1 + \frac{S}{1!} + \cdots + \frac{S^n}{n!}$, where $n\geq 2, S = a_1 + \cdots + a_n,
  a_i > 0, i = 1, \ldots, n$.
\item Prove that $\left(1 + \frac{1}{a}\right)\left(1 + \frac{1}{b}\right)\left(1 + \frac{1}{c}\right)\geq 64$, where $a > 0, b >
  0, c > 0$ and $a + b + c = 1$.
\item Prove that $\sqrt[n]{a^{2n - k}} + \sqrt[n]{a^{2n + k}}\geq 3a - 1$, where $n\geq 2, a > 0, n > k, n, k\in\mathbb{N}$.
\item Prove that $\frac{a^n - 1}{a^n(a - 1)}\geq n + 1 - a^{\tfrac{n(n + 1)}{2}}$, where $a > 0, a \neq 1$.
\item Prove that $na^{n + 1} + 1 \geq (n + 1)a^n$, where $a > 0$.
\item Prove that $\left(\sqrt{k} + \sqrt{k + 1}\right)\left(\sqrt{k + 1} + \sqrt{k + 2}\right)\cdots\left(\sqrt{n} + \sqrt{n +
  1}\right)\geq\left(\sqrt{n} - \sqrt{k}\right)\left(\sqrt{n} + \sqrt{k} - 1\right) + 2$, where $n > k, n, k\in\mathbb{N}$.
\item Prove that $\frac{a_1}{a_2} + \cdots + \frac{a_{n - 1}}{a_n} + \frac{a_n}{a_1}\geq n$, where $a_i > 0, i = 1, 2, \ldots, n$.
\item Prove that $a_{n + 1} + \frac{1}{a_1(a_2 - a_1)(a_3 - a_2)\ldots(a_{n + 1} - a_n)}\geq n + 2$, where $0 < a_k < a_{k + 1}, k
  = 1, 2, \ldots, n$.
\item Prove that $1 + \frac{x}{2}\leq\frac{1}{\sqrt{1 - x}}$, where $0\leq x < 1$.
\item Prove that $\left(\frac{a}{b}\right)^4 + \left(\frac{b}{c}\right)^4 + \left(\frac{c}{d}\right)^4 + \left(\frac{d}{e}\right)^4
  + \left(\frac{e}{a}\right)^4 \geq \frac{a}{b} + \frac{b}{c} + \frac{c}{d} + \frac{d}{e} + \frac{e}{a}$, where $abcde\neq 0$.
\item Prove that $\left(\frac{a}{b}\right)^{1999} + \left(\frac{b}{c}\right)^{1999} + \left(\frac{c}{d}\right)^{1999} +
  \left(\frac{d}{a}\right)^{1999}\geq \frac{a}{b} + \frac{b}{c} + \frac{c}{d} + \frac{d}{a}$, where $a > 0, b > 0, c > 0, d > 0$.
\item Prove that
\end{enumerate}
