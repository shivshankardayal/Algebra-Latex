\chapter{Inequalities}
Ineuqlities come up in different branches of mathematics; for example in algebra, trigonometry and trigonometry. They are very
useful in establishing many relations among various quantities. Certain inequalities are very useful in studying properties of
many common expressions which lead to interesting observations. In this chapter we will only study algebraic inequalitites. The
problems given are quite basic and simple. We start with some useful theorems for these inequalities.

\section{Strum's Method}
Strum's method is given by the German mathematician {\sl Friedrich Otto Rudolf Sturm}. Sturm's method helps prove a large number of
different inequalities under certain conditions along with various other applications.

\begin{theorem}
\label{th:strum:1}
Prove that if the product of positive numbers $x_1, x_2, \cdots, x_n (n\geq 2)$ is euqal to $1$, then $x_1 + x_2 + \cdots + x_n\geq n$.
\end{theorem}
\begin{proof}
  If $x_1 = \cdots = x_n$, then $x_1 + \cdots + x_n = n$. So we see that the statement is true if all the numbers are equal and are
  unity. Now we consider the case when at least two numbers are different such that one is greater than $1$ and the other one is
  smaller. Let us assume that these are $x_1$ and $x_2$ which does not cause loss of generality, and that $x_1 < 1 < x_2$. Note
  that $x_1 + x_2 > 1 + x_1x_2 [\because (1 - x_1)(x2 - 1) > 0]$. If given numbers are substitued by $1, x_1x_2, x_3, \ldots, x_n$,
  then the product is equal to $1$ and $1 + x_1x_2 + x_3 + \ldots + x_n < x_1 + x_2 + \ldots + x_n$. Repeating this we will find $n
  - 1$ numbers equal to $1$ and the $n$th number equal to $x_1x_2\ldots x_n$. Thus, $x_1 + x_2 + \ldots + x_n < 1$. We see that
  equality holds if and only if $x_1 = x_2 = \cdots = x_n = 1$
\end{proof}

\begin{theorem}
\label{th:strum:2}
  Prove that if the sum of the numbers $x_1, x_2, \ldots, x_n (n\geq 2)$ is equal to $1$, then prove that $x_1^2 + x_2^2 + \ldots +
  x_n^2 \geq \frac{1}{n}$.
\end{theorem}
\begin{proof}
  If $x_1 = x_2 = \ldots = x_n = \frac{1}{n}$ then $x_1^2 + x_2^2 + \ldots + x_n^2 = \frac{1}{n}$. Like previous theorem we
  consider two numbers $x_1$ and $x_2$ such that one of them is greater than $\frac{1}{n}$ while the other is smaller than
  $\frac{1}{n}$. Assume that these two numbers are $x_1$ and $x_2$, which does not cause loss of generality, and that $x_1 <
  \frac{1}{n}$ and $x_2 > \frac{1}{n}$. So we obtain a sequence of numbers $\frac{1}{n}, x_1 + x_2 - \frac{1}{n}, x_3, \ldots, x_n$
  suhc that their sum remains equal to $1$. We can easily prove that $x_1^2 + x_2^2 > \frac{1}{n^2} + \left(x_1 + x_2 -
  \frac{1}{n}\right)^2$, and hence $$x_1^2 + x_2^2 + \ldots + x_n^2 > \frac{1}{n^2} + \left(x_1 + x_2 - \frac{1}{n}\right)^2 +
  x_3^2 + \ldots + x_n^2.$$

  Repeating this we obtain a sequence in which all terms will be equal to $\frac{1}{n}$, and sum of their square is less than the
  sum of squares of numbers $x_1, x_2, \ldots, x_n$ i.e. $x_1^2 + x_2^2 + \ldots + x_n^2 > \frac{1}{n^2} +
  \text{~to~}n\text{~times}.$ From this it follows that equality holds if and only if $x_1 = x_2 = \ldots = x_n$.
\end{proof}

\section{A.M., G.M., H.M. and Q.M.}
\begin{theorem}
  \textbf{\rm\bf (A.M.-- G.M. -- H.M. -- Q.M. Inequality)} Let $x_1, x_2, \ldots, x_n$ be positive real numbers, then
  $$\frac{n}{\frac{1}{x_1} + \frac{1}{x_2} + \cdots + \frac{1}{x_n}} \leq \sqrt[n]{x_1x_2 \ldots x_n}\leq \frac{x_1 + x_2 + \cdots
    + x_n}{n}\leq \sqrt{\frac{x_1^2 + x_2^2 + \cdots + x_n^2}{n}}.$$
\end{theorem}

\begin{proof}
  Consider the numbers $\frac{x_1}{\sqrt[n]{x_1x_2\ldots x_n}}, \frac{x_2}{\sqrt[n]{x_1x_2\ldots x_n}}, \cdots,
  \frac{x_n}{\sqrt[n]{x_1x_2\ldots x_n}}$, we see that product is equal to $1$. From theorem \ref{th:strum:1}, we have that
  $$\frac{x_1}{\sqrt[n]{x_1x_2\ldots x_n}} + \frac{x_2}{\sqrt[n]{x_1x_2\ldots x_n}} + \cdots + \frac{x_n}{\sqrt[n]{x_1x_2\ldots
      x_n}}\geq n \Rightarrow \frac{x_1 + x_2 + \ldots + x_n}{n}\geq \sqrt[n]{x_1x_2\ldots x_n}.$$
  The above inequality is also known as Cauchy's inequality.

  In the above inequality, if we substitute $x_i = \frac{1}{x_i}$, then $$\frac{n}{\frac{1}{x_1} + \frac{1}{x_2} + \cdots +
    \frac{1}{x_n}} \leq \sqrt[n]{x_1x_2 \ldots x_n}.$$

  Consider the numbers $\frac{x_1}{x_1 + x_2 + \cdots + x_n}, \frac{x_2}{x_1 + x_2 + \cdots + x_n}, \cdots, \frac{x_n}{x_1 + x_2 +
    \cdots + x_n}$, and note that their sum is equal to $1$. According to theorem \ref{th:strum:2}, we have $$\left(\frac{x_1}{x_1 +
    x_2 + \cdots + x_n}\right)^2 + \left(\frac{x_2}{x_1 + x_2 + \cdots + x_n}\right)^2 + \cdots + \left(\frac{x_n}{x_1 + x_2 +
    \cdots + x_n}\right)^2 \geq \frac{1}{n}$$
  $$\Rightarrow \frac{x_1^2 + x_2^2 + \ldots + x_n^2}{n}\geq \left(\frac{x_1 + x_2 + \ldots + x_n}{n}\right)^2.$$

  Hence, all the inequalities have been proven.
\end{proof}

\section{Cauchy–Bunyakovsky–Schwarz Inequality}
\begin{theorem}
  {\rm \bf (Cauchy–Bunyakovsky–Schwarz Inequality)} Let $a_1, a_2, \ldots, a_n,$ $b_1, b_2, \ldots, b_n \in R$. Then
  $$(a_1^2 + a_2^2 + \ldots + a_n^2)(b_1^2 + b_2^2 + \ldots + b_n)^2 \geq (a_1b_1 + a_2b_2 + \ldots a_nb_n)^2$$
\end{theorem}

\begin{proof}
  Let $x_k = \sqrt{(a_1^2 + a_2^2 + \cdots + a_k^2)(b_1^2 + b_2^2 + \cdots + b_k^2)}$, where $k = 1, 2, \ldots, n$.
  In this case,

  $$x_{k + 1} = \sqrt{(a_1^2 + a_2^2 + \cdots + a_k^2 + a_{k + 1}^2)(b_1^2 + b_2^2 + \cdots + b_k^2 + b_{k + 1}^2)}$$
  $$\sqrt{\left[\left(\sqrt{a_1^2 + a_2^2 + \ldots + a_k^2}\right)^2 + a_{k + 1}^2\right]\left[\left(\sqrt{b_1^2 + b_2^2 + \ldots +
        b_k}\right)^2 + b_{k + 1}^2\right]}$$
  $$\geq \sqrt{\left(\sqrt{a_1^2 + a_2^2 + \ldots + a_k^2}.\sqrt{b_1^2 + b_2^2 + \ldots + b_k^2} + a_{k + 1}.b_{k + 1}\right)^2} =
  x_k + a_{k + 1}b_{k + 1}$$
  {\it Alternative Proof.}
  $$(a_1^2 + a_2^2 + \ldots + a_n^2)(b_1^2 + b_2^2 + \ldots + b_n)^2 - (a_1b_1 + a_2b_2 + \ldots a_nb_n)^2 =
  \sum_{i,j=1\\i\geq j}^n(a_ib_j - b_ja_i)^2\geq 0.$$
\end{proof}

\subsection{Titu's Lemma}
\begin{lemma}
  Let $a_1, a_2, \ldots, a_n, b_1, b_2, \ldots, b_n$ be positive real numbers then $$\frac{a_1^2}{b_1} + \frac{a_2^2}{b_2} + \ldots
  + \frac{a_n^2}{b_n} \geq \frac{(a_1 + a_2 + \ldots + a_n)^2}{b_1 + b_2 + \ldots + b_n}$$
\end{lemma}

\begin{proof}
  This is a direct consequence of \textit{Cauchy–Bunyakovsky–Schwarz Inequality}. It is obtained by substituting $a_i =
  \frac{x_i}{\sqrt{y_i}}$ and $b_i = \sqrt{y_i}$ into Cauchy–Bunyakovsky–Schwarz Inequality. Equality holds if and only if $a_i =
  kb_i$ for a non-zero real constant $k$.
\end{proof}

\section{Young's Inequality}
\begin{theorem}
  If $p\in [1,\infty)$ and $q = p/(p - 1)$. $q\in [1, \infty]$ and $\dfrac{1}{p} + \frac{1}{q} = 1$. If $a, b > 0$,
    then $$\frac{a^p}{p} + \frac{b^q}{q}\geq ab$$
\end{theorem}

\begin{proof}
  Taking $\log$ of L.H.S. $\log\left(\frac{a^p}{p} + \frac{b^q}{q}\right)$

  Notice that, since $\dfrac{1}{p} + \frac{1}{q} = 1$, so the L.H.S. is just a convext combination of $a^p$ and $b^q$. Since $\log
  x$ is a concave function, we have
  $$\log\left(\frac{a^p}{p} + \frac{b^q}{q}\right)\geq \dfrac{\log a^p}{p} + \dfrac{b^q}{q} = \log a + \log b = \log(ab).$$
  Hence, the inequality is proved(since $\log x$ is strictly increasing).

  {\it Alternative Proof.}

  Using generalized AM-GM inequality, $$\dfrac{x^p}{p} + \frac{b^q}{q}\geq \left[(x^p)^{1/p}(y^q)^{1/q}\right] = xy.$$
\end{proof}
