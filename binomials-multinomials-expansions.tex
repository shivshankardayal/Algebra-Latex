\chapter{Binomials, Multinomials and Expansions}
An algebraic expression containing one term is called \textit{monomial}, two terms is callled \textit{binomial} and more than two
is called is called \textit{multinomial}. Examples of a monomial expressions are $2x, 4y$, examples of binomial expressions are $a
+ b, x^2 + y^2, x^3 + y^3, x + \frac{1}{y}$ and exaamples of multinomial expressions are $1 + x + x^2, a^2 + 2a + b^2, a^3 + 3a^2b
+ 3ab^2 + b^3$.

\section{Binomial Theorem}
Newton gave binomial theorem, by which we can expand any opwer of a binomial expression as a series. First we consider only
positive integral values of exponent. For positive integral exponent the formula has the following form:
$$(a + x)^n = {}^nC_0a^nx^0 + {}^nC_1a^{n - 1}x^1 + {}^nC_2a^{n - 2}x^2 + \ldots + {}^nC_na^0x^n$$

\subsection{Proof by Mathematical Induction}
Let $$P(n) = (a + x)^n = {}^nC_0a^nx^0 + {}^nC_1a^{n - 1}x^1 + {}^nC_2a^{n - 2}x^2 + \ldots + {}^nC_na^0x^n$$

When $n = 1, P(1) = a + x ={}^1C_0a + {}^1C_1x$. When $n = 2, P(2) = a^2 + 2ax + x^2 = {}^2C_0a^2 + {}^2C_1ax + {}^2C_2x^2$. Thus
we see that $P(n)$ holds good for $n = 1$ and $n = 2$. Let $P(n)$ is true for $n = k$ i.e.
$$P(k) = (a + x)^k = {}^kC_0a^kx^0 + {}^kC_1a^{k - 1}x^1 + {}^kC_2a^{k - 2}x^2 + \ldots + {}^kC_ka^0x^k$$
Multiplying both sides with $(a + x)$
$$P(k + 1) = (a + x)^{k + 1} = {}^kC_0a^{k + 1}x^0 + {}^kC_1a^kx + {}^kC_2a^{k - 1}x^2 + \ldots + {}^kC_kax^k +$$ $${}^kC_0a^kx +
{}^kC_1a^{k - 1}x^2 + {}^kC_2a^{k - 2}x^3 + \ldots + {}^kC_kx^{k + 1}$$

Combining terms with equal powers of $a$ and $x$, using the formula ${}^nC_r + {}^nC_{r + 1} = {}^{n + 1}C_{r + 1}$ and rewriting
${}^kC_0$ and ${}^kC_k$ as ${}^{k + 1}C_0$ and ${}^{k + 1}C_{k + 1}$, we get
$$P(k + 1) = {}^{k + 1}C_0a^{k + 1}x^0 + {}^{k + 1}C_1a^{k}x^1 + {}^{k + 1}C_2a^{k - 1}x^2 + \cdots + {}^{k + 1}C_{k + 1}a^0x^{k + 1}$$
Thus, we see that $P(n)$ holds good for $n = k + 1$ and we have proven binomial theorem by mathemtical induction.

\subsection{Proof by Combination}
We know that $(a + x)^n = (a + x)(a + x)\cdots[n~\text{factors}]$. If see only $a$, then we see that $a^n$ exists and hence, $a^n$
is a term in the final product. This is the term $a^n$, which can be written as ${}^nC_0a^nx^0$. If we take the letter $a, n - 1$
times and $x$ once then we observe ttat $x$ can be taken in ${}^nC_1$ ways. Thus, we can say that the term in final product is
${}^nC_1a^{n - 1}x$. Similarly, if we choose $a, n - 2$ times and $x$ twice then the term will be ${}^nC_2a^{n - 2}x^2$. Finally,
like $a^n, x^n$ will exist and can be written as ${}^nC_nx^n$ for consistency. Thus, we have proven binomial theorem by
combination.

\section{Special Forms of Binomial Expansion}
We have
\begin{equation}
(a + x)^n = {}^nC_0a^nx^0 + {}^nC_1a^{n - 1}x^1 + {}^nC_2a^{n - 2}x^2 + \ldots + {}^nC_na^0x^n
\label{eq:be1}
\end{equation}

\begin{enumerate}
\item Putting $-x$ instead of $x$
  $$(a - x)^n = {}^nC_0a^nx^0 - {}^nC_1a^{n - 1}x^1 + {}^nC_2a^{n - 2}x^2 - \ldots + (-1)^n{}^nC_na^0x^n$$
\item Putting $a = 1$ in eq.\ \ref{eq:be1}
  $$(1 + x)^n = {}^nC_0 + {}^nC_1x + {}^nC_2x^2 + \ldots + {}^nC_nx^n$$
\item Putting $x = -x$ in above equation
  $$(1 - x)^n = {}^nC_0 - {}^nC_1x + {}^nC_2x^2 - \ldots + (-1)^n{}^nC_nx^n$$
\end{enumerate}

\section{General Term of a Binomial Expansion}
We see that first term is $t_1 = {}^nC_0a^nx^0$, second term is $t_2 = {}^nC_1a^{n - 1}x^1$ so general term will be $$t_r =
{}^nC_{r - 1}a^{n - r + 1}x^{r - 1}$$

\section{Middle Term of a Binomial Expansion}
When $n$ is an even number, i.e. $n = 2m,~m\in\mathbb{P}$. Middle term will be $m + 1$th term i.e. $$t_{m + 1} =
{}^nC_ma^mmx^m$$. When $n$ an odd number, i.e. $n = 2m + 1~m\in\mathbb{N}$. There will be two middle terms i.e. $m + 1$th and $m +
2$th terms will be middle terms. So $$t_{m + 1} = {}^nC_ma^{m +1}x^m, t_{m + 2} = {}^nC_{m + 1}a^{m}x^{m + 1}$$

The middle terms have the largest coefficient. In case of two middle terms the coefficients of both the middle terms are equal.

\section{Equidistant Coefficients}
Binomial coefficients equidistant from start and end are equal. Coefficients of first term from start and end are ${}^nC_0$ and
${}^nC_n$ which are equal. Coefficients of second term from start and end are ${}^nC_1$ and ${}^nC_{n-1}$ which are
equal. Similarly, coefficient of $r$th term from start is ${}^nC_{r - 1}$ and from end is ${}^nC_{n - r +1}$. From combinations we
know that ${}^nC_{r - 1} = {}^nC_{n - r + 1}$. Thus, it is prove that coefficients of terms equidistant from start and end are
equal.

\section{Properties of Binomial Coefficients}
We have proven earlier that $$(1 + x)^n = {}^nC_0 + {}^nC_1x + {}^nC_2x^2 + \cdots + {}^nC_nx^n.$$ Putting $x = 1$, we get $$2^n =
{}^nC_0 + {}^nC_1 + {}^nC_2 + \cdots + {}^nC_n.$$ Putting $x = -1$, we get $$0 = {}^nC_0 - {}^nC_1 + {}^nC_2 - \cdots
+(-1)^n{}^nC_n.$$ Adding the last two, we have $$2^n = 2[{}^nC_0 + {}^nC_2 + {}^nC_4 + \cdots]$$ $$2^{n - 1} \ {}^nC_0 + {}^nC_1 +
{}^nC_2 + \cdots$$ Subtracting, we get $$2^{n - 1} = {}^nC_1 + {}^nC_3 + {}^nC_5 + \cdots$$

\section{Multinomial Theorem}
Consider the multinomila $(x_1 + x_2 + \cdots + x_n)^p$, where $n$ and $p$ are positive integers. The general term of such a
multinomial is givenby $$\frac{p!}{p_1!p_2!\ldots p_n!}x_1^{p_1}x_2^{p_2}\cdots x_n^{p_n}$$ such that $p_1, p_2, \ldots, p_n$ are
nonnegative integers and $p_1 + p_2 + \cdots + p_n = p.$

We can find the general term using the binomial theorem itself. General term in the expansion $[x_1 + (x_2 + x_3 + \cdots +
  x_n)]^n$ is $$\frac{n!}{p!(n - p_1)!}x_1^{p_1}(x_2 + x_3 + \cdots + x_n)^{n- p_1}.$$ General term in expansion of $(x_2 + x_3 +
\cdots + x_n)^{n - p_1}$ is $$\frac{(n - p_1)!}{p_2!(n - p_1 - p_2)!}x_2^{p_2}(x_3 + x_4 + \cdots + x_n)^{n - p_1 -p_2}.$$
Proceding in this manner we obtain the general term given above.

\subsection{Som Results on Multinomial Expansions}
\begin{enumerate}
\item No.\ of terms in the multinomial $(x_1 + x_2 + \cdots + x_n)^p$ is number of nonnegative integral solution of the equation
  $p_1 + p_2 + \cdots + p_n = p$ i.e. ${}^{n + p - 1}C_{p}$ or ${}^{n + p - 1}C_{n - 1}$.
\item Largest coeff. in $(x_1 + x_2 + \cdots + x_n)^p$ is $\frac{n!}{(q!)^{n - r}[(q + 1)!]^r}$, where $q$ is the quotient and $r$
  is the remainder of $p/n$.
\item Coefficient of $x^r$ in $(a_0 + a_1x + a_2x^2 + \cdots + a_nx^n)^p$ is $\sum\frac{n!}{p_0!p_1!p_2!\ldots
  p_n!}a_0^{p_0}a_1^{p_1}a_n^{p_n}$ where $p_0, p_1, \cdots, p_n$ are nonnegative integers satisfying the equation $p_0 + p_1 +
  \ldots + p_n = n$ and $p_1 + 2p_2 + \cdots + np_n = r.$
\end{enumerate}

\section{Binomial Theorem for Any Index}
\subsection{Fractional Index}
Let $f(m) = (1 + x)^m = 1 + mx + \frac{m(m- 1)}{1.2}x^2 + \frac{m(m - 1)(m - 2)}{1.2.3}x^3 + \cdots$, where $m\in R$ then, $f(n) =
(1 + x)^n = 1 + nx + \frac{n(n - 1)}{1.2}x^2 + \frac{n(n - 1)(n - 2)}{1.2.3}x^3 + \cdots$

$$f(m)f(n) = (1 + x)^{m + n} = f(m + n)$$
$$f(m)f(n)\ldots\text{~to~}k\text{~factos~}=f(m + n + \ldots)\text{~to~}k\text{~terms~}$$

Let $m, n, \ldots$ each equal to $\frac{j}{k}$

$$\Rightarrow \left[f\left(\frac{j}{k}\right)\right]^k = f(j)$$

but $j$ is a positive integer, $f(j) = (1 + x)^j$

$$\therefore (1 + x)^{\frac{j}{k}} = f\left(\frac{j}{k}\right)$$

$$\therefore (1 + x)^{\tfrac{j}{k}} = 1 + \frac{j}{k}x + \frac{\frac{j}{k}\left(\frac{j}{k} - 1\right)}{1.2}x^2 + \ldots$$

And thus, we have proven binomial theorem for fractional index.

\subsection{Negative Index}
We can write $$f(n)f(-n) = f(0) = 1$$
$$\Rightarrow f(-n) = \frac{1}{f(n)} = (1 + x)^{-n} = 1 - nx + \frac{n(n - 1)}{1.2}x^2 - ldots$$

\section{General Term in Binomial Theorem for Any Index}
General term is given by $$\frac{n.(n - 1)\ldots(n - r + 1)}{r!}x^r$$

The above expansion does not hold true when $|x| > 1$ which can be quickly proved by making $r$ arbitrarily large. For example, $(1
- x)^{-1} = 1 + x + x^2 + x^3 + \ldots$. However, if we put $x = 2$, then we have $(-1)^{-1} = 1 + 2 + 2^2 + \ldots$ which shows
that when $x > 1$ the above formula does not hold true.

From G.P. we know that $1 + x + x^2 + \ldots$ for $r$ terms is $$\frac{1}{1 - x} - \frac{x^r}{1 - x}$$

Thus, if $r$ is very large and $|x| < 1$, we can ignore the second fraction but not when $|x| > 1$.

\section{General Term for Negative Index}
The $r + 1$th term is given by $$\frac{-n(-n - 1)\ldots(-n - r + 1)}{r!}(-x)^r$$
$$= \frac{n(n + 1)\ldots(n + r - 1)}{r!}x^r$$

\section{Exponential and Logrithmic Series Expansions}
Following expansions are useful for solving problem related to exponential and logarithmic series:
\begin{enumerate}
\item $e^x = 1 + \frac{x}{1!} + \frac{x^2}{2!} + \frac{x^3}{3!} + \ldots$ to $\infty$, where $x$ is any number. $e$ lies between
  $2$ and $3$.
\item If $a > 0, a^x = e^{x\log_ea} = 1 + \frac{x\log_ea}{1!} + \frac{(x\log_ea)^2}{2!} + \ldots$
\item $\log_e(1 + x) = x - \frac{x^2}{2} + \frac{x^3}{3} - \frac{x^4}{4} + \ldots$ to $\infty$ where $-1< x\leq 1$.
\end{enumerate}

\section{Problems}

\begin{enumerate}
\item Expand $\left(x + \frac{1}{x}\right)^5$.
\item Use the bonimial theorem to find the exact value of $(10.1)^5$.
\item Simplify $(x + \sqrt{x - 1})^6 + (x - \sqrt{x - 1})^6$.
\item If $A$ be the sum of odd terms and $B$ be the sum of even terms in the expansion of $(x + a)^n$, prove that $A^2 - B^2 = (x^2
  - a^2)^n$.
\item If $n$ is a positive integer, prove that the integral part of $(7 + 4\sqrt{3})^n$ is an odd number.
\item If $(7 + 4\sqrt{3})^n = \alpha + beta$, where $\alpha$ is a positive integer and $\beta$ is a proper fraction, then prove
  that $(1 - \beta)(\alpha + \beta) = 1$.
\item Find the coefficient of $\frac{1}{y^2}$ in $\left(y + \frac{c^3}{y^2}\right)^{10}$.
\item Find the coefficient in $(1 + 3x + 3x^2 + x^3)^{15}$.
\item Find the term independent of $x$ in $\left(\frac{3}{2}x^2 - \frac{1}{3x}\right)^9$.
\item Find the term independent of $x$ in $(1 + x)^m\left(x + \frac{1}{x}\right)^n$.
\item Find the coefficient of $x^{-1}$ in $(1 + 3x^2 + x^4)\left(x + \frac{1}{x}\right)^n$.
\item If $a_r$ denotes the coefficient of $x^r$ in the expansion $(1 - x)^{2n - 1}$, then prove that $a_{r - 1} + a_{2n - r} = 0$.
\item Find the vallue of $k$ so that the term independent of $x$ in $\left(\sqrt{x} + \frac{k}{x^2}\right)^{10}$ is $405$.
\item Show that there will be no term containing $x^{2r}$ in the expansion $(x + x^{-2})^{n - 3}$, if $n - 2r$ is positive but not
  a multiple of $3$.
\item Show that there will be a term independent of $x$ in the expansion $(x^a + x^{-b})^n$, only if $an$ is a multiple of $a + b$.
\item Expand $\left(x + \frac{1}{x}\right)^7$ using binomial theorem.
\item Use binomial theorem to expand $\left(\frac{2x}{3} - \frac{3}{2x}\right)^6$.
\item If $(1 + ax)^n = 1 + 8x + 24x^2 + \ldots$, find $a$ and $n$.
\item Find the $7$th term in the expansion of $\left(\frac{4x}{5} - \frac{5}{2x}\right)^9$.
\item Find the value of $(\sqrt{2} + 1)^6 + (\sqrt{2} - 1)^6$.
\item Evaluate $(0.99)^{15}$ correct to four decimal places using binomial theorem.
\item Evaluate $999^3$ using binomial theorem.
\item Evalaute $(0.99)^{10}$ correct to four decimal places usinng binomial theorem.
\item Find the value of $(1.01)^{10} + (0.99)^{10}$ correct to $7$ decimal places.
\item If $A$ be the sum of the odd terms and $B$ be the sum of the even terms in the expansion $(x + a)^n$, show that $4AB = (x +
  a)^{2n} - (x - a)^{2n}$.
\item If $n$ be a positive integer, prove that the integral part of $(5 + 2\sqrt{6})^n$ is an odd integer.
\item If $(3 + \sqrt{8})^n = \alpha + \beta$, where $\alpha, n$ are positive integers and $\beta$ is a proper fraction, then prove
  that $(1 - \beta)(\alpha + \beta) = 1$.
\item Find the coefficient of $x$ in the expansion of $\left(2x - \frac{3}{x}\right)^9$.
\item Find the coefficient of $x^7$ in the expansion of $(3x^2 + 5x^{-1})^{11}$.
\item Find the coefficient of $x^9$ in the expansion of $(2x^2 - x^{-1})^{20}$.
\item Find the coefficient of $x^{24}$ in the expansion of $(x^2 + 3ax^{-1})^{15}$.
\item Find the coefficient of $x^9$ in the expansion of $(x^2 - 3x^{-1})^9$.
\item Find the coefficient of $x^{-7}$ in the expansion of $\left(2x - \frac{1}{3x^2}\right)^{11}$.
\item Find the coefficient of $x^7$ in the expansion of $\left(ax^2 + \frac{1}{bx}\right)^{11}$ and the coefficient of $x^{-7}$ in
  the expansion of $\left(ax - \frac{1}{bx}\right)^{11}$. Also, find the relation between $a$ and $b$ so that the coefficients are
  equal.
\item If $x^p$ occurs in the expansion of $\left(x^2 + \frac{1}{x}\right)^{2n}$, show that its coefficient is
  $\frac{2n!}{\left(\frac{4n - p}{3}\right)!\left(\frac{2n + p}{3}\right)!}$.
\item Find the term independent of $x$ in the following binomial expansions:
  \begin{enumerate}
  \item $\left(x + \frac{1}{x}\right)^{2n}$,
  \item $\left(2x^2 + \frac{1}{x}\right)^{15}$,
  \item $\left(\sqrt{\frac{x}{3}} + \frac{3}{2x^2}\right)^{10}$,
  \item $\left(2x^2 - \frac{1}{x}\right)^{12}$,
  \item $\left(x^3 - \frac{3}{x^3}\right)^{25}$,
  \item $\left(x^2 - \frac{3}{x^3}\right)^{25}$,
  \item $\left(x^2 - \frac{3}{x^3}\right)^{10}$, and
  \item $\left(\frac{1}{2}x^{1/3} + x^{-1/3}\right)^8$.
  \end{enumerate}
\item If there is a term independent of $x$ in $\left(x + \frac{1}{x^2}\right)^n,$ show that it is equal to
  $\frac{n!}{\left(\frac{n}{3}\right)!\left(\frac{2n}{3}\right)!}$
\item Prove that in the expansion of $(1 + x)^{m + n}$, coefficients of $x^m$ and $x^n$ are equal, $\forall~m,n > 0,m,
  n\in\mathbb{N}$.
\item Give that the $4$th term in the expansion of $\left(px + \frac{1}{x}\right)^n$ is $\frac{5}{2}$. Find $n$ and $p$.
\item Find the middle term in the expansion of $\left(x - \frac{1}{2x}\right)^{12}$.
\item Find the middle terms in the expansion of $\left(2x^2 - \frac{1}{x}\right)^7$.
\item Prove that the middle term in the expansion of $\left(x + \frac{1}{x}\right)^{2n}$ is $\frac{1.3.5\ldots(2n - 1)}{n!}2^n$.
\item Show that the coefficient of the middle term in $(1 + x)^{2n}$ is equal to the sum of coefficients of the two middle terms in
  $(1 + x)^{2n - 1}$.
\item Find the middle term in the expansions of;
  \begin{enumerate}
  \item $\left(\frac{2x}{3} - \frac{3y}{2}\right)^{20}$,
  \item $\left(\frac{2x}{3} - \frac{3}{2x}\right)^6$,
  \item $\left(\frac{x}{y} - \frac{y}{x}\right)^7$,
  \item $(1 + x)^{2n}$, and
  \item $(1 - 2x + x^2)^n$.
  \end{enumerate}
\item Find the general and middle term of the expansion $\left(\frac{x}{y} + \frac{y}{x}\right)^{2n + 1}$; $n$ being a positive
  integer show that there is no term free of $x$ and $y$.
\item Show that the middle term in the expansion of $\left(x - \frac{1}{x}\right)^{2n}$ is $\frac{1.3.5\ldots (2n -1)}{n!}.(-2)^n$.
\item If in the expansion of $(1 + x)^{43}$, the coefficient of $(2r + 1)$th term is equal to the coefficient of $(r + 2)$th term,
  find $r$.
\item If the $r$th term in the expansion of $(1 + x)^{2n}$ has coefficient equal to that of the $(r + 4)$th term, find $r$.
\item If the coefficient of $(2r + 4)$th term and $(r - 2)$th term in the expansion of $(1 + x)^{18}$ are equal, find $r$.
\item If the coefficient of $(2r + 5)$th term and $(r - 6)$th term in the expansion of $(1 + x)^{39}$ are equal, fin ${}^rC_{12}$.
\item Given positive integers $r>1, n>2, n$ being even and the coefficient of $3r$th term and $(r + 2)$th term in the expansion o f
  $(1 + x)^{2n}$ are equal, find $r$.
\item If the coefficient of $(p + 1)$th term in the expansion of $(1 + x)^{2n}$ be equal to that of the $(p + 3)$th term, show that
  $p = n - 1$.
\item Find the two consecutive coefficients in the expansion of $(3x - 2)^{75}$, whose values are equal.
\item Show that the coefficient of $(r + 1)$th term in the expansion of $(1 + x)^{n + 1}$ is equal to the sum of the coefficients
  of the $r$th and $(r + 1)$th term in the expansion of $(1 + x)^n$.
\item Find the greatest term in the expansion of $\left(7 - \frac{10}{3}\right)^{11}$.
\item Show that if the greatest term in the expansion of $(1 + x)^{2n}$ has also the greatest coefficient $x$ lies between
  $\frac{n}{n + 1}$ and $\frac{n + 1}{n}$.
\item Find the greatest terms in the expansions of:
  \begin{enumerate}
  \item $\left(2 + \frac{9}{5}\right)^{10}$,
  \item $(4 - 2)^7$, and
  \item $(5 + 2)^{13}$.
  \end{enumerate}
\item Find the limits between which $x$ must lie in order that the greatest term in the expansion of $(1 + x)^{30}$ may have the
  greatest coefficient.
\item If $n\in\mathbb{P}$, then prove that $6^{2n} - 35n - 1$ is divisible by $1225$.
\item Show that $2^{4n} - 2^n(7n + 1)$ is some multuple of the square of $14$, where $n\in\mathbb{P}$.
\item Show that $3^{4n + 1} - 16n - 3$ is divisible by $256$, if $n\in\mathbb{P}$.
\item If $n\in\mathbb{P}$, show that
  \begin{enumerate}
  \item $4^n - 3n - 1$ is divisible by $9$,
  \item $2^{5n} - 31n - 1$ is divisible by $961$,
  \item $3^{2n + 2} - 8n  - 9$ is divisible by $64$,
  \item $2^{5n + 5} - 31n - 32$ is divisible by $961$ if $n > 1$, and
  \item $3^{2n} - 32n^2 + 24n - 1$ is divisible by $512$ if $n > 2$.
  \end{enumerate}
\item If three consecutive coefficients in the expansion of $(1 + x)^n$ be $165, 330$ and $462$, find $n$ and $r$.
\item If $a_1, a_2, a_3$ and $a_4$ be any four consecutive coefficients in the expansion of $(1 + x)^n$, prove that $\frac{a_1}{a_1
  + a_2} + \frac{a_3}{a_3 + a_4} = \frac{2a_2}{a_2 + a_3}$.
\item If $2$nd, $3$rd and $4$th terms in the expansion of $(x + y)^n$ be $240, 720$ and $1080$ respectively, find $x, y$ and $n$.
\item If $a, b, c$ be thre three consecutive terms in the expansion of some power of $(1 + x)$, prove that the exponent is
  $\frac{2ac + ab + bc}{b^2 - ac}$.
\item If $14$the, $15$th and $16$th term in the expansion of $(1 + x)^n$ are in A.P., find $n$.
\item If three consecutive terms in the expansion of $(1 + x)^n$ be $56, 70$ and $56$, find $n$ and the position of the
  coefficients.
\item If $3$rd, $4$th and $5$th terms in the expansion of $(a + x)^n$ be $84, 280$ and $560$, find $a, x$ and $n$.
\item If $6$th, $7$th and $8$th terms in the expansion of $(x + y)^n$ be $112, 7$ and $\frac{1}{4}$, find $x, y$ and $n$.
\item If $a, b, c$ and $d$ be the $6$th, $7$th, $8$th and $9$th terms respectively in any binomial expansion, prove that $\frac{b^2
  - ac}{c^2 - bd} = \frac{4a}{3c}$.
\item If the four consecutive coefficients in any binomial expansion be $a, b, c,$ and $d$, then prove that (a) $\frac{a + b}{a},
  \frac{b + c}{b}, \frac{c + d}{c}$ are in H.P., and (b) $(bc + ad)(b - c) = 2(ac^2 - b^2d)$.
\end{enumerate}
