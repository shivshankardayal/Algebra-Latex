\chapter{Logarithm}
\begin{enumerate}
\item $\log_{\sqrt{8}}x = \frac{10}{3} \Rightarrow \log_{2^{\tfrac{3}{2}}}x = \frac{10}{3}\Rightarrow \frac{2}{3}\log_2x = \frac{10}{3}$

  $\Rightarrow \log_2x = 5 \Rightarrow x = 2^5 = 32.$
\item L.H.S. $= \log_ba.\log_cb\log_ac = \frac{\log a}{\log b}.\frac{\log b}{\log c}.\frac{\log c}{\log a} = 1 =$ R.H.S.
\item L.H.S. $= \log_3\log_2\log_{\sqrt{5}}(\sqrt{5})^8 = \log_3\log_28 = \log_33 = 1 =$ R.H.S.
\item Given $a^2 + b^2 = 23ab \Rightarrow (a + b)^2 = 25ab \Rightarrow \frac{a + b}{5} = \sqrt{ab}$

  Taking $\log$ of both sides, we get

  $\log\frac{a + b}{5} = \frac{1}{2}(\log a + \log b)$.
\item L.H.S. $= 7\log\frac{16}{15} + 5\log\frac{25}{24} + 3\log\frac{81}{80} = \log 2$

  $= 7[\log 2^4 - \log 3.5] + 5[\log 5^2 - \log 2^3.3] + 3[\log 3^4 - \log 2^4.5]$

  $= 7[4\log 2 - \log 3 - \log 5] + 5[2\log 5 - 3\log 2 - \log 3] + 3[4\log 3 - 4\log 2 - \log 5]$

  $= \log 2 =$ R.H.S.
\item L.H.S. $= \log\tan1^\circ + \log\tan2^\circ + \ldots + \log\tan89^\circ$

  $= (\log\tan1^\circ + \log\tan89^\circ) + (\log\tan2^\circ + \log\tan88^\circ) + \cdots + \log\tan45^\circ$

  $=(\log\tan1^\circ\cot1^\circ) + (\log\tan2^\circ\cot2^\circ) + \cdots + \log\tan45^\circ [\because \tan(90^\circ - \theta) =
  \cot\theta]$

  $= \log 1 + \log 1 + \cdots + \log 1 = 0 [\because \tan\theta\cot\theta = 1]$
\item Given $\log_9\tan\frac{\pi}{6} = \log_9\frac{1}{\sqrt{3}} = -\log_9\sqrt{3} = -\log_99^{1/4} = -\frac{1}{4}$.
\item Given $\frac{\log_{a^2}b}{\log_{\sqrt{a}}b^2} = \frac{\frac{1}{2}\log_ab}{2.2\log_ab} = \frac{1}{8}$.
\item Given $\log_{\sqrt{5}}.008 = 2\log_6\frac{8}{1000} = 2[\log_58 - \log_51000] = 2[\log_58 - \log 8.125]$

  $= 2[\log_58 - \log_68 - \log_5125] = -2.\log_55^3 = -6$.
\item Given $\log_{2\sqrt{3}}144 = \log_{2\sqrt{3}}(2\sqrt{3})^4 = 4$.
\item L.H.S. $= \log_3\log_2\log_{\sqrt{3}}81 = \log_3\log_2\log_{\sqrt{3}}(\sqrt{3})^8 = \log_3\log_28 = \log_33 = 1 =$ R.H.S.
\item L.H.S. $= \log_ax\log_by = \frac{\log x}{\log a}.\frac{\log y}{\log b} = \frac{\log x}{\log b}.\frac{\log y}{\log a}$

  $= \log_bx\log_ay =$ R.H.S.
\item L.H.S. $= \log_2\log_2\log_216 = \log_2\log_2\log_22^4 = \log_2\log_24 = \log_22 = 1 =$ R.H.S.
\item R.H.S. $= \log_bx\log_cb\ldots\log_nm\log_an = \frac{\log x}{\log b}.\frac{\log b}{\log c}\cdots\frac{\log m}{\log
  n}.\frac{\log n}{\log a}$

  $= \frac{\log x}{\log a} = \log_ax =$ L.H.S.
\item Let $10^x\log_{10}a = z$.

  Taking $\log$ of both sides, we get

  $x\log_{10}a = \log z \Rightarrow \log_{10}a^x = \log z \Rightarrow z = a^x$.
\item Given $a^2 + b^2 = 7ab \Rightarrow a^2 + b^2 + 2ab = (a + b)^2 = 9ab$

  $\Rightarrow \left(\frac{a + b}{3}\right)^2 = ab \Rightarrow \frac{a + b}{3} = \sqrt{ab} = (ab)^{1/2}$

  Taking $\log$ of both sides,

  $\log \frac{a + b}{3} = \frac{1}{2}(\log a + \log b)$.
\item L.H.S. $= \frac{\log_a\log_ba}{\log_b\log_ab}$

  Let $\log_ba = z$, then L.H.S. $= \frac{\log_az}{\log_b\tfrac{1}{z}} = -\frac{\log_az}{\log_bz} = -\frac{\log z}{\log a}.\frac{\log z}{\log b}$

  $= - \frac{\log b}{\log a} = -\log_ab =$ R.H.S.
\item L.H.S. $= \log(1 + 2 + 3) = \log 6 = \log(1.2.3) = \log 1 + \log 2 + \log 3 =$ R.H.S.
\item L.H.S. $= 2\log(1 + 2 + 4 + 7 + 14) = 2\log 28 = \log 784$

  $= \log(1.2.4.7.14) = \log 1 + \log 2 + \log 4 + \log 7 + \log 14 =$ R.H.S.
\item L.H.S. $= \log 2 + 16\log\frac{16}{15} + 12\log\frac{25}{24} + 7\log\frac{81}{80}$

  $= \log 2 + 16[\log 2^4 - \log 3 - \log 5] + 12[\log 5^2 - \log 2^3 - \log 3] + 7[\log 3^4 - \log 2^4 - \log 5]$

  $= \log 2 + 16[4\log 2 - \log 3 - \log 5] + 12[2\log 5 - 3\log 2 - \log 3] + 7[4\log 3 - 4\log 2 - \log 5]$

  $= \log 2[1 + 64 - 36 - 28] + \log 3[28 - 16 -1 12] + \log 5[24 - 7 - 15]$

  $= \log 2 + \log 5 = \log 10 = 1$ [$\because$ default base of $\log$ is $10$.]
\item Given $\frac{\log_911}{\log_513}\div\frac{\log_311}{\log_{\sqrt{5}}13} =
  \frac{\log_{3^2}11}{\log_513}.\frac{\log_{5^{\tfrac{1}{2}}}11}{\log_3 11}$

  $= \frac{\tfrac{1}{2}\log_311}{\log_513}.\frac{2\log_513}{\log_311} = 1$.
\item Given, $3^{\sqrt{\log_32}} - 2^{\sqrt{\log_23}}$

  Taking $\log$ with base $10$,

  $\sqrt{\log_32}\log 3 - \sqrt{\log_23}\log 2 = \sqrt{\frac{\log 2}{\log 2}(\log 3)^2} - \sqrt{\frac{\log 3}{\log 2}(\log 2)^2}$

  $= \sqrt{\log 2\log 3} - \sqrt{\log 3\log 2} = 0$.
\item Given $\log_{10}343 = 2.5353 \Rightarrow \log_{10}7^3 = 2.5353 \Rightarrow \log_{10}7 = o.8451$

  For $7^n > 10^5 \Rightarrow n\log_{10}7 > 5 \Rightarrow n > \frac{5}{0.8451}$

  Thus, least such integer is $6$.
\item Since $a, b, c$ are in G.P., we can write $b^2 = ac$

  Taking $\log$ of both sides, we get

  $2\log b = \log a + \log c \Rightarrow \log a, \log b, \log c$ are in A.P.

  i.e. $\frac{1}{\log a}, \frac{1}{\log b}, \frac{1}{\log c}$ are in H.P.

  Multiplying each term with $\log x$,

  $\frac{\log x}{\log a}, \frac{\log x}{\log b}, \frac{\log x}{\log c}$ are in H.P.

  $\log_ax, \log_bx, \log_cx$ are in H.P.
\item R.H.S. $= 3\log2 + \log\sin x + \log\cos x + \log\cos2x + \log\cos4x$

  $= 2\log 2 + (\log 2.\sin x\cos x) + \log\cos2x + \log\cos4x$

  $= 2\log 2 + \log\sin2x + \log\cos2x + \log\cos4x = \log 2 + (\log 2.\sin2x\cos2x) + \log\cos4x$

  $= \log 2 + \log\sin4x + \cos4x = \log 2.\sin4x\cos4x$

  $= \log\sin8x =$ L.H.S.
\item We have to prove that $xyz + 1 = 2yz \Rightarrow x + \frac{1}{yz} = 2$

  L.H.S. $= x + \frac{1}{yz}$, substituting the values of $x, y$ and $z$,

  $\log_{2a}a + \frac{1}{\log_{3a}2a\log_{4a}3a} = \frac{\log a}{\log 2a} + \frac{\log 3a.\log 4a}{\log 2a.\log 3a}$

  $= \frac{\log a + \log 4a}{\log 2a} = \frac{\log(2a)^2}{\log 2a} = 2 =$ R.H.S.
\item We have to prove that $\log_{c + b}a + \log_{c - b}a = 2\log_{c + a}a\log_{c - b}a$

  Dividing both sides by $\log_{c + b}a\log_{c - b}a$,

  $\frac{1}{\log_{c - b}a} + \frac{1}{\log_{c + b}\log a} = 2$

  $\Rightarrow \log_a(c - b) + \log_a(c + b) = 2$

  $\Rightarrow \log_a(c^2 - b^2) = 2\Rightarrow c^2 = a^2 + b^2$

  which is true because $c$ is hypotenuse and $a$ and $b$ are sides of a right-angle triangle.
\item Let $\frac{\log x}{y - z} = \frac{\log y}{z - x} = \frac{\log z}{x - y} = k$

  $\log x = k(y - z), \log y = k(z - x), \log z = k(x - y)$

  $\Rightarrow x\log x + y\log y + z\log z = k(xy - zx + yz - xy + zx - yz) = 0$

  $\Rightarrow \log x^x + \log y^y + \log z^z = \log x^xy^yz^z = 0$

  $\Rightarrow x^xy^yz^z = 1$.
\item Given $\frac{yz\log(yz)}{y + z} = \frac{zx\log(zx)}{z + x} = \frac{xy\log(xy)}{x + y}$

  Dividing by $xyz, \frac{\log(yz)}{x(y + z)} = \frac{\log(zx)}{y(z + x)} = \frac{\log(xy)}{z(x + y)} = k$ (let)

  $\log y + \log z = k(xy + yz), \log z + \log x = k(yz + xy), \log x + \log y = k(yz + zx)$

  $\Rightarrow x\log x = kyz \Rightarrow x\log x = kxyz = y\log y = z\log z$

  $\Rightarrow x^x = y^y = z^z$.
\item We have to prove that $(yz)^{\log y - \log z}(zx)^{\log z - \log x}(xy)^{\log x - \log y} = 1$

  Taking $\log$ of both sides,

  $\Rightarrow (\log y - \log z)(\log y + \log z) + (\log z - \log x)(\log z + \log x) + (\log x - \log y)(\log x + \log y) = 0$

  $\Rightarrow (\log y)^2 - (\log z)^2 + (\log z)^2 - (\log x)^2 + (\log x)^2 - (\log y)^2 = 0$

  $\Rightarrow 0 = 0$.
\item L.H.S $= \log_N2 + \log_n3 + \cdots + \log_n1988$

  $= \log_N(2.3.4.\ldots1988) = \log_N1988! = \frac{1}{\log_{1988!}N} =$ R.H.S.
\item L.H.S. $= \log(1 + x) + \log(1 + x^2) + \log(1 + x^4)\ldots~{\rm to}~\infty$

  $= \log(1 + x + x^2 + \ldots~{\rm to}~\infty)$

  $= \log\frac{1}{1 - x}[\because 0 < x < 1]$ (from the formula for the sum of an infinite G.P.)

  $= -\log(1 - x) =$ R.H.S.
\item Let $S_n = \frac{1}{\log_2a} + \frac{1}{\log_4a} + \cdots$ up to $n$ terms

  $S_n = \log_a2 + \log_a4 + \log_a8 + \cdots$ up to $n$ terms

  $S_n = (1 + 2 + 3 + \cdots + n)\log_a2 = \frac{n(n + 1)}{2}\log_a2$.
\item L.H.S. $= \frac{1}{x + 1} + \frac{1}{y + 1} + \frac{1}{z + 1}$

  $= \frac{1}{\log_410 + \log_44} + \frac{1}{\log_220 + \log_220} + \frac{1}{\log_58 + \log_55}$

  $= \frac{1}{\log_440} + \frac{1}{\log_240} + \frac{1}{\log_540}$

  $= \log_{40}4 + \log_{40}2 + \log_{40}5 = \log_{40}(4.2.5) = \log_{40}40 = 1 =$ R.H.S.
\item L.H.S. $= \frac{1}{\log_abc + 1} + \frac{1}{\log_bca + 1} + \frac{1}{\log_cab + 1}$

  $= \frac{1}{\log_abc + \log_aa} + \frac{1}{\log_bca + \log_bb} + \frac{1}{\log_cab + \log_cc}$

  $= \frac{1}{\log_aabc} + \frac{1}{\log_babc} + \frac{1}{\log_cabc}$

  $= \log_{abc}a + \log_{abc}b + \log_{abc}c = \log_{abc}abc = 1 =$ R.H.S.
\item Given, $\frac{1}{1 + \log_ba + \log_bc} + \frac{1}{1 + \log_ca + \log_cb} + \frac{1}{1 + \log_ab + \log_ac} = 1$

  L.H.S. $= \frac{1}{\log_ba + \log_ba + \log_bc} + \frac{1}{log_cc + \log_ca + \log_cb} + \frac{1}{\log_aa + \log_ab + \log_ac}$

  $= \frac{1}{\log_babc} + \frac{1}{\log_cabc} + \frac{1}{\log_aabc}$

  Like previous problem the above expression will evaluate to $1$.
\item We have to prove that $x^{\log y - \log z}y^{\log z - \log x}z^{\log x - \log y} = 1$

  Taking $\log$ of both sides,

  $(\log y - \log z)\log x + (\log z - \log x)\log y + (\log x - \log y)\log z = 0$

  $\Rightarrow \log y\log z - \log z\log x + \log z\log y - \log x\log y + \log x\log z - \log y\log z = 0$

  $\Rightarrow 0 = 0$.
\item Let $\frac{\log a}{y - z} = \frac{\log b}{z - x} = \frac{\log c}{x - y} = k$

  $\Rightarrow x\log a = k(xy - zx), y\log b = k(yz - xy), z\log c = k(zx - yz)$

  Adding all,

  $x\log a + y\log b + z\log c = k(xy - zx + yz - xy + zx - yz) = 0$

  $\log a^xb^yx^z = 0\Rightarrow a^xb^yc^z = 1$
\item Let $\frac{x(y + z - x)}{\log x} = \frac{y(z + x - y)}{\log y} = \frac{z(x + y - z)}{\log z} = \frac{1}{k}$

  $\Rightarrow \log x = kx(y + z - x), \log y = ky(z + x - y), \log z = kz(x + y - z)$

  Let $y^zz^y = z^xz^z = x^yy^x$

  Taking $\log$, we have

  $z\log y + y\log z = x\log z + z\log x = y\log x + x\log y$

  $\Rightarrow zky(z + x - y) + ykz(x + y - z) = xkz(x + y - z) + zkx(y + z - x) = ykx(y + z - x) + xky(x + z - y)$

  $\Rightarrow yz^2 + xyz - y^2z + xyz + y^2 - z^2y = x^2z + xyz - xz^2 + xyz + xz^2 - x^2z = xy^2 + xyz - x^2y + x^2y + xyz -
  xy^2$

  $\Rightarrow 2xyz = 2xyz = 2xyz$.
\item Let $\frac{\log a}{b - c} = \frac{\log b}{c - a} = \frac{\log c}{a - b} = k$

  $\Rightarrow \log a = k(b - c), \log b = k(c - a), \log c = k(a - b)$

  $\Rightarrow (b + c)\log a = k(b^2 - c^2), (c + a)\log b = k(c^2 - a^2), (a + b)\log c = k(a^2 - b^2)$

  Adding all, $\log a^{b + c} + \log b^{c + a} + \log c^{a + b} = 0$

  $\Rightarrow a^{b + c}b^{c + a}c^{a + b} = 1$.
\item Let $\frac{\log x}{q - r} = \frac{\log y}{r - p} = \frac{\log z}{p - q} = k$

  $\Rightarrow \log x = k(q - r), \log y = k(r - p), \log z = k(p - q)$

  $\Rightarrow (q + r)\log x = k(q^2 - r^2), (r + p)\log y = k(r^2 - p^2), (p + q)\log z = k(p^2 - q^2)$

  Adding all $\log x^{q + r} + \log y^{r + p} + \log z^{p + q} = 0$

  $\Rightarrow x^{q + r}y^{r + p}z^{p + q} = 1$.

  Similarly, $p\log x = kp(q - r), q\log y = kq(r - p), r\log z = kr(p - q)$

  Adding all, $\log x^p + \log y^q + \log z^r = 0 \Rightarrow x^py^qz^r = 1$.
\item Given $y = a^{\tfrac{1}{1 - \log_ax}}$ and $z = a^{\tfrac{1}{1 - \log_ay}}$

  $\therefore z = a^{\tfrac{1}{1 - \log_aa^{\left(\tfrac{1}{1 - \log_ax}\right)}}} = a^{\tfrac{1}{1 - \tfrac{1}{1 - \log_ax}}}$

  Taking $\log$ of both sides with base $a$,

  $\log_az = \frac{1}{1 - \tfrac{1}{1 - \log_ax}} = \frac{1 - \log_ax}{-\log_ax} = 1 - \frac{1}{\log_ax}$

  $\Rightarrow x = a^{\tfrac{1}{1 - \log_az}}$.
\item Given $f(y) = e^{f(z)}$ and $z = e^{f(x)}$, where $f(x) = \frac{1}{1 - \log_ex}$

  $f(y) = e^{\tfrac{1}{1 - \log_ez}} = e^{\tfrac{1}{1 - \log_ee^{\tfrac{1}{1 - \log_ex}}}} = e^{\tfrac{1}{1 - \tfrac{1}{1 -
      \log_ex}}}$

  Following like above exercie $x = e^{f(y)}$.
\item L.H.S. $= \frac{1}{\log_2n} + \frac{1}{\log_3n} + \frac{1}{\log_4n} + \cdots + \frac{1}{\log_{43}n}$

  $= \log_n2 + \log_n3 + \log_n4 + \cdots + \log_n43 = \log_n(2.3.4\ldots 43)$

  $= \log_n43! = \frac{1}{\log_{43!}n} =$ R.H.S.
\item L.H.S. $= (1 + 2 + 3 + \cdots + n).2\log a = \frac{n(n + 1)}{2}.2\log a = n(n + 1)\log a =$ R.H.S.
\item We will use of the fact that positive characteristics of $n$ of a logarithmm means that there $n + 1$ digits in the number.

  Let $\log y = 12\log 12 = 12\log(2.2.3) = 12[2\times0.301 + 0.477] = 12.96$.

  Thus, number of digits is $13$.
\item We can use the fact that the number of positive integers having base $b$ and characteristics $n$ is $b^{n + 1} - b^n$.

  Thus, number of integer with base $3$ and characteristics $2$ is $3^3 - 3^3 = 18$.
\item Let $y = (0.0504)^{10} \Rightarrow \log_{10}y = 10\log_{10}(0.504) = 10\log_{10}(504\times10^-4)$

  $= -10\log_{10}[-4 + \log(2^3.3^2.7)] = -12.98$.

  Thus, characteristics is $-13$. Therefore, number of zeros after decimal and first significant digit is $12$.
\item Let $x = 72^{15} \therefore \log_{10}x = 15\log_{10}72 = 15\log_{10}(2^3\times3^2) = 15[3\log_{10}2 + 2\log_{10}3]$

  $i = 15[3\times0.301 + 2\times0.477] = 15[0.903 + 0.954] = 15\times1.857 = 27.855$

  So the characteristics is $27$ and hence the number of digits will be $28$.
\item Given $b = 5, n = 2$, therefore the number of integers will be $5^3 - 5^2 - 100$.
\item Let $x = 3^{15}\times 2^{10} \therefore \log_{10}x = 15\log_{10}3 + 10\log_{10}2$

  $= 15\times0.477 + 10\times 0.301 = 10.165$.

  So no. of digits will be $11$.
\item Let $x = 6^{20} \therefore \log_{10}x = 20\log_{10}(2\times3) = 20[\log_{10}2 + \log_{10}3]$

  $= 20[0.301 + 0.477] = 15.56$.

  So no. of digits will be $16$.
\item Let $x = 5^{25} \therefore \log_{10}x = 25\log_{10}\frac{10}{2}= 25[1 - \log_{10}2]$

  $= 25\times 0.699 = 17.475$

  So no. of digits will be $18$.
\item Given $\log_a[1 + \log_b\{1 + \log_c(1 + \log_px)\}] = 0$

  $\Rightarrow 1 + \log_b\{1 + \log_c(1 + \log_px)\} = 1$

  $\Rightarrow \log_b\{1 + \log_c(1 + \log_px)\} = 0$

  $\Rightarrow 1 + \log_c(1 + \log_px) = 1$

  $\Rightarrow \log_c(1 + \log_px) = 0$

  $\Rightarrow 1 + \log_px = 1$

  $\Rightarrow \log_px = 0 \Rightarrow x = 1$
\item Given $\log_7\log_5(\sqrt{x + 5} + \sqrt{x}) = 0 \Rightarrow \log_5(\sqrt{x + 5} + \sqrt{x}) = 1$

  $\Rightarrow \sqrt{x + 5} + \sqrt{x} = 5 \Rightarrow \sqrt{x + 5} = 5 - \sqrt{x}$

  Squaring both sides,

  $x + 5 = 25 + x - 10\sqrt{x}\Rightarrow \sqrt{x} = 2\Rightarrow x = 4$.
\item $\log_2x + \log_4(x + 2) = 2 \Rightarrow \log_2x + \frac{1}{2}\log_2(x + 2) = 2$

  $\Rightarrow 2\log_2x + \log_2(x + 2) = 4 \Rightarrow \log_2x^2(x + 3) = 4$

  $\Rightarrow x^2(x + 2) = 16 \Rightarrow x = 2$
\item $\log_{(x + 2)}x + \log_x(x + 2) = \frac{5}{2} \Rightarrow \frac{1}{\log_x(x + 2)} + \log_x(x + 2) = \frac{5}{2}$

  Let $z = \log_x(x + 2) \Rightarrow \frac{1}{z} + z = \frac{5}{2}$

  $2z^2 + 2 - 5z = 0 \Rightarrow z = 2, \frac{1}{2}$

  $\Rightarrow \log_x(x + 2) = 2, \frac{1}{2}$

  $\Rightarrow x + 2 = 2^2, x + 2 = \sqrt{x}$

  $x = 2, x^2 - 4x + 4 = 0 \Rightarrow x = \frac{3\pm\sqrt{-7}}{2}$

  However, $x$ cannot be a complex number. $\therefore x = 2$.
\item $\frac{\log(x + 1)}{\log x} = 2\Rightarrow \log_x(x + 1) = 2 \Rightarrow x + 1 = x^2$

  $\Rightarrow x = \frac{1\pm \sqrt{5}}{2}$

  $\because x > 0, x = \frac{1 + \sqrt{5}}{2}$.
\item $2\log_xa + \log_{ax}a + 3\log_{a^2x}a = 0 \Rightarrow \frac{2}{\log_ax} + \frac{1}{\log_aax} + \frac{1}{\log_aa^2x} = 0$

  $\Rightarrow \frac{2}{\log_ax} + \frac{1}{\log_aa + \log_ax} + \frac{1}{\log_aa^2 + \log_ax} = 0$

  $\Rightarrow \frac{2}{\log_ax} + \frac{1}{1 + \log_ax} + \frac{1}{2 + \log_ax} = 0$

  Substituting $\log_ax = z, \frac{2}{z} + \frac{1}{1 + z} + \frac{1}{2 + z} = 0$

  $\Rightarrow 6z^2 + 11z + 4 = 0 \Rightarrow z = -\frac{1}{2}, -\frac{4}{3}$

  $\therefore x = a^{-\tfrac{1}{2}}, a^{-\tfrac{4}{3}}$.
\item $x + \log_{10}(1 + 2^2) = x\log_{10}5 + \log_{10}6$

  $\Rightarrow \log_{10}10^x + \log_{10}(1 + 2^x) = \log_{10}5^x + \log_{10}6$

  $\Rightarrow \log_{10}10^x(1 + x^x) = \log_{10}(5^x.6)$

  $\Rightarrow 2^x(1 + 2^x) = 2.3 \Rightarrow 2^x = 2, 1 + 2^x = 3 \Rightarrow x = 1$.
\item $x^{\tfrac{3}{4}(\log_2x)^2 + \log_2x - \tfrac{5}{4}} = \sqrt{2}$

  Taking $\log_2$ of both sides,

  $\left[\frac{3}{4}(\log_2x)^2 + \log_2x - \frac{5}{4}\right]\log_2x = \frac{1}{2}\log_22$

  $\left[\frac{3}{4}(\log_2x)^2 + \log_2x - \frac{5}{4}\right]\log_2x = \frac{1}{2}$

  Let $\log_2x = z, \Rightarrow \left(\frac{3}{4}z^2 + z - \frac{5}{4}\right)z = \frac{1}{2}$

  Solving this qubic equation yields $x = 2, \frac{1}{4}, \frac{1}{\sqrt[3]{2}}$.
\item Given $(x^2 + 6)^{\log_3x} = (5x)^{\log_3x}$

  $\log_3x$ has a possible value of $0$, in that case $x = 1$

  If $\log_3x\neq 0, \Rightarrow x^2 + 6 = 5x \Rightarrow x = 2, 3$.
\item Given, $(3 + 2\sqrt{2})^{x^2 - 6x + 9} + (3 - 2\sqrt{2})^{x^2 - 6x + 9} = 6$

  We observe that $3 + 2\sqrt{2} = \frac{1}{3 - 2\sqrt{2}}$, thus, given equation becomes

  $(3 + 2\sqrt{2})^{x^2 - 6x + 9} + (3 + 2\sqrt{2})^{-(x^2 - 6x + 9)} = 6$

  Let $z = (3 + 2\sqrt{2})^{x^2 - 6x + 9}\Rightarrow z + \frac{1}{z} = 6 \Rightarrow z = 3\pm2\sqrt{2}$

  Thus, $x^2 - 6x + 9 = \pm 1 \Rightarrow x = 2, 4$ because other roots are irrational.
\item Given, $\log_8\left(\frac{8}{x^2}\right)\div(\log_8x)^2 = 3$

  $\Rightarrow \log_88 - \log_8x^2 = 3(\log_8x)^2 \Rightarrow 1 - 2\log_8x = 3(\log_8x)^2$

  Let $z = \log_8x\Rightarrow 1 - 2z = 3z^2 \Rightarrow z = -1, \frac{1}{3}\Rightarrow x = 2, \frac{1}{8}$.
\item Given, $\sqrt{\log_2(x)^4} + 4\log_4\sqrt{\tfrac{2}{x}} = 2$

  $\Rightarrow \sqrt{\log_2(x)^4} + 2\log_2\sqrt{\tfrac{2}{x}} = 2$

  $\Rightarrow \sqrt{4\log_2x} + \log_2\tfrac{2}{x} = 2$

  $\Rightarrow \sqrt{4\log_2x} + 1 - \log_2x = 2 \Rightarrow \sqrt{4\log_2x} = 1 + \log_2x$

  Squaring, $4\log_2x = 1 + 2\log_2x + (\log_2x)^2 \Rightarrow (\log_2x - 1)^2 = 0$

  $\Rightarrow \log_2x = 1 \Rightarrow x = 2$.
\item Given, $2\log_{10}x - \log_x0.01 = 5 \Rightarrow 2\log_{10}x - \log_x(10)^{-2} = 5$

  $\Rightarrow 2\log_{10}x - \log_x(10)^{-2} = 5 \Rightarrow 2\log_{10}x + 2\log_x10 = 5$

  $\Rightarrow 2\log_{10}x + \frac{2}{\log_{10}x} = 5$

  Let $z = \log_{10}x \Rightarrow 2z + \frac{2}{z} = 5 \Rightarrow z = 2, \frac{1}{2}$

  $\Rightarrow x = 100, \sqrt{10}$.
\item Given, $\log_{\sin x}2\log_{\cos x}2 + \log_{\sin x}2 + \log_{\cos x}2 = 0$

  $\Rightarrow \log_{\sin x}2(\log_{\cos x}2 + 1) + \log_{\cos x}2 = 0$

  $\Rightarrow \frac{\ln 2}{\ln \sin x}\left(\frac{\ln 2}{\ln \cos x} + 1\right) + \frac{\ln 2}{\ln \cos x} = 0$

  $\Rightarrow \frac{1}{\ln \sin x}\left(\frac{\ln 2}{\ln \cos x} + 1\right) + \frac{1}{\ln \cos x} = 0$

  $\Rightarrow \frac{1}{\ln \sin x}\left(\frac{\ln 2}{\ln \cos x} + 1\right) = -\frac{1}{\ln \cos x}$

  $\Rightarrow \frac{1}{\ln \sin x}(\ln 2 + \ln\cos x) = -1$

  $\Rightarrow \ln(\sin2x) = 0\Rightarrow x = 2k\pi + \frac{\pi}{4}, k\in\mathbb{I}$.
\item Given, $2^{x + 3} + 2^{x+2} + 2^{x + 1} = 7^x + 7^{x - 1}$

  $\Rightarrow 2^{x + 1}(2^2 + 2 + 1) = 7^{x - 1}(7 + 1) \Rightarrow 2^{x - 2} = 7^{x - 2}$

  Taking $\log$ of both sides

  $(x - 1)\log 2 = (x - 2)(\log 7), \because 2\neq 7 \Rightarrow x = 2$.
\item Given, $\log_{\sqrt{2}\sin x}(1 + \cos x) = 2$

  $\Rightarrow 1 + \cos x = (\sqrt{2}\sin x)^2 = 2\sin^2x = 2 - 2\cos^2 x$

  $\Rightarrow 2\cos^2x + \cos x - 1 = 0 \Rightarrow \cos x = -1, \frac{1}{2}$

  $\Rightarrow x = 2n\pi,\,2n\pi + \frac{\pi}{3}, n\in I$
\item Given, $\log_{10}[98 + \sqrt{x^2 - 12x + 36}] = 2$

  $\Rightarrow 98 + \sqrt{x^2 - 12x + 36} = 10^2 = 100$

  $\Rightarrow x^2 - 12x + 36 = 4 \Rightarrow x^2 - 12x + 32 = 0$

  $\Rightarrow x = 4, 8$.
\item Given, $2^x3^{2x} - 100 = 0 \Rightarrow x\log_{10}2 + 2x\log_{10}3 = \log_{10}100 = 2$

  Substituting values for $\log_{10}2$ and $\log_{10}3$, we get

  $0.30103x + 0.95424x = 2 \Rightarrow x = 1.593$.
\item Given, $\log_x3\log_{\tfrac{x}{3}}3 + \log_{\tfrac{x}{81}}3 = 0$

  $\Rightarrow \frac{1}{\log_3x}.\frac{1}{\log_x\tfrac{x}{3}} + \frac{1}{\log_3\tfrac{x}{81}} = 0$

  $\Rightarrow \frac{1}{\log_3x}.\frac{1}{\log_3x - \log_33} + \frac{1}{\log_3x - \log_381} = 0$

  Let $z = \log_3x, \Rightarrow \frac{1}{z}.\frac{1}{z - 1} + \frac{1}{z - 4} = 0$

  $\Rightarrow z - 4 + z^2 - z = 0 \Rightarrow z^2 - 4 = 0 \Rightarrow z = \pm 2$

  $\Rightarrow x = 9, \frac{1}{9}$.
\item Given, $\log_{(2x + 3)}(6x^2 + 23x + 21) = 4 - \log_{(3x + 7)}(4x^2 + 12x + 9)$

  $\Rightarrow \log_{(2x + 3)}(2x + 3)(3x + 7) = 4 - \log_{(3x + 7)}(2x + 3)^2$

  $\Rightarrow 1 + \log_{(2x + 3)}(3x + 7) = 4 - 2\log_{(3xx + 7)}(2x + 3)$

  Let $z = \log_{(2x + 3)}(3x + 7)$,

  $\Rightarrow 1 + z = 4 - \frac{2}{z}\Rightarrow z = 1, 2 \Rightarrow x = -4, -3, -\frac{1}{4}$.

  For logarithm to be defined, $2x + 3 > 0,\,2x + 3\neq 1$ and $3x + 7 > 0,\,3x + 7 \neq 1$.

  Thus, $x = -\frac{1}{4}$ is the only valid solution.
\item Given, $\log_2(x^2 - 1) = \log_{\tfrac{1}{2}}(x - 1)$

  $\Rightarrow \log_2(x^2 - 1) = \log_{2^{-1}}(x - 1) = -\log_2(x - 1) = \log_2\frac{1}{x - 1}$

  $\Rightarrow x^2 - 1 = \frac{1}{x - 1}\Rightarrow x = 0, x^2 - x - 1 = 0$

  $\Rightarrow x = 0, \frac{1\pm\sqrt{5}}{2}$

  For logarithm to be defined $x^2 - 1 > 0$ and $x - 1 > 0$

  Thus, $x = \frac{1+\sqrt{5}}{2}$ is the only acceptable solution.
\item Given, $\log_5\left(5^{\tfrac{1}{x} + 125}\right) = \log_56 + 1 + \frac{1}{2x}$

  $\Rightarrow \log_5\left(5^{\tfrac{1}{x} + 125}\right) - \log_56 = 1 + \frac{1}{2x}$

  $\Rightarrow \log_5\left(\tfrac{5^{\tfrac{1}{x} + 125}}{6}\right) = 1 + \frac{1}{2x}$

  $\Rightarrow 5^{\tfrac{1}{x} + 125} = 30.5^{\tfrac{1}{2x}}$

  Let $z= 5^{\tfrac{1}{2x}}$

  $\Rightarrow z^2 - 30z + 125 = 0 \Rightarrow z = 5, 25 \Rightarrow x = \frac{1}{2}, \frac{1}{4}$.
\item For $\log_{100}|x + y| = \frac{1}{2} \Rightarrow (x + y)^2 = 100$

  And for $\log_{10}y - \log_{10}|x| = \log_{100}4\Rightarrow \log_{10}\frac{y}{|x|} = \log_{10}2$

  $\Rightarrow y = 2|x| \Rightarrow y^2 = 4x^2 \Rightarrow 5x^2 + 4x|x| = 100$

  When $x > 0,\, x = \frac{10}{3}$ and when $x < 0,\,x = -10$

  $\Rightarrow y = \frac{20}{3}, 20$.
\item Given, $2\log_2\log_2x + \log_{\tfrac{1}{2}}\log_2(2\sqrt{2}x) = 1$

  $\Rightarrow \log_2(\log_2x)^2 - \log_2\log_2(2\sqrt{2}x) = 1$

  $\Rightarrow \log_2\left(\frac{(\log_2)^2}{\log_2(2\sqrt{2}x)}\right) = 1$

  $\Rightarrow \frac{(\log_2x)^2}{\log_2(2\sqrt{2}x)} = 2$

  $\Rightarrow (\log_2x)^2 = \log_2(2\sqrt{2}x)^2$

  $\Rightarrow (\log_2x)^2 - 3 - 2\log_2x = 0$

  Let $z = \log_2x$, then $z^2 - 2z - 3 = 0 \Rightarrow z = -1, 3$

  $\Rightarrow x = \frac{1}{2}, 8$

  For logarithm to be defined $x > 0, 2\sqrt{2}x > 0,\log_2x > 0, \log_2(2\sqrt{2}x) > 0$.

  Thus, $x = 8$ is only acceptable solution.
\end{enumerate}
