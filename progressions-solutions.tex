\chapter{Progressions}
\begin{enumerate}
\item Given $t_n = 2n^2 + 1 \Rightarrow t_{n - 1} = 2(n - 1)^2 + 1$

  $\therefore d = t_n - t_{n - 1} = 4n - 2$, which is not constant. Hence, the sequence is not in A.P.
\item Given, $t_1 = 1, t_2 = 2$ and $t_{n+2} = t_n + t_{n + 1}$

  $\therefore t_3 = t_1 + t_2 = 3, t_4 = t_2 + t_3 = 5, t_5 = t_3 + t_4 = 8$.
\item Given $t_n = 3n + 5 \Rightarrow t_1 = 3\times1 + 5 = 8, t_2 = 3\times2 + 5 = 11, t_3 = 3\times3 + 5 = 14$. So the seuquence
  is $8, 11, 14, \ldots, 3n + 5$.
\item Given $t_n = 2n^2 + 3 \Rightarrow t_1 = 2\times1^2 + 3 = 5, t_2 = 2\times2^2 + 3 = 11, t_3 = 2\times3^2 + 5 = 23$. So the
  sequence is $5, 11, 23, \ldots, 2n^2 + 3$.
\item Given, $t_n = \frac{3n}{2n + 4}\Rightarrow t_1 = \frac{3\times1}{2\times1 + 4} = \frac{3}{6} = \frac{1}{2}, t_2 =
  \frac{3\times2}{2\times2 + 4} = \frac{6}{8} = \frac{3}{4}, t_3 = \frac{3\times3}{2\times3 + 4} = \frac{9}{10}$. So the
  sequence is $\frac{1}{2}, \frac{3}{4}, \frac{9}{10}, \cdots, \frac{3n}{2n + 4}$.
\item Given, $t_1 = 2, t_{n + 1} = \frac{2t_n + 1}{t_n + 3} \Rightarrow t_2 = \frac{2t_1 + 1}{t_1 + 3} = \frac{2\times1 + 1}{1 + 3}
  = \frac{3}{4}, t_3 = \frac{2t_2 + 1}{t_2 + 3} = \frac{2\times\tfrac{3}{4} + 1}{\tfrac{3}{4} + 3} = \frac{10}{15} = \frac{2}{3}$.
  So the sequence is $2, \frac{3}{4}, \frac{2}{3}, \cdots$.
\item Given, $t_n = 4n^2 + 1 \Rightarrow t_{n - 1} 4(n - 1)^2 + 1$

  $\therefore d = t_n - t_{n - 1} = 8n - 4$, which is not constant. Hence the sequence is not in A.P.
\item Given $t_n = 2an + b \Rightarrow t_{n - 1} = 2a(n - 1) + b$

  $\therefore d = t_n - t_{n - 1} = 2a$. which is a constant. Hence the sequence will be an A.P.
\item Given, $t_1 = 3, t_2 = 3, t_3 = 6$ and $t_{n + 2} = t_n + t_{n + 1}$

  $\therefore t_4 = t_2 + t_3 = 3 + 6 = 9$ and $t_5 = t_3 + t_4 = 6 + 9 = 15$.
\item $t_1 = 1 = a + b + c, t_2 = 5 = 4a + 2b + c$ and $t_3 = 11 = 9a + 3b + c$

  $\therefore t_2 - t_1 = 4 = 3a + b$ and $t_3 - t_2 = 6 = 5a + b$

  $\Rightarrow 2a = 2 \Rightarrow a = 1 \Rightarrow b = 1 \Rightarrow c = -1$

  $\Rightarrow t_{10} = 1\times10^2 + 1\times10 - 1 = 109$.
\item Difference between successive terms i.e. commond difference, $d = 12 - 9 = 15 - 12 = 18 - 15 = 3$ which is a constant, hence,
  the given sequence is an A.P.

  Here first term $t_1 = 9$ and $d = 3 \therefore t_{16} = 9 + (16 - 1)3 = 54$ and $t_n = 9 + (n - 1)3 = 3(n + 2)$.
\item $t_1 = \log a, t_2 = \log(ab) = \log a + \log b, t_3 = \log(ab^2) = \log a + 2\log b$

  $t_2 - t_1 = t_3 - t_2 = \log b$. Clearly, $t_1 = \log a, d = \log b$ which is constant so the sequence is an A.P.

  $\therefore t_n = \log a + (n - 1)\log b = \log(ab^{n - 1})$.
\item Given, $t_n = 5 - 6n \Rightarrow t_1 = 5 - 6 = -1$

  $S_n = \frac{n}{2}[t_1 + t_n] = n(2 - 3n)$.
\item $d = 7 - 3 = 11 -7 = 4, t_n = 407 = 3 + (n - 1)d \Rightarrow n = \frac{404}{4} + 1 = 102$.
\item Since $a, b, c, d, e$ are in A.P. $\therefore a + e = b + d = 2c = k$(say)

  $\therefore a - 4b + 6c - 4d + e = (a + e) - 4(b + d) + 3.2c = k - 4k + 3k = 0$.
\item Let $a$ be the first term and $d$ be the common difference of the given A.P.

  Given, $5t_5 = 8t_8 \Rightarrow 5a + 20d = 8a + 56d \Rightarrow 3a = -36d \Rightarrow a = -12d$

  $\Rightarrow t_{13} = a + 12d = 0$.
\item Let $n$-th term be the smallest positive number. From the sequence we obtain that $t_1 = 25$ and $d = -2\frac{1}{4} = -\frac{9}{4}$.

  Then $t_n > 0 \Rightarrow 25 - (n - 1)\frac{9}{4} > 0\Rightarrow n < \frac{25\times4}{9} + 1 \Rightarrow n = 12$.
\item The given pay scale represents an A.P. with $t_1 = 700, d = 40$ and $t_n = 1500$.

  $\therefore t_n = t_1 + (n - 1)d \Rightarrow n = \frac{t_n - t_1}{d} + 1 = \frac{1500 - 700}{40} + 1 = 21$.

  Thus, the person will reach maximum payment in $21$ years.
\item Let $a$ be the first term and $d$ be the common difference of the A.P. According to the question,

  $t_7 = a + 6d = 34$ and $t_{13} = a + 12d = 64$

  Subtracting $6d = 30 \Rightarrow d = 5 \Rightarrow a = 4$. So the A.P. is $4, 9, 14, \ldots$.
\item If $55$ is the $n$-th term then $n$ will have to be an integer. From the given sequence $a = 1, d = 3 - 1 = 5 - 3 = 2$.

  $55 = 1 + (n - 1)2 \Rightarrow n = 28$, which is an integer and hence, $55$ will be $28$-th term of the A.P.
\item From the given sequence $a = 2000, d = 1995 - 2000 = 1990 - 1995 = -5$.

  Let $n$-th term be first negative term, then, $a + (n - 1)d < 0 \Rightarrow 2000 -(n - 1)5 < 0$

  \Rightarrow $n > 401 \Rightarrow n = 402 \Rightarrow t_{402} = 2000 - (402 - 1)5 = -5$.
\item Common different of the sequence $2, 4, 6, 8, \ldots$ is $2$ and common difference of the seqquence $3, 6, 9, \ldots$ is $3$.

  Thus, common terms will have a common different which is L.C.M. of these two commond differences i.e. $6$.

  Last term of first sequence is $200$ and last term of second sequence is $240$. Clearly, last identical(common) number will be
  less than $200$. We also observe that $6$ is the first identical term. Let there be $n$ such terms. Then

  $6 + (n - 1)6 \leq 200 \Rightarrow n\leq \frac{194}{6} + 1 \Rightarrow n = 33$. Thus there will be $33$ identical terms in the two
  given A.P.
\item Clearly the first number of three digits divisible by $5$ is $100$; while the last such number is $995$. Since these numbers
  are all divisible by $5$ they will form an A.P. with common difference $5$.

  Clearly, $t_1 = 100, t_n = 995, d = 5$ and we have to find $n$.

  $t_n = 995 = 100 + (n - 1)5\Rightarrow n = 180$.
\item Given sequence is $4, 9, 14, \ldots$. So $a = 4, d = 9 - 4 = 14 - 9 = 5$. Let $105$ be $n$-th term of this A.P. then $n$ has
  to be an integer for this assumption to be true.

  $105 = 4 + (n - 1)5 \Rightarrow n = \frac{106}{5}$ which is not an integer and therefore $105$ is not a term in the given A.P.
\item This problem is same as problem $21$ and has been left as an exercise.
\item This problem is same as problem $22$ and has been left as an exercise.
\item Let $a$ be the first term and $d$ be the common difference of the A.P. Given,

  $mt_m = nt_n \Rightarrow ma + (m - 1)md = na + (n - 1)nd \Rightarrow (m - n)a = (n^2 - n - m^2 + m)d$

  $\Rightarrow a = -(m + n - 1)d\,\therefore t_{m + n} = a + (m + n - 1)d = 0$.
\item Let $x$ be the first term and $y$ be the common difference of the A.P. Then,

  $a = x + (p - 1)y, b = x + (q - 1)y, c = x + (r - 1)y$

  We have to prove that $a(q - r) + b(r - p) + c(p - q) = 0$.

  Substituting the values of $a, b$ and $c$ in the above equation

  L.H.S. $= [x + (p - 1)y](q - r) + [x + (q - 1)y](r - p) + [x + (r - 1)y](p - q)$

  $= x(q - r + r - p + p - q) + y[(p - 1)(q - r) + (q - 1)(r - p) + (r - 1)(p - q)]$

  $= 0 =$ R.H.S.
\item First number after $100$ which is divisible by $7$ is $105$. The last number divisible by $7$ before $1000$ is $994$.

  Let $n$ be the numbers divisible by $7$ between $100$ and $1000$. Then $994 = 105 + (n - 1)7$

  $\Rightarrow n = 128$. Then no. of numbers not divisible by $7$ is $1000 - 100 - 128 = 772$.
\item Let $x$ be the first term and $y$ be the common difference of the A.P. Then,

  $a = x + (p - 1)y, b = x + (q - 1)y, c = x + (r - 1)y$

  We have to prove that $(a - b)r + (b - c)p + (c - a)q = 0$

  Substituting the values of $a, b$ and $c$ in the above equation

  L.H.S. $= (p - q)yr + (q - r)yp + (r - p)yq = 0 =$ R.H.S.
\item Let the numbers in A.P. be $a - d, a$ and $a + d$. Given their sum is $27$ and sum of squares is $293$.

  $\therefore a - d + a + a + d = 27 \Rightarrow a = 9$

  $\therefore (a - d)^2 + a^2 + (a + d)^2 = 293 \Rightarrow 3a^2 + 2d^2 = 293 \Rightarrow 3\times81 + 2d^2 = 293$

  $\Rightarrow 2d^2 = 50 \Rightarrow d = \pm5$

  So the numbers are $4, 9, 14$ or $14, 9, 4$.
\item Let the numbers in A.P. be $a - 3d, a - d, a + d, a + 3d$. Given their sum is $24$ and product is $945$.

  $\therefore a - 3d + a - d + a + d + a + 3d = 24 \Rightarrow 4a = 24 \Rightarrow a = 6$

  Also, $(a - 3d)(a - d)(a + d)(a + 3d) = 945 \Rightarrow (a^2 - 9d^2)(a^2 - d^2) = 945$

  $\Rightarrow a^4 - 10a^2d^2 + 9d^4 = 945 \Rightarrow 9d^4 - 360d^2 + 1296 - 945 = 0$

  $\Rightarrow 9d^4 - 360d^2 + 351 = 0 \Rightarrow d^4 - 40d^2 + 39 = 0$

  $\Rightarrow (d^2 - 1)(d^2 - 39) = 0$. Since the numbers are integers $\Rightarrow d^2 \neq 39$.

  $\Rightarrow d = \pm 1$. So the numbers are $3, 5, 7, 9$ or $9, 7, 5, 3$.
\item Let $a$ be the first term and $d$ be the common ratio of the A.P. Given,

  $t_p = a + (p - 1)d = q$ and $t_q = a + (q - 1)d = p$

  $\Rightarrow (p - q)d = q - p \Rightarrow d = -1 \Rightarrow a = p + q - 1$

  $\Rightarrow t_{p + q} = a + (p + q - 1)d = p + q - 1 - (p + q - 1) = 0$.
\item Let $a$ be the first term and $d$ be the common ratio of the A.P.

  $\Rightarrow t_m = a + (m - 1)d, t_{2n + m} = a + (2n + m - 1)d$

  $\Rightarrow t_m + t_{2n + m} = 2a + (2m + 2n - 2)d = 2[a + (m + n - 1)d] = 2t_{m + n}$
\item Let the three numbers be $a - d, a, a + d$. Given that their sum is $15$ and sum of their square is $83$.

  $\Rightarrow a - d + a + a + d = 15 \Rightarrow 3a = 15 \Rightarrow a = 5$

  $\Rightarrow (a - d)^2 + a^2 + (a + d)^2 = 83 \Rightarrow 3a^2 + 2d^2 = 83 \Rightarrow 3\times5^2 + 2d^2 = 83^2$

  $\Rightarrow d = \pm2$. So the numbers are $3, 5, 7$ or $7, 5, 3$.
\item This problem is similar to previous problem and has been left as an exercise.
\item Let the three numbers be $a - d, a, a + d$. Given their sum as $12$ and sum of cubes as $408$.

  $\therefore a - d + a + a + d = 12 \Rightarrow 3a = 12 \Rightarrow a = 4$

  $\therefore (a - d)^3 + a^3 + (a + d)^3 = 3a^3 + 6ad^2 = 408 \Rightarrow 24d^2 = 216 \Rightarrow d = \pm 3$

  Hence, the numbers are $1, 4, 7$ or $7, 4, 1$.
\item Let the numbers in A.P. be $a - 3d, a - d, a + d, a + 3d$. Given their sum is $24$ and product of first and
  fourth to product of second and third is $2:3$.

  $\therefore a - 3d + a - d + a + d + a + 3d = 20 \Rightarrow 4a = 20 \Rightarrow a = 5$

  $\therefore \frac{(a - 3d)(a + 3d)}{(a - d)(a + d)} = \frac{2}{3}$

  $\Rightarrow 3a^2 - 27d^2 = 2a^2 - 2d^2 \Rightarrow a^2 = 25d^2 \Rightarrow d = \pm1$.

  Therefore numbers are $2, 4, 6, 8$ or $8, 6, 4, 2$.
\item Let the three numbers be $a - d, a, a + d$. Given their sum is $-3$ and product is $8$.

  $\therefore a - d + a + a + d = -3 \Rightarrow 3a = -3 \Rightarrow a = -1$

  $\therefore (a - d).a.(a + d) = 8 \Rightarrow a^2 - d^2 = -8 \Rightarrow d = \pm3$

  Hence the numbers are $-4, -1, 2$ or $2, -1, -4$.
\end{enumerate}
