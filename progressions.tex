\chapter{Progressions}
There are three different progressions: arithmetic progression, geometric progression and harmonic progression. We start this
chapter with arithmetic progression or A.P.

\section{Arithmetic Progressions}
Consider sequences like $1, 2, 3, 4, \ldots$ or $-1, -2, -3, -4, \ldots$ or $1, 3, 5, 7, \ldots$ or $a, a + d, a + 2d, \ldots$

These sequences increase or decrease with a common difference. When quantities increase or decrease with a common difference they
are said to be in \textit{Arithmetic Progression}. The \textit{common difference} can be found by subtracting any term of the
series that follows it. For example for the first series it is $1$ and for the last it is $d$.

Consider the series $a, a + d, a + 2d, a + 3d, \ldots$

Simple observation tells us that $1$st term is $a$, $2$nd term is $a + d$, the $3$rd term is $a + 2d$ and hence the $n$th term will
be $a + (n - 1)d$. These terms are typically written as $t_1, t_2, t_3, \ldots, t_n$.

\subsection{$n$th Term of Arithmetic Progression}
Following above discussion, we can clearly say that the $n$th term of an arithmetic progression is given by $t_n = a + (n - 1)d$,
where $a$ is called the first term and $d$ the common difference.

\subsection{Sum of an Arithmetic Progression}
Let $S_n$ represent the sum of first $n$ terms of an arithmetic progression, then we can write.

$$S_n = a + (a + d) + (a + 2d) + \ldots + [a + (n - 2)d] + [a + (n - 1)d]$$
Writing the terms in reverse order we have
$$S_n = [a + (n - 1)s] + [a + (n - 2)d] + \ldots + (a + d) + a$$
Adding term by term, we get
$$2S_n = [2a + (n - 1)d] + [2a + (n - 1)d] + \ldots\text{~to~}n\text{~terms~}$$
$$2S_n = n[2a + (n - 1)d]\Rightarrow S_n = \frac{n}{2}[2a + (n - 1)d]$$

We also see that $S_n = \frac{n}{2}(t_1 + t_n)$

We also see that if a series is $1 + 2 + 3 + \ldots + n = \sum_{i=0}^ni = \frac{n(n + 1)}{2}$.

\subsection{Arithmetic Mean}
When three quantities are in arithmetic progression the quantity in the middle is known to be arithmetic mean of the other two. For
example, if $a, b, c$ are in A.P., then $b$ is said to be arithmetic mean of $a$ and $c$. In general, it is written $b = \frac{a +
  c}{2}$. This can be examined further. Let $b = a + d$, then $c = a + 2d$. Clearly, $b = \frac{a + c}{2}$.

It is also possible to insert $n$ numbers between any two numbers such that all of them are in A.P. Consider two numbers $a$ and
$b$ in between which we want to insert $n$ numbers such that they are in A.P. Clearly, $b$ will become $n +2$th term of A.P. Let
common difference be $d$ then we can write $b = a + (n + 1)d \Rightarrow d = \frac{b - a}{n + 1}$. Now all the $n$ arithmetic means
can be deduced. Let those be $m1, m2, \ldots, m_n$ then $m_1 = a + \frac{b - a}{n + 1}, m_2 = a + \frac{2(b - a)}{n + 1}, \ldots,
m_n = a + \frac{n(b - a)}{n + 1}$.

First A.M. $= a + d = \frac{an + b}{n +1}$

Second A.M. $= a + 2d = \frac{a(n - 1) + b}{n + 1}$

$\ldots$

$n$th A.M. $= a + nd = \frac{a + nb}{n + 1}$

Suppose there are $n$ terms of an A.P., then the arithmetic mean of those $n$ terms is given by $\frac{t_1 + t_2 + \ldots +
  t_n}{n}$.

\subsection{Deducing Number of Terms}
We know that $S_n = \frac{n}{2}[2a + (n - 1)d]$. Say $S_n, a$ and $d$ are known and we have to evaluate $n$. This being a quadratic
equaion will have two roots for $n$. If the results are positive and integral then there is no problem in interpreting the
results. In some cases for a negative root a suitable interpretation can be given.

\textbf{Example:} How many terms of the series $-8, -6, -4, \ldots$ must be added for the sum to be $36$?

$\frac{n}{2}[-16 + (n - 1)2] = 36\Rightarrow n^2 - 9n - 36 = 0 \Rightarrow n = 12, -3$

If we take $12$ terms of the series, we have $-8, -6, -4, -2, 0, 2, 4, 6, 8, 10, 12, 14$. The sum of these terms is $36$ and sum of
last three terms is also $36$ which is represented by $n = -3$.

\subsection{Properties of an A.P.}
\begin{enumerate}
\item If a fixed number is added to or subtracted from each item of a given A.P., then the resulting is also an A.P., and it has
  the same common difference as that of the given A.P.
\item If each term of an A.P. is multiplied or divided by a non-zero fixed constant then the resulting sequence is also an A.P. The
  common difference is multiplied or divided by the same factor.
\item If $a_1, a_2, a_3, \ldots$ and $b_1, b_2, b_3, \ldots$ are two arithmetic progressions then $a_1 + b_1, a_2 + b_2, a_3 + b_3,
  \ldots$ are also in A.P.
\item If we have to choose three unknown terms in an A.P. then it is best to choose them as $a - d, a, a + d$.
\item If we have to choose four unknown terms in an A.P. then it is best to choose them as $a - 3d, a - d, a + d, a + 3d$.
\item In an A.P., the sum of terms equidistant from the beginning and end is constant and is equal to the sum of first and last
  term.
\item Any term of an A.P., except the first, is equal to half the sum of terms which are equidistant from it:
  $$a_n = \frac{1}{2}(a_{n - k} + a_{n + k}), ~k<n,\text{~and for~} k = 1$$
  $$a_n = \frac{1}{2}(a_{n - 1} + a_{n + 1})$$
\item $t_n = S_n - S_{n - 1}, n\geq 2$
\item If $t_n = pn + q$ i.e. a linear expression in $n$ then it will form an A.P. of common difference $p = t_n - t_{n - 1}$ and
  first term $p + q$. For example, if $t_n = 3n + 4$, then it is an A.P. of common difference $3$ anda the first term as $7$.
\item If $S_n = an^2 + bn + c$ i.e. a quadratic function in $n$, then the series in an A.P. where $a = 2a,$ twice the coefficient
  of $n^2$.
\end{enumerate}

\subsection{Sum of Squares and Cubes and More}
We observe that
$$i^3 - (i - 1)^3 = 3i^3 - 3i + 1 \Rightarrow \sum_{i = 1}^n[i^3 - (i - 1)^3] = 3\sum_{i = 0}^ni^2 - \frac{3n(n + 1)}{2} + n$$
$$n^3 = 3\sum_{i = 0}^ni^2 - \frac{3n(n + 1)}{2} + n \Rightarrow 3\sum_{i=0}^ni^2 = n^3 + \frac{3n(n + 1)}{2} - n$$
$$\sum_{i=0}^ni^2 = \frac{n(n + 1)(2n + 1)}{6}$$

Following in a similar fashion, we can show that

$$\sum_{i=0}^n = \left\{\frac{n(n + 1)}{2}\right\}^2$$

More powers can be evaluated in a similar fashion.

\section{Geometric Progressions}
A succession of numbers is said to be in geometric progressions or geometric sequence if the ratio of any term and the term
preceeding it is constant throughout. This constant is called \textit{common ratio} of the G.P.

Example: $1, 2, 4, 8, 16, \ldots$

Here, $\frac{t_2}{t_1} = \frac{t_3}{t_2} = \ldots = 2$.

Also, $1, 3, 9, 27,\ldots$ are in geometric progression whose first term is $1$ and common ratio is $3$.

Also, $2, -4, 8, -16, \ldots$ are in geometric progression whose firts term is $2$ and common ratio is $-2$.

\subsection{Properties of a G.P.}
\begin{enumerate}
\item If the each term of a G.P. be multiplied by a non-zero number, then the sequence obtained is also a G.P.

  \textbf{Proof:} Let the given G.P. be $a, ar, ar^2, ar^3, \ldots$

  Let $k$ be a non-zero number, the sequence obtained by multiplying each term of the given G.P. by $k$ is $ak, ark, ar^2k, ar^3k,
  \ldots$

  Clearly, the series is in G.P. with the same common ratio as previous ratio i.e. $r$.

  Again, dividing each term of G.P. $a, ar, ar^2, a^3, \ldots$ we obtain the sequence $\frac{a}{k}, \frac{ar}{k}, \frac{ar^2}{k},
  \ldots$

  It is clear that this new sequence is also a G.P., whose common ratio is $r$.
\item The reciprocals of the terms of a G.P. are also in G.P.

  \textbf{Proof:} Let the G.P. be $a, ar, ar^2, \ldots$, the sequence whose terms are reciprocals of this G.P. is $\frac{1}{a},
  \frac{1}{ar}, \frac{1}{ar^2}, \ldots$

  It is clear that this sequence is in G.P., whose first term is $\frac{1}{a}$ and common ratio is $\frac{1}{r}$.
\end{enumerate}

\subsection{Sum of the First $n$ Terms of a G.P.}
Let $a$ be the first term and $r$ be the common ratio of a G.P. and $S_n$ be the sum of its first $n$ terms

\textbf{Case I:} When $r\neq 1$
$$S_n = a + ar + ar^2 + \ldots + ar^{n - 2} + ar^{n - 1}$$
$$rS_n = ar + ar^2 + \ldots + ar^{n - 1} + ar^n$$

Subtracting, we get $(1 - r)S_n = a - ar^n = a(1 - r^n)$

$\therefore S_n = \frac{a(1 - r^n)}{1 - r} = \frac{a(r^n - 1)}{r - 1}$

\textbf{Case II:} When $r = 1$

$S_n = a + a + \ldots + a = na$ and this G.P. is also an A.P. whose common difference is $0$.

\subsection{Sum of Infinite Terms of a G.P.}
If $|r|\geq 1$ then sum would be $\pm\infty$. However, if $|r|< 1$ then sum would be finite.

We have obtained that $S_n = \frac{a(1 - r^n)}{1 - r}$

We see that as $n$ approaches $\infty, r^n$ will approach $0$. Thus, $S_\infty = \frac{a}{1 - r}$

\subsection{Recurring Decimals}
Recurring decimals are a very interesting and nice example to demonstrate the infinite G. P. and the value can be obtained by the
formula derived in previous section. Consider a recurring decimal $\dot{7}$.

$$.\dot{7} = .777777 ... \text{to }\infty$$
$$= .7 + .07 + .007 + .0007 + \ldots$$
$$= \frac{7}{10} + \frac{7}{100} + \frac{7}{1000} + \ldots$$
$$= \frac{7}{10} + \frac{7}{10^2} + \frac{7}{10^3} + \ldots$$
$$= 7\left(\frac{1}{10} + \frac{1}{10^2} + \frac{1}{10^3} + \ldots\right)$$
$$= \frac{7}{9}$$

\subsection{Geometric Mean}
Like arithmetic means; we also have geometric means. Say two numbers $a$ and $b$ are in G.P. and $x$ is a geometric mean between
them then by definition $a, x, b$ will be in G.P. Then,
$$\frac{x}{a} = \frac{b}{x}$$
$$\Rightarrow x^2 = ab \Rightarrow x = \sqrt{ab}$$

If $G_1, G_2, \ldots, G_n$ are $n$ geometric means between two numbers $a$ and $b$, then $G_1G_2\ldots G_n = \sqrt{ab^n}$

\noindent\textbf{Proof:} $b$ is the $n + 2$nd term. Thus, $b = ar^{n + 1}$ where common ratio is $r$.

Thus, $G_1 = ar, G_2 = ar^2, \ldots, G_n = ar^n$

$G_1G_2\ldots G_n = ar^{1 + 2 + \ldots + n} = ar^{\frac{n(n + 1)}{2}}$

$= \sqrt{ab^n}$

If $a_1, a_2, \ldots, a_n$ are $n$ positive numbers in G.P. then their geometric mean is given by $G = (a_1a_2\ldots
a_n)^{\frac{1}{n}}$

Thus, first G.M. $= ar = a\left(\frac{b}{a}\right)^{1/(n + 1)}$

Second G.M. $= ar^2 = a\left(\frac{b}{a}\right)^{2/(n + 1)}$

$\ldots$

$n$th G.M. $= ar^{n} = a\left(\frac{b}{a}\right)^{n/(n + 1)}$
\subsection{Notes}
\begin{enumerate}
\item Odd number of terms in a G.P. should be taken as $\ldots\frac{a}{r^2}, \frac{a}{r}, a, ar, ar^2, \ldots$
\item Even number of terms in a G.P. should be taken as $\ldots, \frac{a}{r^5}, \frac{a}{r^3}, \frac{a}{r}, ar, ar^3, ar^5, \ldots$
\item If $a_1, a_2, \ldots, a_n$ and $b_1, b-2, \ldots, b_n$ be two G.P. of common ratios $r_1$ and $r_2$ then $a_1b_1, a_2b_2,
  a_3b_3, \ldots$ and $\frac{a_1}{b_1}, \frac{a_2}{b_2}, \frac{a_3}{b_3}, \ldots$ also form G.P., where common ratios will be
  $r_1r_2$ and $\frac{r_1}{r_2}$ respectively.
\item Let $a_1, a_2, a_3, \ldots$ be a G.P. of positive terms, then $\log a_1, \log a_2, \log a_3, \ldots$ will be an A.P. and
  vice-versa.

  Let $a$ be the first term and $r$ be the common ratio of the G.P. then $a_i = ar^{i - 1}$. Now $\log a_i = \log a + (i - 1)\log
  r$ which represents $i$th term of an A.P. with first term as $\log a$ and common difference $\log r$.

  Conversely, let us assume that $\log a_1, \log a_2, \log a_3, \ldots$ are in A.P. then $a_i = x^{a + (i - 1)d} = x^ax^{{i - 1}d}$
  where $x$ is the base of the logarithm. This shows that $a_1, a_2, a_3,\ldots$ will be in G.P., whose first term is $x^a$ and
  whos ecommon ratio is $x^d$.
\item Increasing and decreasing G.P.

  \textbf{Case I:} Let the first term $a$ be positive. Then if $r > 1$, then it is an increasing G.P. but if $0< r< 1$ then it is a
  decreasing G.P.

  \textbf{case II:} Let the first term $a$ be negative. Then if $r > 1$, then it is a decreasing G.P. but if $0 < r < 1$ then it is
  an increasing G.P.
\end{enumerate}

\subsection{Arithmetico Geometric Series}
If the termms of an A.P. are multiplied y corresponding terms of a G.P., then the new series obtained is called an
Arithmetico-Geometric series.

\textbf{Exmaple:} If the terms of the arithmetic series $2 + 5 + 8 + \ldots$ are multiplied with the corresponsing terms of the
geometric series $x + x^2 + x^3 + \ldots$ then the resulting arithmetico-geometric series is $2x + 5x^2 + 8x^3 + \ldots$

\subsection{Sum of $n$ terms of an Arithmetico-Geometric Series}
Let $a_1, a_2, \ldots, a_n$ be an A.P. and $b_1, b_2, \ldots, b_n$ be a G.P. Let $d$ be the common difference of the A.P. and $r$
be the common ratio of the G.P. Also, let $a = a_1$ and $b = b_1$, then
$$S_n = ab + (a + d)br + (a+ 2d)br^2 + \ldots + [a + (n - 1)d]br^{n - 1}$$
$$rS_n = abr + (a + d)br^2 + (a + 2d)br^3 + \ldots + [a + (n - 1)d]br^n$$
$$\Rightarrow (1 - r)S_n = ab + dbr + dbr62 + \ldots + dbr^{n - 1} - [a + (n - 1)d]br^n$$
$$= ab + \frac{dbr(1 - r^{n - 1})}{(1 - r) - [a + (n - 1)d]br^n}$$
$$S_n = \frac{ab}{1 - r} + \frac{dbr(1 - r^{n - 1})}{(1 - r)^2} - \frac{[a + (n - 1)d]br^n}{1 - r}(r\neq 1)$$

If $|r|< 1$, then $lim_{n\to \infty}r^n = 0$, therefore , sum of an infinite number of terms of an arithmetico-geometric series is
given by
$$S_\infty = \frac{ab}{1 - r} + \frac{dbr}{(1 - r)^2}$$

\section{Harmonic Progressions}
Consider an A.P. then an H.P. is formed by terms given by reciprocal of terms of the A.P. respectively. So if the terms of A.P. are
$a_1, a_2, \ldots, a_n$ then terms of H.P. are given by $\frac{1}{a_1}, \frac{1}{a_2}, \ldots, \frac{1}{a_n}$.

When we study H.P. and its properties we do that by studying the properties of the corresponding A.P.

\subsection{Harmonic Means}
Numbers $H_1, H_2, \ldots, H_n$ are said to be the $n$ H.M. between two numbers $a$ and $b$, if $a, H_1, H_2, \ldots, H_n, b$ are
in H.P. For example, $\frac{1}{2}, \frac{1}{3}, \frac{1}{4}$ are the H.M. between $1$ and $\frac{1}{5}$ because $1, \frac{1}{2},
\frac{1}{3}, \frac{1}{4}, \frac{1}{5}$ are in H.P.

Let $a$ and $b$ be the two given quantities and $H$ be the H.M. between them. Then $a, H, b$ will be in H.P.

$\therefore \frac{1}{a}, \frac{1}{H}, \frac{1}{b}$ will be in H.P.

$\frac{1}{H} - \frac{1}{a} = \frac{1}{b} - \frac{1}{H} \Rightarrow H = \frac{2ab}{a =b}$

Let $H_1, H_2, \ldots, H_n$ be the $n$ H.M. between two given quantities $a$ and $b$, and $d$ be the c.d. of the corresponding A.P.
Then $a, H_1, H_2, \ldots, H_n, b$ will be in H.P.

$\therefore \frac{1}{a}, \frac{1}{H_1}, \frac{1}{H_2}, \ldots, \frac{1}{H_n}, \frac{1}{b}$ will be in A.P.

$\frac{1}{b} = t_{n + 2} = \frac{1}{a} + (n + 1)d \Rightarrow d = \frac{a - b}{ab(n + 1)}$

$\therefore \frac{1}{H_1} = \frac{1}{a} + d \Rightarrow H_1 = \frac{ab(n + 1)}{a + nb}$

$H_2 = \frac{ab(n + 1)}{2a + (n - 1)b}$

$\ldots$

$H_n = \frac{ab(n + 1)}{an + b}$

\section{Relation between A.M., G.M. and H.M.}
Let $a$ and $b$ be two real, positive and unequal quantities and $A, G$ and $H$ be the single A.M., G.M. and H.M. between them
respectively.

Then, $A = \frac{a + b}{2}, G = \sqrt{ab}, H = \frac{2ab}{a + b}$

$AH = ab = G^2$ and thus $A, G, H$ form a G.P.

Similarly it can be probve that $A > G > H$

For equal $a$ and $b$, it can be easily verified that $A = G = H$

\section{Problems}
\begin{enumerate}
\item If $n$th term of a sequence is $2n^2 + 1,$ find the sequence. Is this seuquence in A.P.?
\item Find the first five terms of the sequence for which $t_1 = 1, t_2 = 2$ and $t_{n + 2} = t_n + t_{n + 1}$.
\item Write the sequence whose $n$th term is $3n + 5$.
\item Write the sequence whose $n$th term is $2n^2 + 3$.
\item Write the sequence whose $n$th term is $\frac{3n}{2n + 4}$.
\item Write the first three terms of sequence defined by $t_1 = 2, t_{n + 1} = \frac{2t_n + 1}{t_n + 3}$.
\item If $n$th term of a sequence is $4n^2 + 1$, find the sequence. Is this sequence an A.P.?
\item If $n$th term of a sequence is $2an + b$, where $a, b$ are constants, is this sequence an A.P.?
\item Find the 5th term of the sequence whose first three terms are $3, 3, 6$ and each term after the second is the sum of two
  preceding terms.
\item Consider the sequence defined by $t_n = an^2 + bn + c$. If $t_1 = 1, t_2 = 5$ and $t_3 = 11$ then find the value of
  $t_{10}$.
\item Show that the seuquence $9, 12, 15, 18, \ldots$ is an A.P. Find its $16^{th}$ term and the general term.
\item Show that the sequence $\log a, \log (ab), \log(ab^2), \log (ab^3), \ldots$ is an A.P. Find its $n^{th}$ term.
\item Find the sum to $n$ terms of the sequence $\langle t_n \rangle$, where $t_n = 5 -6n, n\in N$.
\item How many terms are there in the A.P. $3, 7, 11, \ldots, 407?$
\item If $a, b, c, d, e$ are in A.P. find the value of $a - 4b + 6c - 4d + e$.
\item In a certain A.P. $5$ times the $5$th term is equal to $8$ times the $8$th term, then prove that $13$th term is
  zero.
\item Find the term of the series $25, 22\frac{3}{4}, 20\frac{1}{2},18\frac{1}{4}, \ldots$ which is numerically smallest positive
  number.
\item A person was appointed in the pay scale of Rs. $700 - 40 - 1500$. Find in how many years he will reach the maximum of the
  scale.
\item Find the A.P. whose $7$th and $13$th terms are respectively $34$ and $64$.
\item Is $55$ a term of the sequence $1, 3, 5, 7, \ldots$? If yes, find which term it is.
\item Find the first negative term of the sequence $2000, 1995, 1990, \ldots$
\item How many terms are identical in two arithmetic progressions $2, 4, 6, 8, \ldots$ up to $100$ terms and $3, 6, 9, \ldots$ up
  to $80$ terms.
\item Find the number of all positive integers of $3$ digits which are divisible by $5$.
\item Is $105$ a term of the arithmetic progression $4, 9, 14, \ldots?$
\item Find the first negative term of the sequence $999, 995, 991, \ldots$.
\item Each of the series $3 + 5 + 7 + \ldots$ and $4 + 7 + 10 + \ldots$ is continued to $100$ term. Find how many terms are
  identical?
\item If $m$ times the $m$th term of an A.P. is equal to $n$ times the $n$th term, find its $(m + n)$th term.
\item If $a, b, c$ be the $p$th, $q$th and $r$th terms respectively of an A.P., prove that $a(q - r) + b(r - p) + c(p - q) =
  0$.
\item Find the number of integers between $100$ and $1000$ that are divisible by $7$ and not divisible by $7$.
\item If $a, b, c$ be the $p$th, $q$th and $r$th terms respectively of an A.P., prove that $(a - b)r + (b - c)p + (c - a)q =
  0$.
\item The sum of three numbers in A.P. is $27$ and the sum of their squares is $293.$ Find the numbers.
\item The sum of four integers in A.P. is $24$ and their product is $945.$ Find the numbers.
\item If the $p$th term of an A.P. is $q$ and the $q$th term is $p$, find the first term and common difference. Also, show that $(p
  + q)$th term is zero.
\item For an A.P. show that $t_m + t_{2n + m} = 2t_{m + n}$.
\item Divide $15$ into three parts which are in A.P. and the sum of their squares is $83$.
\item Three numbers are in A.P. Their sum is $27$ and the sum of their squares is $275$. Find the numbers.
\item The sum of three numbers in A.P. is $12$ and the sum of their cubes is $408$. Find the numbers.
\item Divide $20$ into four parts which are in A.P. such that the product of first and fourth is to product of second and third is
  $2:3$.
\item The sum of three numbers in A.P. is $-3$ and their product is $8$. Find the numbers.
\item Divide $32$ into four parts which are in A.P. such that the ratio of product of extremes to the product of means is $7:15$.
\item If $(b + c - a)/a, (c + a - b)/b, (a + b - c)/c$ are in A.P. then prove that $1/a, 1/b, 1/c$ are also in A.P.
\item If $a, b, c \in R+$ form an A.P., then prove that $a + 1/bc, b + 1/ca, c + 1/ab$ are also in A.P.
\item If $a, b, c$ are in A. P., then prove that $a^2(b + c), b^2(c + a), c^2(a + b)$ are also in A.P.
\item If $a, b, c$ are in A.P., then prove that $\frac{1}{\sqrt{b} + \sqrt{c}}, \frac{1}{\sqrt{c} + \sqrt{a}}, \frac{1}{\sqrt{a} +
  \sqrt{b}}$ are also in A.P.
\item If $a, b, c$ are in A.P., then prove that $a\left(\frac{1}{b} + \frac{1}{c}\right), b\left(\frac{1}{c} + \frac{1}{a}\right),
  c\left(\frac{1}{a} + \frac{1}{b}\right)$ are also in A.P.
\item If $(b - c)^2, (c - a)^2, (a - b)^2$ are in A.P. then prove that $\frac{1}{b - c}, \frac{1}{c - a}, \frac{1}{a - b}$ are also
  in A.P.
\item If $a, b, c$ are in A.P. then prove that $b + c, c + a, a + b$ are also in A.P.
\item If $a^2, b^2, c^2$ are in A.P. then prove that $\frac{1}{b+c}, \frac{1}{c+a}, \frac{1}{a + b}$ are in A.P.
\item If $a, b, c$ are in A.P., show that $2(a - b) = a - c = 2(b - c)$.
\item If $a, b , c$ are in A.P., then prove that $(a - c)^2 = 4(b^2 - ac)$.
\item In an A.P. if $S_1 = t_1 + t_2 + \ldots + t_n$ ($n$ odd), $S_2 = t_2 + t_4 + \ldots + t_{n - 1},$ then find the value of
  $S_1/S_2$ in terms of $n$.
\item Find the degree of the polynomial $(1 + x)(1 + x^6)(1 + x^{11})\ldots (1+ x^{101})$.
\item Prove that a sequence is an A.P. if the sum of its terms is of the form $An^2 + Bn,$ where $A, B$ are constants.
\item If the sequence $a_1, a_2, \ldots, a_n$ form an A.P., then prove that $a_1^2 - a_2^2 + a_3^2 - a_4^2 + \ldots + a_{2n - 1}^2
  - a_{2n}^2 = \frac{n}{2n - 1}(a_1^2 - a_{2n}^2)$.
\item Find the sum of first $24$ terms of the A.P. $a_1, a_2, a_3, \ldots, a_{24},$ if it is known that $a_1 + a_5 + a_{10} +
  a_{15} + a_{20} + a_{24} = 225$
\item If the arithmetic progression whose common difference is non-zero, the sum of first $3n$ terms is equal to next $n$
  terms. Then, find the ratio of sum of first $2n$ terms to the sum of next $2n$ terms.
\item If the sum of $n$ terms of a series be $5n^2 + 3n,$ find its $n$th term. Are the terms of this series in A.P.?
\item Find the sum of the series $(a + b)^2 + (a^2 + b^2) + (a - b)^2 + \ldots$ to $n$ terms.
\item Find $1 - 3 + 5 - 7 + 9 - 11 + \ldots$ to $n$ terms.
\item The interior angles of a polygon are in A.P. The smallest angle is $120$\textdegree and the commnon difference is
  $5$\textdegree. Find the number of sides of the polygon.
\item $25$ trees are planted in a straight line at intervals of $5$ meters. To water them the gardener must bring water for each
  tree separately from a well $10$ meters from the first tree. How far he will have to travel to water all the trees beginning with
  the first if he starts from the well.
\item If $a$ be the first term of an A.P. and the sum of its first $p$ terms is equal to zero, show that the sum of the next $q$
  terms is $-\frac{a(p + q)}{p - 1}q$.
\item The sum of the first $p$ terms of an A.P. is equal to the sum of its first $q$ terms, prove that the sum of its first $(p +
  q)$ terms is zero.
\item Prove that the sum of latter half of $2n$ terms of a series in A.P. is equal to the one third of the sum of first $3n$ terms.
\item If $S_1, S_2, S_3, \ldots, S_p$ be the sum of $n$ terms of arithmetic progressions whose first terms are respectively $1, 2,
  3, \ldots$ and common differences are $1, 2, 3, \ldots$ prove that $$S_1 + S_2 + S_3 + \ldots + S_p = \frac{np}{4}(n+1)(p+1)$$
\item If $a,b$ and $c$ be the sum of $p, q$ and $r$ terms rspectively of an A.P., prove that $$\frac{a}{p}(q - r) + \frac{b}{q}(r -
  p) + \frac{c}{r}(p - q) = 0$$
\item If the sum of $m$ terms of an A.P. is equal to half the sum of $(m + n)$ terms and is also equal to half the sum of $(m + p)$
  terms, prove that $(m + n)\left(\frac{1}{m} - \frac{1}{p}\right) = (m + p)\left(\frac{1}{m} - \frac{1}{n}\right)$.
\item If there are $(2n + 1)$ terms in an A.P., then prove that the ratio of sum of odd terms and the sum of even terms is $n + 1:
  n$.
\item The sum of $n$ terms of two series in A.P. are in the ration $(3n - 13): (5n + 21)$. Find the ratio of their $24$th terms.
\item If the $m$th term of an A.P. is $\frac{1}{n}$ and $n$th term of an A.P. is $\frac{1}{m}$ then prove that the sum to $mn$
  terms is $\frac{mn + 1}{2}$.
\item If the sum of $m$ terms of an A.P.is $n$ and the sum of its $n$ terms is $m$, show that sum of $(m + n)$ terms is $-(m + n)$.
\item If $S$ be the sum of $2n + 1$ terms of an A.P., and $S_1$ that of alternate terms beginning with the first, then show that
  $\frac{S}{S_1} = \frac{2n + 1}{n + 1}$
\item If $a, b, c$ be the $1$st, $3$rd, $n$th terms respectively of an A.P., prove that the sum of $n$ terms is $\frac{c + a}{2} +
  \frac{c^2 - a^2}{b - a}$.
\item The sum of $n$ terms of two series in A.P. are in ratio $(3n + 8):(7n + 15)$. Find the ratio of their $12$th terms.
\item If the ratio of the sum of $m$ terms and $n$ terms of an A.P. is $m^2:n^2,$ prove that the ratio of its $m$th and $n$th term
  wil be $(2m - 1):(2n -1)$.
\item How many terms are in the G.P. $5, 20, 80, ..., 5120$?
\item How many terms are in the G.P. $0.03, 0.06, 0.12, \ldots, 3.84$?
\item A boy agrees to work at the rate of one rupee the first day, two rupee the second day, four rupees the third day, eight
  rupees the fourth day and so on. How much would he get on $20th$ day?
\item The population of a city in January $1987$ was $20,000$. It increased at the rate of $2\%$ per annum. Find the population of
  the city in January $1997$.
\item The sum of $n$ terms of a sequence is $2^n - 1,$ find its $n$th term. Is the sequence in G.P.?
\item If the fifth term of a G.P. is $81$ and second term is $24$. Find the G.P.
\item The seventh term of a G.P. is $8$ times the fourth term. Find the G.P. when its $5$th term is $48$.
\item If the $5$th and $8$th terms of a G.P. be $48$ and $384$ respectively, find the G.P
\item If the $6$th and $10$th terms of a G.P. are $\frac{1}{16}$ and $\frac{1}{256}$ respectively, find the G.P.
\item If the $p$th, $q$th and $r$th terms of a G.P. be $a, b, c (a, b, c >0)$, then prove that $(q - r)\log a + (r - p)\log b + (p
  - q)\log c = 0$.
\item If the $(p + q)$th term of a G.P. is $a$ and the $(p - q)$th term is $b$, show that its $p$th term is $\sqrt{ab}$.
\item If the $p$th, $q$th and $r$th terms of a G.P. be $x, y$ and $z$ respectively, prove that $x^{q - r}.y^{r - p}.z^{p - q} = 1$.
\item The first term of a G.P. is $1$. The sum of third and fifth terms is $90$. Find the common ratio of G.P.
\item Fifth term of a G.P. is $2$. Find the product of its first nine terms.
\item The fourth, seventh and last term of a G.P. are $10, 80$ and $2560$ respectively. Find the first term and number of terms in
  the G.P.
\item Three numbers are in G.P. If we double the middle term they form an A.P. Find the common ratio of the G.P.
\item If $p, q$ and $r$ are in A.P. show that $p$th, $q$th and $r$th term of a G.P. are in G.P.
\item If $a, b, c$ and $d$ are in G.P., show that $(ab + bc + cd)^2 = (a^2 + b^2 + c^2)(b^2 + c^2 + d^2)$.
\item Three non-zero numbers $a, b$ and $c$ are in A.P. Increasing $a$ by 1 or increading $c$ by 2,the numbers are in G.P. Then
  find $b$.
\item Three numbers are in G.P. whose sum is $70$. If the extremes be each multiplied by $4$ and the mean by $5$, they will be in
  A.P. Find the numbers.
\item If the product of three numbers in G.P. be $216$ and their sum is $19$, find the numbers.
\item A number consists of three digits in G.P. The sum of the right hand and left hand digits exceed twice the middle digit by $1$
  and the sum of left hand and middle digit is two-third of the sum of the middle and right hand digits. Find the number.
\item In a set of four numbers, the first three are in G.P. and the last three are in A.P. with a common difference of $6$. If the
  first number is same as fourth, find the four numbers.
\item The sum of three numbers in G.P. is $21$ and the sum of their squares is $189$. Find the numbers.
\item The prodduct of three consecutive terms of a G.P. is $-64$ and the first term is four times the third. Find the terms.
\item Three numbers whose sum is $15$ are in A.P. If $1, 4, 19$ be added to them respectively the resulting numbers are in
  G.P. Find the numbers.
\item From three numbers in G.P. other three numbers in G.P. are subtracted. Resulting numbers are found to be in G.P. again. Prove
  that the three sequences have the same common ratio.
\item If $a, b, c, d$ are in G.P., show that $(b - c)^2 + (c - a)^2 + (d - b)^2 = (a - d)^2$.
\item If $a,b,c,d$ are in G. P., then show that $(a^2 + b^2 + c^2)(b^2 + c^2 + d^2) = (ad + bc + cd)^2$.
\item If $a^x = b^y = c^z$ where $x, y, z$ are in G.P., show that $\log_ba = \log_cb$.
\item If the continued product of three numbers in a G.P. is $216$ and the sum of their products in pairs is $156$, find the
  numbers.
\item If $a, b, c, d$ are in G.P., show that $(a + b)^2, (b + c)^2, (c + d)^2$ are in G.P.
\item If $a, b, c, d$ are in G.P., show that $(a - b)^2, (b - c)^2, (c - d)^2$ are in G.P.
\item If $a, b, c, d$ are in G.P., show that $a^2 + b^2 + c^2, ab + bc + cd, b^2 + c^2 + d^2$ are in G.P.
\item If $a, b, c, d$ are in G.P., show that $\frac{1}{(a + b)^2}, \frac{1}{(b + c)^2}, \frac{1}{(c + d)^2}$ are in G.P.
\item If $a, b, c, d$ are in G.P., show that $a(b - c)^3 = d(a - b)^3$.
\item If $a, b, c, d$ are in G.P., show that $(a + b + c + d)^2 = (a + b)^2 + (c + d)^2 + 2(b + c)^2$.
\item If $a, b, c$ are in G.P., show that $a^2b^2c^2\left(\frac{1}{a^3} + \frac{1}{b^3} + \frac{1}{c^3}\right) = a^3 + b^3 + c^3$.
\item If $a, b, c$ are in G.P., show that $(a^2 - b^2)(b^2 + c^2) = (b^2 - c^2)(a^2 + b^2)$.
\item If $a, b, c$ are in G.P., show that $\log a, \log b, \log c$ are in A.P.
\item Find $1 + \frac{1}{2} + \frac{1}{4} + \frac{1}{8} + \ldots$ to $n$ terms.
\item Find $1 + 2 + 4 + 8 + \ldots$ to $12$ terms.
\item Find $1 - 3 + 9 - 27 + \ldots$ to $9$ terms.
\item Find $1 + \frac{1}{3} + \frac{1}{9} + \frac{1}{27} \ldots$ to $n$ terms.
\item Find the sum of $n$ terms of the series $(a + b) + (a^2 + 2b) + (a^3 + 3b) + \ldots$ to $n$ terms.
\item A man agrees to work at the rate of one dollar the first day, two dollars the second day, four dollars the third day, eight
  dollars the fourth day and so on. How much would he get at the end of $120$ days.
\item Find the sum to $n$ terms of the series $8 + 88 + 888 + \ldots$.
\item  Find the sum to $n$ terms of the series $6 + 66 + 666 + \ldots$.
\item Find the sum to $n$ terms of the series $4 + 44 + 444 + \ldots$.
\item Find the sum to $n$ terms of the series $.5 + .55 + .555 + \ldots$.
\item Find $1 - \frac{1}{2} + \frac{1}{4} - \frac{1}{8}$ to $n$ terms.
\item If you had a choice of a salary of a salary of $\$ 1000$ a day for a month of $31$days or $\$ 1$ for the first day, doubling
  every day which choice would you make?
\item How many terms of the series $1 + 3 + 3^2 + 3^3 + \ldots$ must be taken to make $3280$?
\item Find the least value of $n$ for which $1 + 3 + 3^2 + \ldots + 3^{n - 1} > 1000$.
\item Find $1 + \frac{1}{2} + \frac{1}{4} + \frac{1}{8}$ to $\infty$.
\item A person starts collecting \$ 1 first day, \$ 3 second day, \$ 9 third day and so on. What will be his collection in $20$
  days.
\item Find the sum of $\left(x^2 + \frac{1}{x^2} + 2\right) + \left(x^4 + \frac{1}{x^4} + 5\right) + \left(x^6 + \frac{1}{x^6} +
  8\right) + \ldots$ to $n$ terms.
\item How many terms of the series $1 + 2 + 2^2 + \ldots$ must be taken to make $511?$
\item Find the least value of $n$ such that $1 + 2 + 2^2 + \ldots + 2^{n - 1} \geq 300$.
\item Determine the no. of terms of a G.P. if $a_1 = 3, a_n = 96$ and $S_n = 189$.
\item Prove that $a^n - b^n$ is divisible by $a - b$ for any $n \in N$.
\item Prove that $a^n + b^n$ is divisible by $a + b$ for any $n \in N$.
\item Express $0.4\dot{2}\dot{3}$ as a rational number.
\item Find $\frac{1}{5} + \frac{1}{7} + \frac{1}{5^2} + \frac{1}{7^2}$ to $\infty$.
\item Prove that the sum of $n$ terms of the series $11 + 103 + 1005+ \ldots$ is $\frac{10}{9}(10^n - 1) + n^2$.
\item Find the sum to $n$ terms of the series $\left(x + \frac{1}{x}\right)^2 + \left(x^2 + \frac{1}{x^2}\right)^2 + \left(x^3 +
  \frac{1}{x^3}\right)^2 + \ldots$.
\item If $S$ be the sum, $P$ be the product and $R$ the sum of reciprocals of $n$ terms in G.P., prove that $P^2 =
  \left(\frac{S}{R}\right)^n$.
\item Find $1 + \frac{x}{1 + x} + \frac{x^2}{(1 + x)^2} + \ldots$ to $\infty$ if $x > 0$.
\item Prove that in an infinite G.P. whose common ratio is $r$ is numerically less than one, the ratio of any term to the
  sum of all the succeediing terms is $\frac{1 - r}{r}$.
\item If $S_1, S_2, S_3, \ldots, S_p$ are the sum of infinite geometric series whose first terms are $1, 2, 3, \ldots, p$
  and whose common ratios are $\frac{1}{2}, \frac{1}{3}, \frac{1}{4}, \ldots, \frac{1}{p + 1}$ respectively, prove that $S_1 + S_2
  + S_3 + \ldots + S_p = p(p + 3)/2$.
\item If $x = 1 + a + a^2 + a^3 + \ldots~\text{to}~\infty$ and $y = 1 + b + b^2 + b^3 + \ldots~\text{to}~\infty,$ show
  that $1 + ab + a^2b^2 + a^3b^3 + \ldots~\text{to}~\infty = \frac{xy}{x + y - 1},$ where $0<a< 1$ and $0<b<1$.
\item Find the sum to infinity for the series $1 + (1 + a)r + (1 + a + a^2)r^2 + \ldots,$ where $0<a<1$ and $0<r<1$.
\item After striking the floor a certain ball rebound to $\frac{4}{5}$th of the height from which it has fallen. Find the total
  distance it travels before coming to rest if it is gently dropped from a height of $120$ meters.
\item If $a$ be the first term and $b$ be the $n$th term and $p$ be the product of $n$ terms of a G.P., show that $p^2 = (ab)^n$.
\item Show that the ratio of sum of $n$ terms of two G.P.'s having the same common ratio is equal to the ratio of their $n$th
  terms.
\item If $S_1, S_2, S_3$ be the sum of $m, 2n, 3n$ terms respectively of a G.P. show that $(S_2 - S_1)^2 = S_1(S_3 - S_2)$.
\item If $S_n$ denotes the sum of $n$ terms of a G.P.,whose first term is $a$ and common ratio is $r,$ find $S_1 + S_2 +  \ldots +
  S_{2n - 1}$.
\item The sum of $n$ terms of a series is $a.2^n - b,$ find its $n$th term. Are the terms of this series in G.P.
\item Find $\frac{1}{1 + x^2}\left[1 + \frac{2x}{1 + x^2} + \left(\frac{2x}{1 + x^2}\right)^2 + \ldots~\text{to}~\infty\right]$
  where $x\geq 0$.
\item The sum of an infinite G.P. whose common ratio is numerically less than $1$ is $32$ and the sum of their first two
  terms is $24.$ Find the terms of the G.P.
\item The sum of infinite number of terms of a decreasing G.P. is $4$ and the sum of the squares of its terms to infinity
  is $\frac{16}{3},$ find the G.P.
\item If $p(x) = (1 + x^2 + x^4 + \ldots + x^{2n - 2})/(1 + x + x^2 + \ldots + x^{n - 1})$ is a polynomial in $x$, then
  find the possible values of $n$.
\item If each term in a G.P. is twice the terms following it, then find the common ratio of the G.P.
\item If $x = a + \frac{a}{r} + \frac{a}{r^2} + \ldots \infty, y = b - \frac{b}{r} + \frac{b}{r^2} - \ldots \infty$ and $z
  = c + \frac{c}{r^2} + \frac{c}{r^4} + \ldots \infty,$ then prove that $\frac{xy}{z} = \frac{ab}{c}$.
\item A G.P. consists of an even number of terms. If the sum of all terms is $5$ times the sum of the terms occupying odd
  places, then find the common ratio.
\item If sum of $n$ terms of a G.P. is $3 - \frac{3^{n + 1}}{4^{2n}},$ then find the common ratio.
\item In an infinite G.P. whose terms are all positive, the common ratio being less than unity, prove that any term $>, =,
  <$ the sum of all the succeeding terms according as the common ratio $<, =, \frac{1}{2}$.
\item Prove that $(666\ldots n~\text{digits})^2 + 888\ldots n~\text{digits} = 444\ldots 2n~\text{digits}$.
\item Find the sum $(x + y) + (x^2 + xy + y^2) + (x^3 + x^2y + xy^2 + y^3) + \ldots$ to $n$ terms.
\item If the sum of the series $\sum_{n = 0}^\infty r^n, |r| < 1$ is $s,$ then find the sum of the series $\sum_{n=0}^\infty
  r^{2n}$.
\item If for a G.P. $t_m = \frac{1}{n^2}$ and $t_n = \frac{1}{m^2}$ then find the term $t_{\frac{m + n}{2}}$.
\item If $a, b, c$ be three successive terms of a G.P. with common ratio $r$ and $a < 0$ satisfying the condition $c > 4b
  - 3a,$ then prove that $r > 3$ or $r < 1$.
\item If $(1 - k)(1 + 2x + 4x^2 + 8x^3 + 16x^4 + 32x^5) = 1 - k^6,$ where $k \neq 1,$ then find $\frac{k}{x}$.
\item If $(a^2 + b^2 + c^2)(b^2 + c^2 + d^2) \leq (ab + bc + cd)^2,$ where $a, b, c, d$ are non-zero real numbers, then
  show that they are in G.P.
\item If $a_1, a_2, \ldots, a_n$ are $n$ non-zero numbers such that $(a_1^2 + a_2^2 + \ldots + a_{n - 1}^2)(a_2^2 + a_3^2
  + \ldots + a_n^2) \leq (a_1a_2 + a_2a_3 + \ldots + a_{n - 1}a_n)^2,$ then show that $a_1, a_2, \ldots, a_n$ are in G.P.
\item $\alpha, \beta$ be the roots of $x^2 - 3x + a = 0$ and $\gamma, \delta$ be the roots of $x^2 - 12x + b = 0$ and the
  numbers $\alpha, \beta, \gamma, \delta$ form an increasing G.P., then find the values of $a$ and $b$.
\item There are $4n + 1$ terms in a certain sequence of which the first $2n + 1$ terms are in A.P. of common difference
  $2$ and the last $2n + 1$ terms are in G.P. of common ratio $\frac{1}{2}.$ If the middle terms of both the A.P. and G.P. are same
  then find the mid term of the sequence.
\item If $f(x) = 2x + 1$ and three unequal numbers $f(x), f(2x), f(4x)$ are in G.P, then find the number of values for
  $x$.
\item Three distinct real numbers, $a, b, c$ are in G.P. such that $a + b + c = xb,$ then show that $x < -1$ or $x > 3$.
\item If $x = \sum_{n=0}^\infty a^n, y = \sum_{n=0}^\infty b^n, z = \sum_{n=0}^\infty c^n$ where $a, b, c$ are in A.P.,
  such that $|a| < 1, |b|< 1, |c| < 1,$ then show that $\frac{1}{x}, \frac{1}{y}, \frac{1}{z}$ are in A.P. as well.
\item Given that $0 < x< \frac{\pi}{4}, \frac{\pi}{4} < y <\frac{\pi}{2}$ and $\sum_{k=0}^\infty(-1)^k\tan^{2k}x = p,
  \sum_{k=0}^\infty(-1)^k\cot^{2k}y = q$ then prove that $\sum_{k = 0}^\infty \tan^{2k}x\cot^{2k}y$ is $\frac{1}{\frac{1}{p} +
    \frac{1}{q} - \frac{1}{pq}}$
\item An equilateral triangle is drawn by joining the mid-points of a given equilateral triangle. A third equilateral
  triangle is drawn inside the second in the same manner and the process is continued indefinitely. If the side of first
  equilateral triangle is $3^{1/4}$ inch, then find the sum of areas of all these triangles.
\item If $S = exp(1 + |\cos x| + \cos^2x + |\cos^3x| + \cos^4x \ldots~\text{to}~\infty)\log_c 4$ satisfies the roots of
  the equation $t^2 - 20t + 64 = 0$ for $0 < x< \pi$ then find the values of $x$.
\item If $S\subset (-\pi, \pi),$ denote the set of values of $x$ satisfying the equation

  $8^{1 + |\cos x| + \cos^2x + |\cos^3x| + \ldots~\text{to}~\infty} = 4^3$ then find the value of $S$.
\item If $0 < x <\frac{\pi}{2}$ and $2^{\sin^2x + \sin^4x + \ldots~\text{to}~\infty}$ satisfies the roots of the equation
  $x^2 - 9x + 8 = 0,$ then find the value of $\cos x/(\cos x + \sin x)$.
\item If $S_\lambda = \sum_{r=0}^\infty \frac{1}{\lambda^r},$ then find $\sum_{\lambda = 1}^n(\lambda - 1)S_\lambda$.
\item If $a, b, c$ are in A.P. then prove that $2^{ax + 1}, 2^{bx + 1}, 2^{cx + 1}$ are in G.P. $\forall x\neq 0$.
\item If $\frac{a + be^x}{a - be^x} = \frac{b + ce^x}{b - ce^x} = \frac{c + de^x}{c - de^x}$ then prove that $a, b, c, d$
  are in G.P.
\item If $x, y, z$ arein G.P. and $\tan^{-1}x, \tan^{-1}y, \tan^{-1}z$ are in A.P. then prove that $x = y = z$ but their
  common values are not necessarily zero.
\item If $a, b, c$ are three unequal numbers such that $a, b, c$ are in A.P. and $b - a, c - b, a$ are in G.P. then prove
  that $a:b:c = 1:2:3$.
\item The sides $a,b,c$ of a triangle are in G.P. sych that $\log a - \log 2b, \log 2b - \log 3c, \log 3c - a$ are in
  A.P., then prove that $\triangle ABC$ is an obtuse angled triangle.
\item If the roots of the equation $ax^3 + bx^2 + cx + d = 0$ be in G.P. then prove that $c^3a = b^3d$.
\item Find the $100$th term of the sequence $1, \frac{1}{3}, \frac{1}{5}, \frac{1}{7}, \ldots$.
\item If $p$th term of an H.P. is $qr,$ and $q$th term is $rp,$ prove that $r$th term is $pq$.
\item If the $p$th, $q$th and $r$th terms of an H.P. be respectively $a, b$ and $c,$ then prove that $(q - r)bc + (r - p)ca + (p -
  q)ab = 0$.
\item If $a, b, c$ are in H.P., prove that $\frac{a - b}{b - c} = \frac{a}{c}$.
\item If $a, b, c, d$ are in H.P., then, prove that $ab + bc + cd = 3ad$.
\item If $x_1, x_2, x_3, \ldots, x_n$ are in H.P., prove that $x_1x_2 + x_2x_3 + x_3x_4 + \ldots + x_{n - 1}x_n = (n - 1)x_1x_n$.
\item If $a, b, c$ are in H.P., show that $\frac{a}{b + c}, \frac{b}{c + a}, \frac{c}{a + b}$ are in H.P.
\item If $a^2, b^2, c^2$ are in A.P. show that $b + c, c + a, a + b$ are in A.P.
\item Find the sequence whose $n$th term is $\frac{1}{3n - 2}.$ Is tihs sequence an H.P.?
\item If $m$th term of an H.P. be $n$ and $n$th term be $m,$ prove that $(m + n)$th term $= \frac{mn}{m + n}$ and $(mn)$th term $=
  1$.
\item The sum of three rational numbers in H.P. is $37$ and the sum of their reciprocals is $\frac{1}{4},$ find the numbers.
\item If $a, b, c$ are in H.P., prove that $\frac{1}{b - a} + \frac{1}{b - c} = \frac{1}{a} + \frac{1}{c}$.
\item If $a, b, c$ are in H.P., prove that $\frac{b + a}{b - a} + \frac{b + c}{b - c} = 2$.
\item If $x_1, x_2, x_3, x_4, x_5$ are in H.P., prove that $x_1x_2 + x_2x_3 + x_3x_4 + x_4x_5 = 4x_1x_5$.
\item If $x_1, x_2, x_3, x_4$ are in H.P., prove that $(x_1 - x_3)(x_2 - x_4) = 4(x_1 - x_2)(x_3 - x_4)$.
\item If $b + c, c + a, a + b$ are in H.P., prove that $\frac{a}{b + c}, \frac{b}{c + a}, \frac{c}{a + b}$ are in A.P.
\item If $b + c, c + a, a + b$ are in H.P., prove that $a^2, b^2, c^2$ are in A.P.
\item If $a, b, c$ are in A.P., prove that $\frac{bc}{ab + ac}, \frac{ca}{bc + ab}, \frac{ab}{ca + cb}$ are in H.P.
\item If $a, b, c$ are in H.P., prove that $\frac{a}{b + c - a}, \frac{b}{c + a - b}, \frac{c}{a + b - c}$ are in H.P.
\item If $a, b, c$ are in H.P., prove that $\frac{a}{b + c}, \frac{b}{c + a}, \frac{c}{a + b}$ are in H.P.
\item Find the sum of $n$ terms of the series whose $n$th term is $12n^2 - 6n + 5$.
\item Find the sum to $n$ terms of the series $1^2 + 3^2 + 5^2 + 7^2 + \ldots$.
\item Find the sum to $n$ terms of the series $1.2.3 + 2.3.4 + 3.4.5 + \ldots$.
\item Find the sum of the series $1.n + 2.(n - 1) + 3.(n - 2) + \ldots + n.1$.
\item Find the sum to $n$ terms of the series $1 + (1 + 2) + (1 + 2 + 3) + \ldots$.
\item Find the sum to $n$ terms of the series $1 + (2 + 3) + (4 + 5 + 6) + \ldots$.
\item Find the sum of series $\frac{1^3}{1} + \frac{1^3 + 2^3}{1 + 3} + \frac{1^3 + 2^3 + 3^3}{1 + 3 + 5} + \ldots$ to $16$ terms.
\item Find $(3^3 - 2^3) + (5^3 - 4^3) + (7^3 - 6^3) + \ldots$ to $10$ terms.
\item Find $\frac{1}{1.2} + \frac{1}{2.3} + \frac{1}{3.4} + \ldots$ to $n$ terms.
\item Find the sum of $\frac{1}{1.2.3} + \frac{1}{2.3.4} + \frac{1}{3.4.5} + \ldots$ to infinity.
\item Find the sum of $n$ terms of the series $1 + 5 + 11 + 19 + \ldots$.
\item A sum is distributed among certain number of persons. Second person gets one rupee more than the first, third person
  gets two rupees more than the second, fourth person gets three rupees more than the third and so on. If the first person gets one
  rupee and the last person get $67$ rupees, find the number of persons.
\item Natural numbers have been grouped in the following way $1, (2, 3), (4, 5, 6), (7, 8, 9, 10), \ldots$ Show that the
  sum of the numbers in the $n$th group is $\frac{n(n^2 + 1)}{2}$.
\item Find $1 + 3 + 7 + 15 + \ldots$ to $n$ terms.
\item Find $1 + 2x + 3x^2 + 4x^3 + \ldots$ to $n$ terms.
\item Find $1 + 2.2 + 3.2^2 + 4.3^3 + \ldots + 100.2^{99}$.
\item Find $1 + 2^2x + 3^2x^2 + 4^2x^4 + \ldots \text{~to~}\infty, |x| < 1$
\item If the sum of $n$ terms of a sequence be $2n^2 + 4,$ find its $n$th term. Is this sequence in A.P.?
\item Find the sum of $n$ terms of the series whose $n$th term is $n(n - 1)(n + 1)$.
\item Find the sum of $80$ terms of the series whose $n$th term is $n(n^2 - 1)$.
\item Find the sum of the series $1^3 + 3^3 + 5^3 + \ldots$ to $n$ terms.
\item Find the sum of the series $1^2 + 4^2 + 7^2 + 10^2 + \ldots$ to $n$ terms.
\item Find the sum of the series $1^2 + 2 + 3^2 + 4 + 5^2 + 6 + \ldots$ to $2n$ terms.
\item Find the sum of the series $1^2 - 2^2 + 3^2 - 4^2 + \ldots$ to $n$ terms.
\item Find the sum of the series $1.3 + 3.5 + 5.7 + \ldots$ to $n$ terms.
\item Find the sum of the series $1.2 + 2.3 + 3.4 + \ldots$ to $n$ terms.
\item Find the sum of the series $1.2^2 + 2.3^2 + 3.4^2 + \ldots$ to $n$ terms.
\item Find the sum of the series $2.1^2 + 3.2^2 + 4.3^2 + \ldots$ to $n$ terms.
\item Find the sum of the series $1 + (1 + 3) + (1 + 3 + 5) + \ldots$ to $n$ terms.
\item Find the sum of the series $1^2 + (1^2 + 2^2) + (1^2 + 2^2 + 3^2) + \ldots$ to $n$ terms.
\item Find the sum of the series $1.2.3 + 2.3.5 + 3.4.7 + \ldots$ to $n$ terms.
\item Find the sum of the series $1.2.3 + 2.3.4 + 3.4.5 + \ldots$ to $n$ terms.
\item Find the sum of the series $1.3^2 + 2.5^2 + 3.7^2 + \ldots$ to $20$ terms.
\item Find the sum of the series $(n^2 - 1^2) + 2(n^2 - 2^2) + 3(n^2 - 3^2) + \ldots$ to $n$ terms.
\item Find the sum of the series $1^2 + (1^2 + 2^2) + (1^2 + 2^2 + 3^2) + \ldots$ to $10$ terms.
\item Find the sum of the series $(3^3 - 2^3) + (5^3 - 4^3) + (7^3 - 6^3) + \ldots$ to $10$ terms.
\item Find the sum of the series $1 + \frac{1}{1.2} + \frac{1}{1 + 2 + 3} + \ldots$ to $n$ terms.
\item Find the sum to infinity of the series $\frac{1}{2.4} + \frac{1}{4.6} + \frac{1}{6.8} + \frac{1}{8.10} + \ldots$.
\item Find the sum of the series $2 + 6 + 12 + 20 + \ldots$ to $n$ terms.
\item Find the sum of the series $3 + 6 + 11 + 18 + \ldots$ to $n$ terms.
\item Find the sum of the series $1 + 9 + 24 + 46 + 75 + \ldots$ to $n$ terms.
\item Find the $n$th term of the series $2 + 4 + 7 + 11 + 16 + \ldots$.
\item Find the sum to $10$ terms of the series $1 + 3 + 6 + 10 + \ldots$.
\item The odd natural numbers have been divided in groups as $(1, 3), (5, 7, 9, 11)$,

  $(13, 15, 17, 19, 21, 23), \ldots$
  Show that the sum of numbers in the $n$th group is $4n^3$.
\item Show that the sum of numbers in each of the following groups is an square of an odd positive integer $(1), (2,3,4),
  (3,4,5,6,7), \ldots$.
\item Find the sum to $n$ terms of the series $2 + 5 + 14 + 41 + \ldots$.
\item Find the sum to $n$ terms of the series $1.1 + 2.3 + 4.5 + 8.7 + \ldots$.
\item If $a_1, a_2, a_3, \ldots, a_{2n}$ are in A.P., show that $a_1^2 - a_2^2 + a_3^2 - a_4^2 + \ldots + a_{2n - 1}^2 -
  a_{2n}^2 = \frac{n}{2n - 1}(a_1^2 - a_{2n}^2)$.
\item If $\alpha_1, \alpha_2, \alpha_3, \ldots, \alpha_n$ are in A.P., whose common difference is $d$ show that $\sin
  d$

  $[\sec\alpha_1\sec\alpha_2 + \sec\alpha_2\sec\alpha_3$ $+ \ldots + \sec\alpha_{n - 1}\sec\alpha_n] = \tan\alpha_n -
  \tan\alpha_1$.
\item If $a_1, a_2, a_3, \ldots, a_n$ be in A.P., prove that $\frac{1}{a_1a_n} + \frac{1}{a_2a_{n - 1}} + \ldots +
  \frac{1}{a_na_1} =\frac{2}{a_1 + a_n}$

  $\left(\frac{1}{a_1} + \frac{1}{a_2} + \ldots + \frac{1}{a_n}\right)$.
\item If $a_1, a_2, a_3, \ldots$ be in A.P. such that $a_i\neq 0,$ show that $S = \frac{1}{a_1a_2} + \frac{1}{a_2a_3}
  + \ldots + \frac{1}{a_na_{n + 1}} = \frac{n}{a_1a_{n + 1}}$.
\item If $a_1, a_2, a_3, \ldots, a_n$ be in A.P. and $a_1 = 0,$ show that $\frac{a_3}{a_2} + \frac{a_4}{a_3} + \ldots +
  \frac{a_n}{a_{n - 1}} - a_2\left(\frac{1}{a_2} + \frac{1}{a_3} + \ldots + \frac{1}{a_{n - 2}}\right) = \frac{a_{n - 1}}{a_2} +
  \frac{a_2}{a_{n - 1}}$.
\item If $a_1, a_2, \ldots, a_n$ are in A.P., whose common difference is $d,$ show that $\sum_{k = 1}^n\frac{a_ka_{k +
    1}a_{k + 2}}{a_k+a_{k + 2}}$ $= \frac{n}{2}\left[a_1^2 + (n + 1)a_1d + \frac{(n - 1)(2n + 5)}{6}d^2\right]$.
\item If $x, y$ and $z$ are positive real numbers different from $1,$ and $x^{18} = y^{21} = z^{28},$ show that $3, 3\log_y
  x, 3\log_z y, 7\log_x z$ are in A.P.
\item If $I_n = \int_{0}^{\frac{\pi}{2}}\frac{\sin^2nx}{\sin^2x}dx,$ then $I_1, I_2, I_3, \ldots$ are in A.P.
\item Can there be an A.P. whose terms are distinct prime numbers?
\item Four distinct no. are in A.P. If one of these integers is sum of the squares of remaining three, then $0$ must be
  one of the numbers in A.P.
\item In an A.P. of $2n$ terms the middle pair of terms are $p + q$ and $p - q.$ Show that the sum of cubes of the terms
  in A.P. are $2np[p^2 + (4n^2 - 1)q^2]$.
\item Find the sum $S_n$ of the cubes of the first $n$ terms of an A.P. and show that the sum of the first $n$ terms of
  the A.P. is a factor of $S_n$.
\item Show that any positive integral power (greater than $1$) of a positive integer $m,$ is the sum of $m$ consecutive
  odd positive integers. Find the first odd integer for $m^r(r > 1)$.
\item If $a$ be the sum of $n$ terms and $b^2$ the sum of the square of $n$ terms of an A.P., find the first term and
  common difference of the A.P.
\item If $a_1, a_2, \ldots, a_n$ are in A.P., whose common diference is $d$, then find the sum of the series $\sin
  d[\csc a_1\csc a_2 + \csc a_2\csc a_3 + \ldots + \csc a_{n - 1}\csc a_n]$.
\item If $a_1, a_2, \ldots, a_n$ are in A.P. where $a_i > 0~\forall i,$ show that $$\frac{1}{\sqrt{a_1} + \sqrt{a_2}} +
  \frac{1}{\sqrt{a_2} + \sqrt{a_3}} + \ldots + \frac{1}{\sqrt{a_{n - 1}} + \sqrt{a_n}} = \frac{n - 1}{\sqrt{a_1} + \sqrt{a_n}}$$
\item If $a_1, a_2, \ldots, a_n$ are in A.P., whose common differemce is $d$ show that $\sum_{2}^n\tan^{-1}\frac{d}{1 +
  a_{n - 1}a_n} = \tan^{-1}\frac{a_n - a_n}{1 + a_na_1}$.
\item If $a_1, a_2, \ldots, a_n$ are the first $n$ items of an A.P. with first term $a$ and common difference $d$ such
  that $ad > 0.$ Let $S_n = \frac{1}{a_1a_2} + \frac{1}{a_2a_3} - \ldots + \frac{1}{a_{n - 1}a_n}$ Prove that the product
  $a_1a_nS_n$ does not depend on $a$ or $d$.
\item If $a_1, a_2, \ldots, a_n, a_{n + 1}, \ldots$ be in A.P., whose common difference is $d$ and $S_1 = a_1 + a_2 +
  \ldots + a_n,$ $S_2 = a_{n + 1} + \ldots + a_{2n}, S_3 = a_{2n + 1} + \ldots + a_{3n}$ Show that $S_1, S_2, S_3, \ldots$ are in
  A.P. whose common difference is $n^2d$.
\item If $a, b, c$ are three terms of an A.P. such that $a\neq b,$ show that $(b - c)/(a - b)$ is a rational number.
\item Prove that $\tan 70^\circ, \tan 50^\circ + \tan 20^\circ, \tan 20^\circ$ are in A.P.
\item If $\log_l x, \log_m x, \log_n x$ are in A.P. and $x \neq 1,$ prove that $n^2 = (nl)^{\log_l m}$.
\item The length of sides of a right angled triangle are in A.P., show that their ratio is $3:4:5$
\item If $\log_3 2, \log_3(2^x - 5), \log_3\left(2^x - \frac{7}{2}\right)$ are in A.P., determine the value of $x$.
\item Find the values of $a$ for which $5^{1 + x} + 5^{1 - x}, \frac{a}{2}, 25^x + 25^{-x}$ are in A.P.
\item If $\log 2, \log(2^x - 1)$ and $\log(2^x + 3)$ are in A.P., then find $x$.
\item If $1, \log_y x, \log_zy, -15\log_xz$ are in A.P., then prove that $x = z^3$ and $y = z^{-3}$.
\item Show that $\sqrt{2}, \sqrt{3}, \sqrt{5}$ cannot be terms of a single A.P.
\item A circle of one centimeter radius is drawn on a piece of paper and with the same center $3n - 1$ other circles are
  drawn of radii $2$ cm, $3$ cm, $4$ cm and so on. The inner circle is painted blue, the ring between that and next circle is
  painted red, the next ring yellow then other rings blue, red, yellow and so on in this order. Show that the successive aread of
  each color are in A.P.
\item If $x, y, z(x, y, z\neq 0)$ are in A.P. and $\tan^{-1}x, \tan^{-1}y, \tan^{-1}z$ are also in A.P., then prove that
  $x = y = z$.
\item If $\theta$ and $\alpha$ are two real numbers such that $\frac{\cos^4\theta}{\cos^2\alpha}, \frac{1}{2},
  \frac{\sin^4\theta}{\sin^2\alpha}$ are in A.P., prove that $\frac{\cos^{2n + 2}\theta}{\cos^{2n}\alpha}, \frac{1}{2},
  \frac{\sin^{2n + 2}\theta}{\sin^{2n}\alpha}$.
\item If $a_n = \displaystyle\int_0^\pi (\sin 2nx/\sin x)dx,$ show that $a_1, a_2, a_3, \ldots$ are in A.P.
\item If $l_n = \displaystyle\int_0^{\frac{\pi}{4}}tan^nxdx,$ show that $\frac{1}{l_2 + l_4}, \frac{1}{l_3 + l_5}, \frac{1}{l_4 + l_6},
  \ldots$ are in A.P. Find the common difference of A.P.
\item If $\alpha, \beta, \gamma$ are in A.P.and $\alpha = \sin(\beta + \gamma), \beta = \sin(\gamma + \alpha)$ and $\gamma
  = \sin(\alpha + \beta).$ Prove that $\tan \alpha = \tan \beta = \tan \gamma$.
\item Suppose $a, b, c$ are three positive real numbers in A.P., such that $abc = 4.$ Prove that the minimum value of $b$
  is $4^{\frac{1}{3}}$.
\item The sixth term of an A.P. is $2,$ and its common difference is greater than $1.$ Show that the value of the common
  difference of the progression so that the product of first, fourth and fifth terms is greatest is $\frac{8}{5}$.
\item Find the sum of $n$ terms of the series: $\log a + \log\frac{a^3}{b} + \log \frac{a^5}{b^2} + \log \frac{a^7}{b^3} +
  \ldots$.
\item The first, second and the last terms of an A.P. are $a,b, c$ respectively. Prove that the sum of al the terms is
  $\frac{(b + c - 2a)(a + c)}{2(b - a)}$.
\item If $S_n$ denotes the sum of $n$ terms of an A.P., show that $S_{n + 3} = 3(S_{n + 2} - S_{n + 1}) + S_n$.
\item If $a_1, a_2, \ldots, a_n$ are in arithmetic progression with common difference $d,$ prove that
  $\sum_{r < s}a_ra_s = \frac{1}{2}n(n - 1)[a_1^2 + (n - 1)a_1d + \frac{1}{12}(3n^2 - 7n + 2)d^2]$.
\item Balls are arranged in rows to form an equilateral triangle. The first row consists of one ball, the second of two
  balls and so on. If $669$ more balls are added, then all balls can be arranged in the shape of a square and each of the sides
  contained $8$ balls less than each side of the triangle did. Determine the initial no. of balls.
\item Find the sum of the  product of the first $n$ natural numbers takes two at a time.
\item A postman delivered daily for $42$ days $4$ more letters each day than on the previous day. The total delivery made
  for the first $24$ days of the period was the same as that for the last $18$ days. How many letters did he deliver during the
  whole period?
\item If $S_n$ denotes the sum to $n$ terms of an A.P. and $S_n = n^2p, S_m = m^2p, m\neq n,$ prove that $S_n = p^3$.
\item There are $n$ A.P.'s whose common difference are $1, 2, 3, \ldots, n$ respectively the first term of each being
  unity. Prove that the sum of their $n$th terms is $\frac{n}{2}(n^2 + 1)$.
\item If $S_1, S_2, \ldots, S_m$ are the sum of $n$ terms of $m$ A.P.s whose first terms are $1, 2, \ldots, m$ and whose
  common differences are $1, 3, 5, \ldots, 2m - 1$ respectively, show that $S_1 + S_2 + \ldots + S_m = \frac{1}{2}mn(mn + 1)$
\item A straight line is drawn through the center of a square $ABCD$ intersecting side $AB$ at point $N$ so that $AN:NB =
  1:2.$ On this line take an arbitrary point $M$ lying inside the square. Prove that the distances from $M$ to the sides $AB, AD,
  BC, CD$ of the square taken in that order, form an A.P.
\item If the sides of a right-angled triangle are in G.P., find the cosine of the greater acute angle.
\item If $a, b, c, d$ are non-zero real numbers and $(a^2 + b^2 + c^2)(b^2 + c^2 + d^2) = (ab + bc + cd)^2,$ prove that
  $a, b, c, d$ are in G.P.
\item Does there exist a geometric progression containing $27, 8$ and $12$ as three of its terms? If it exists, how many
  such progressions are possible?
\item Show that $10, 11, 12$ cannot be terms of a G.P.
\item If $\displaystyle\int_0^{\frac{\pi}{2}}\cos^nx\cos(nx)dx,$ then prove that $I_1, I_2, I_3, \ldots$ are in G.P.
\item Let $\displaystyle I_n = \int_0^\pi \frac{\sin(2n - 1)x}{\sin x}dx.$ Show that $I_1, I_2, I_3, \ldots$ are in A.P. as well as in
  G.P.
\item Prove that the three successive terms of a G.P. will form sides of a triangle if the common ratio $r$ satisfied the
  inequality $\frac{1}{2}(\sqrt{5} - 1) < r < \frac{1}{2}(\sqrt{5} + 1)$.
\item Find out whether $111\ldots1( 91$ digits $)$ is a prime number.
\item Find the natural number $a$ for which $\sum_{k = 1}^nf(a + k) = 16(2^n - 1),$ where the function $f$ satisfied the
  relation $f(x + y) = f(x)f(y)$ for all natural nuumbers $x, y$ and further $f(1) = 2$.
\item In a certain test, there are $n$ questions. In this test $2^{n - i}$ students give wrong answers to at least $i$
  questions ($1\leq i \leq n.$) If total no. of wrong answers given is $2047,$ find the value of $n$.
\item If $S_1, S_2, S_3, \ldots, S_2n$ are the sums of infinite geometric series whose first terms are respectively $1, 2,
  3, \ldots, 2n$ and common ratio are respectively $\frac{1}{2}, \frac{1}{3}, \ldots, \frac{1}{2n + 1},$ find the value of $S_1^2 +
  S_2^2 + \ldots + S_{2n - 1}^2$.
\item A sqaure is given, a second square is made by joining the middle points of the first square and then a third square
  is made by joining the middle points of the sides of second square and so on till infinity. Show that the area of first square is
  equal to sum of the areas of all the succeeding squares.
\item If $a$ is the value of $x$ for which the function $7 + 2x\log 25 - 5^{x - 1} - 5^{2 - x}$ has the greatest value and
  $r = \displaystyle\lim_{x\to 0}\int_{0}^x\frac{t^2}{x^2\tan(\pi + x)}dt,$ find $\displaystyle\lim_{n \to \infty}\sum_{n =
  1}^nar^{n - 1}$.
\item If $p$th, $q$th, $r$th terms of a G.P. are positive numbers $a, b, c$ respectively, show that the vectors $(\log
  a).\vec{i} + (\log b)\vec{j} + (\log c)\vec{k}$ and $(q - r)\vec{i} + (r - p)\vec{j} + (p - q)\vec{k}$ are perpendicular.
\item The pollution in a normal atmosphere is less that $0.01\%.$ Due to leakage of gas from a factory the pollution
  increased to $20\%.$ If everyday $80\%$ of the pollution us neutralised, in how many days the atmosphere will be normal?
\item The sides of a triangle are in G.P. and its largest angle is twice the smallest one. Prove that the common ratio of
  the G.P. lies in the interval $(1, \sqrt{2})$.
\item If $a, b, c, d$ are in G.P., then prove that $ax^3 + bx^2 + cx + d$ is divisible by $ax^2 + c$.
\item If $a, b, c$ are three distinct real numbers and they are in G.P. If $a + b + c = xb,$ then prove that $x < -1$ or
  $x > 3$.
\item If $a, b, c, d, p$ are real and $(a^2 + b^2 + c^2)p^2 - 2(ab + bc + cd)p + (b^2 + c^2 + d^2)\leq 0.$ Show that $a,
  b, c, d$ are in G.P. whose common ratio is $p$.
\item If $2x^4 = y^4 + z^4, xyz = 8$ and $\log_yx, \log_zy, \log_xz$ are in G.P., show that $x = y = z = 2$.
\item If $a, b, c, d$ are in both A.P. and G.P. and $b = 2,$ then find the number of such sequences.
\item If $\log_x a, a^{x/2}, \log_b x$ are in G.P., then find $x.$
\item The $(m + n)$th and $(m - n)$th terms of a G.P. are $p$ and $q$ respectively. Show that $m$th and $n$th terms are
  $\sqrt{pq}$ and $p\left(\frac{q}{p}\right)^\frac{m}{2n}$ respectively.
\item If the $p$th, $q$th and $r$th terms of an A.P. are in G.P., then find the common ratio of the G.P.
\item A G.P. consists of $2n$ terms. If the sum of the terms occupying the odd places is $S_1,$ and that of the terms in
  even places is $S_2,$ show that the common ratio of the progression is $S_2/S_1$.
\item If $x\neq 1, y\neq 1, x \neq y,$ find the sum to $n$ terms of the series $(x + y) + (x^2 + xy + y^2) + (x^3 + x^2y +
  xy^2 + y^3) + \ldots$.
\item Find a geometric progression of real numbers such that the sum of its first four terms is equal to $30$ and sum of
  the squares of its first four terms is $340$.
\item If $S_n$ denotes the sum of $n$ terms of a G.P. whose first term and common ratio are $a$ and $r$ respectively, show
  that $$rS_n + (1 - r)\sum_{n = 1}^nS_n = na$$
\item Find the sum of $2n$ terms of the series where every even term if $x$ times the term just before it and every odd
  term is $y$ times the term just before it, the first term being $1$.
\item Prove that in the sequence of numbers $49, 4489, 444889, \ldots$ in which every number is made by inserting $48$ in
  the middle of previous number as indicated, each number is the square of an integer.
\item If there be $m$ quantities in a G.P., whose common ratio is $r$ and $S_m$ denotes the sum of the first $m$ terms
  then prove that the sum of their products taken two and two together is $\frac{r}{r + 1}S_mS_{m - 1}$.
\item Solve the following equations for $x$ and $y$
  $$\log_{10}x + \log_{10}x^{1/2} + \log_{10}x^{1/4} + \ldots = y$$
  $$\frac{1 + 3 + 5 + (2y - 1)}{4 + 7 + 10 + \ldots + 3y + 1} = \frac{20}{7\log_{10}x}$$
\item If $a_1, a_2, \ldots, a_n$ are in G.P. and $S = a_1 + a_2 + \ldots + a_n, T = \frac{1}{a_1} + \frac{1}{a_2} + \ldots
  + \frac{1}{a_n}$ and $P = a_1.a_2.\ldots.a_n$ show that $P^2 = \left(\frac{S}{T}\right)^n$.
\item Let $a, b, c$ be respectively the sums of the first $n$ terms, the next $n$ terms and the next $n$ terms of a
  G.P. show that $a, b, c$ are in G.P.
\item If $S_n$ denotes the sum to $n$ terms of a G.P. whose first term and common ratio are $a$ and $r$ respectively, then
  prove that $S_1 + S_2 + \ldots + S_n = \frac{na}{1 - r} - \frac{ar(1 - r^n)}{(1 - r)^2}$
\item If $S_n$ denotes the sum to $n$ terms of a G.P. whose first term and common ratio are $a$ and $r$ respectively, then
  prove that $S_1 + S_3 + S_5 + \ldots + S_{2n - 1} = \frac{na}{1 - r} - \frac{ar(1 - r^{2n})}{(1 - r)^2(1 + r)}$
\item Let $s$ denote the sum of terms of an infinite geometric progression and $\sigma^2$ the sum of squares of the
  terms. Show that the sum of first $n$ terms of this geometric progression is given by $s\left[1 - \left(\frac{s^2 - \sigma^2}{s^2
      + \sigma^2}\right)^n\right],$ where $|r| < 1$.
\item Let $a_1, a_2, a_3, \ldots, a_n$ be a geometric progression with first term $a$ and common ratio $r,$ then the sum
  of the products $a_1, a_2, \ldots, a_n$ taken two at a time i.e. $\sum_{i < j}a_ia_j = \frac{a^2r(1 - r^{n - 1})(1 - r^n)}{(1 -
    r)^2(1 + r)}$.
\item If $a_1, a_2, a_3, \ldots$ is a G.P. with first term $a$ and common ratio $r,$ show that $\frac{1}{a_1^2 - a_2^2} +
  \frac{1}{a_2^2 - a_3^2} + \ldots + \frac{1}{a_{n - 1}^2 - a_n^2}$ $ = \frac{r^2(1 - r^{2n - 2})}{a^2r^{2n - 2}(1 - r^2)^2}$.
\item If $a_1, a_2, a_3, \ldots$ is a G.P. with first term $a$ and common ratio $r,$ show that $\frac{1}{a_1^m + a_2^m} +
  \frac{1}{a_2^m + a_3^m} + \ldots + \frac{1}{a_{n - 1}^m + a_n^m}$ $= \frac{r^{mn - m} - 1}{a^m(1 + r^m)(r^{mn - m} - r^{mn -
      2m})}$.
\item If $a_1, a_2, \ldots, a_{2n}$ are $2n$ positive real numbers which are in G.P. show that $\sqrt{a_1a_2} +
  \sqrt{a_3a_4} + \sqrt{a_5a_6} + \ldots$ $ + \sqrt{a_{2n - 1}a_{2n}} = \sqrt{a_1 + a_3 + \ldots + a_{2n - 1}}\sqrt{a_2 + a_4 +
    \ldots + a_{2n}}$.
\item Find the solution of the system of equations $1 + x + x^2 + \ldots + x^{23} = 0$ and $1 + x + x^2 + \ldots + x^{19}
  = 0$.
\item A man invests $\$a$ at the end of the first year, $\$2a$ at the end of the second year, $\$3a$ at the end of the
  third year, and so on up to the end of $n$th year. If the rate of interest is $\$r$ per rupee and the interest is compounded
  annually, find the amount the man will receive at the end of $(n + 1)$th year.
\item Find the value of $(0.16)^{\log_{2.5}\left(\frac{1}{3} + \frac{1}{3^2} + \frac{1}{3^3} + \ldots \infty\right)}$
\item One side of an equilateral triangle is $24$ cm. The mid-point of its sides are joined to form another triangle,
  whose mid-points are in turn joined for form still another triangle. The process continues indefinitely. Find the sum of
  perimeter of all the triangles.
\item A ball is dropped from a height of $900$ cm. Each time is rebounces, it rises to $2/3$ of the height it has fallen
  through. Find the total distance travelled by the ball before it comes to rest.
\item A square is drawn by joining the mid-points of the sides of a given square. A third square is drawn inside the
  second square in the same way and this process continutes indeinitely. If the sides of the first square is $4$ cm, determine the
  sum of the areas of all the squares.
\item In an increasing G.P., the sum of the first and the last term is $66,$ the product of the second and the last term
  but one term is $128,$ and the sum of all the terms is $126.$ How many terms are there in the progression?
\item The sum of an infinite G.P. is $2$ and the sum of the G.P. made from the cubes of the terms of this infinite series
  is $24.$ Then find the series.
\item The sum of an infinite G.P. is $3$ and the sum of squares of the terms of this series is also $3.$ Find the sequence.
\item If the sum of an infinitely decreasing G.P. is $3.5$ and the sum of the squares of its terms is $147/16.$ Show that
  the sum of the cubes of the terms is $1029/38$.
\item Find the value of $x$ in $]-\pi, \pi[$ which satisfy the equation $8^{1 + |\cos x| + \cos^2 x + |\cos^3 x| + \ldots
      \text{~to~}\infty} = 64$.
\item If $A = 1 + r^a + r^{2a} + \ldots \text{~to~}\infty$ and $B = 1 + r^b + r^{2b} + \ldots \text{~to~}\infty,$ prove
  that $r = \left(\frac{A - 1}{A}\right)^{\frac{1}{a}} = \left(\frac{B - 1}{B}\right)^{\frac{1}{b}}$.
\item If $s_1, s_2, \ldots, s_n$ are the sums of infinite geometric series whose first terms are $1, 2, 3, \ldots, n$ and
  common ratios are $\frac{1}{2}, \frac{1}{3}, \ldots, \frac{1}{n + 1}$ respectively, then prove that $s_1 + s_2 + \ldots + s_n =
  \frac{1}{2}n(n + 3)$.
\item If $S_n$ be the sum of infinite G.P.'s whose first term is $n$ and the common ratio is $\frac{1}{n + 1},$ find
  $\displaystyle\lim_{n\to \infty} \frac{S_1S_n + S_2S_{n - 1} + \ldots + S_nS_1}{S_1^2 + S_2^2 + \ldots + S_n^2}$.
\item The sum of the terms of an infinitely decreasing G.P. is equal to the greatest value of the function $f(x) = x^3 +
  3x - 9$ on the interval $[-5, 3],$ and the difference between the first and second terms is $f'(0).$ Prove that the common ratio
  of the progression is $\frac{2}{3}$.
\item Find the sum of the series $\frac{5}{13} + \frac{55}{13^2} + \frac{555}{13^3} + \ldots \infty$.
\item If $e^{{\sin^2x + \sin^4x + \sin^6x + \ldots \infty}\log_e2}$ satisfies the equation $x^2 - 9x + 8 = 0,$ find the
  value of $\frac{\cos x}{\cos x + \sin x}, 0 < x < \frac{\pi}{2}$.
\item If $-\frac{\pi}{2} < x < \frac{\pi}{2}$ and the sum to infinite number of terms of series $\cos x + \frac{2}{3}\cos
  x\sin^2x + \frac{4}{9}\cos x\sin^4x + \ldots$ is finite, then show that $x$ lies in the set $\left(-\frac{\pi}{2},
  \frac{\pi}{2}\right)$
\item Suppose $0 < x < \pi$ and the expression $e^{{1 + |\cos x| + \cos^2x + |\cos^3x| + \ldots \infty}\log_e 4}$
  satisfies the quadratic equation $y^2 - 20y + 64 = 0,$ then find the value of $x$.
\item An A.P. and a G.P. with positive terms have the same number of terms and their first terms as well as the last
  terms are equal. Show that the sum of A.P. is greater than or equal to the sum of the G.P.
\item Given a G.P. and A.P. of positive terms $a, a_1, a_2, \ldots, a_n, \ldots$ and $b, b_1, b_2, \ldots, b_n, \ldots$
  respectively, with the common ratio of the G.P. being different from $1,$ prove that there exists $x\in R, x > 0$ such that
  $\log_x a_n - b_n = \log_x a - b,~\forall n\in N$.
\item If the $(m + 1)$th, $(n + 1)$th and $(r + 1)$th terms of an A.P. are in G.P., and $m, n, r$ are in H.P., show that
  the ratio of the first term to the common difference of the A.P. is $-n/2$.
\item If $a, b, c$ are in G.P. and $a - b, c - a, b - c$ are in H.P., then show that $a + 4b + c = 0$.
\item If $S_1, S_2$ and $S_3$ denote the sum to $n(> 1)$ terms of three sequences in A.P., whose first terms are unity
  and common differences are in H.P., prove that $n = \frac{2S_3S_1 - S_1S_2 - S_2S_3}{S_1 - 2S_2 + S_3}$
\item Find a three-digit number such that its digits are in G.P. and the digits of the number obtained from it by
  subtracting $400$ form an A.P.
\item If $a, b, c$ be distinct positive numbers in G.P. and $\log_c a, \log_b c, \log_a b$ be in A.P., prove that the
  common difference of the progression is $3/2$.
\item If $p$ be the first of the $n$ arithmetic means between two numbers $a$ and $b$ and $q$ the first of the $n$
  harmonic means between the same two numbers, prove that the value of $q$ cannot lie between $p$ and $\left(\frac{n + 1}{n -
    1}\right)^2p$.
\item Find a three digit number whose consecutive numbers form a G.P. If we subtract $792$ from this number, we get a
  number consisting of the same digits written in the reverse order. Now if we increase the second digit of the required number by
  $2,$ the resulting number will form an A.P.
\item An A.P. and a G.P. each has $p$ as first term and $q$ as second term where $0 < q < p.$ Find the sum to infinity,
  $s$ of the G.P., and prove that the sum of first $n$ terms of the A.P. may be written as $np - \frac{n(n - 1)}{2}.\frac{p^2}{s}$.
\item If $\log_xy, \log_zx, \log_yz$ are in G.P., $xyz = 64$ and $x^3, y^3, z^3$ are in A.P., then find $x, y$ and $z.$
\item Find all complex numbers $x$ and $y$ such that $x, x + 2y, 2x + y$ are in A.P. and $(y + 1)^2, xy + 5, (x + 1)^2$
  are in G.P.
\item Find A.P. of distinct terms whose first term is $3$ and second, tenth and thirty fourth terms form a G.P.
\item Let $a, b, c, d$ be four positive real numbers such that the geometric mean of $a$ and $b$ is equal to the
  gerometric mean of $c$ and $d$ and the arithmetic mean of $a^2$ and $b^2$ is equal to the arithmetic mean of $c^2$ and $d^2.$
  Show that the arithmetic mean of $a^n$ and $b^n$ is equal to the arithmetic mean of $c^n$ and $d^n$ for every integral value of
  $n$.
\item The sum of first ten terms of an A.P. is equal to $155,$ and the sum of first two terms of a G.P. is $9.$ Find
  these progressions if the first term of the A.P. euqals the common ratio of the G.P. and the first term of G.P. equals the common
  difference of A.P.
\item If $a, b, c$ be in H.P., prove that $\left(\frac{1}{a} + \frac{1}{b} - \frac{1}{c}\right)\left(\frac{1}{b} +
  \frac{1}{c} - \frac{1}{a}\right) = \frac{4}{ac} - \frac{3}{b^2}$.
\item If $a, b, c$ are positive real numbers which are in H.P. show that $\frac{a + b}{2a - b} + \frac{b + c}{2c - b}\geq
  4$.
\item If $(a + b)/(1 - ab), b, (b + c)/(1 - bc)$ are in A.P., then prove that $a, b^{-1}, c$ are in H.P.
\item Suppose $a, b, c$ are in A.P. and $|a|, |b|, |c| < 1$ if $x = 1 + a + a^2 + \ldots \infty, y = 1 + b + b^2 + \ldots
  \infty,$ $z = 1 + c + c^2 + \ldots \infty$ then prove that $x, y, z$ are in H.P.
\item If $a^{\frac{1}{x}} = b^{\frac{1}{y}} = c^{\frac{1}{z}}$ and $a, b, c$ are in G.P. prove that $x, y, z$ are in A.P.
\item If $a, b, c$ be in A.P., $l, m, n$ be in H.P. and $al, bm, cn$ be in G.P. with common ratio not equal to $1$ and
  $a, b, c, l, m, n$ are positive show that $a:b:c = \frac{1}{n}:\frac{1}{m}:\frac{1}{l}$.
\item Find three numbers $a, b, c$ between $2$ and $18$ such that their sum is $25$, the numbers $2, a, b$ are
  consecutive terms of an A.P. and the numbers $b, c, 18$ are consecutive terms of a G.P.
\item If $a, b, c$ are in A.P. and $a, mb, c$ are in G.P.; prove that $a, m^2b, c$ are in H.P.
\item An A.P., a G.P. and an H.P. have the same first term $a$ abd same second term $b,$ show that $n + 2$th terms will
  be in G.P. is $\frac{b^{2n + 2} - a^{2n + 2}}{ab(b^{2n} - a^{2n})} = \frac{n + 1}{n}$.
\item The second, third and sixth terms of an A.P. are consecutive terms of a geometric progression. Find the common
  ratio of the G.P.
\item A sequence of numbers is formed by adding together corresponding terms of an A.P. and a G.P. with common ratio $2.$
  The first term of the sequence is $57,$ the second term is $94$ and the third term is $171.$ Find the fourth term. Find also an
  expression for the $n$th term of the sequence.
\item The first, eighth and twenty second terms of an arithmetic progression are three consecutive terms of a geometric
  progression. Find the common ratio of the geometric progression. If sum of the first twenty two terms of arithmetic progression
  is $275,$ find its first term.
\item An arithmetic progression has common difference $2$ and a geometric progression has common ratio $2.$ A new
  sequence is formed by adding together the corresponding terms of these progressions. Given that the first term of this new
  sequence is $8$ and the fifth term is $91,$ find the first terms.
\item If $a, b, c$ are in A.P. and $b, c, d$ are in H.P., prove that $ad = bc$.
\item If $a, b, c$ are in H.P., $b, c, d$ are in G.P. and $c, d, e$ are in A.P., show that $e = \frac{ab^2}{(2a - b)^2}$.
\item If an A.P. and a G.P. have the same 1st and 2nd terms then show that every other term of the A.P. will be less than
  the corresponding term of G.P. all the terms being positive.
\item If three unequal numbers are in H.P. and their squares are in A.P. show that they are in the ratio $1 +
  \sqrt{3}:-2:1 - \sqrt{3}$ or $1 - \sqrt{3}:-2:1 + \sqrt{3}$.
\item If $A, G, H$ are the arithmetic, geometric and harmonic means of two positive real numbers $a$ and $b$, and if $A =
  kh,$ prove that $A^2 = kG^2.$ Find the ratio of $a$ to $b.$ For what value of $k$ does the ratio exist.
\item If $p$ be the $r$th term when $n$ A.M.'s are inserted between $a$ and $b$ and $q$ be the $r$th term when $n$ H.M.'s
  are inserted between $a$ and $b,$ then show that $\frac{p}{a} + \frac{b}{q}$ is independent of $n$ and $r$.
\item Two trains $A$ and $B$ start from the same station $P$ at the same time. $A$ covers half the distance between first
  station $P$ and second station $Q$ with speed $x$ and other half distance with speed $y$. Train $B$ covers the whole distance
  with speed $\frac{x + y}{2}.$ Which train will reach $Q$ earlier.
\item If $n$ is a root of equation $x^2(1 - ac) - x(a^2+c^2) - (1 + ac) = 0$ and if $n$ H.M.'s are inserted between $a$
  and $c,$ show that the difference between the first and last mena is equal to $ac(a - c)$.
\item If $A_1, A_2, \ldots, A_n$ are the $n$ A.M.'s and $H_1, H_2, \ldots, H_n$ the $n$ H.M.'s between $a$ and $b,$ show
  that $A_rH_{n - r + 1} = ab$ for $1\leq r\leq n$.
\item Find the coefficient of $x^{99}$ and $x^{98}$ in the polynomial $(x - 1)(x - 2)(x - 3)\ldots(x - 100)$.
\item Find the $n$th term and sum to $n$ terms of the series $12, 40, 90, 168, 280, 432, \ldots$
\item Find the $n$th term and the sum to $n$ terms of the series $10, 23, 60, 169, 494, \ldots$.
\item Find the sum of the series $3 + 5x + 9x^2 + 15x^3 + 23x^4 + 33x^5 + \ldots \infty$.
\item If $H_n = 1 + \frac{1}{2} + \frac{1}{3} + \ldots + \frac{1}{n}$ and $H_n' = \frac{n + 1}{2} - \left\{\frac{1}{n(n - 1)}
  + \frac{2}{(n - 1)(n - 2)} + \ldots + \frac{n - 2}{2.3}\right\},$ show that $H_n = H_n'$.
\item Show that $\tan^{-1}\left(\frac{x}{1 + 1.2x^2}\right) + \tan^{-1}\left(\frac{x}{1 + 2.3x^2}\right) + \ldots+
  \tan^{-1}\left(\frac{x}{1 + n(n + 1)x^2}\right) = \tan^{-1}\left(\frac{nx}{1 + (n + 1)x^2}\right)$.
\item Find the sum to $n$ terms of the series $\frac{1}{1 + 1^2 + 1^4} + \frac{2}{1 + 2^2 + 2^4} + \frac{3}{1 + 3^2 +
  3^4} + \ldots$.
\item Find $\displaystyle\sum_{k=n}^n\tan^{-1}\frac{2k}{2 + k^2 + k^4}$
\item Show that $\frac{1^4}{1.3} + \frac{2^4}{3.5} + \frac{3^4}{5.7} + \ldots + \frac{n^4}{(2n - 1)(2n + 1)} =
  \frac{n(4n^2 + 6n + 5)}{48} + \frac{n}{16(2n + 1)}$
\item If $a_1, a_2, \ldots, a_n, \ldots$ are in A.P. with first term $a$ and common difference $d,$ find the sum for $r >
  1$ of $a_1a_2\ldots a_r + a_2a_3\ldots a_{r + 1} + \ldots \text{~to~}n\text{~terms}$.
\item If $a_1, a_2, \ldots, a_n, \ldots$ are in A.P. and none of them is zero. Then prove that $\frac{1}{a_1a_2\ldots
    a_r} + \frac{1}{a_2a_3\ldots a_{r + 1}} + \ldots + \frac{1}{a_na_{n + 1}\ldots a_{n + r - 1}} = \frac{1}{(r - 1)(a_2 -
  a_1)}$

  $\left[\frac{1}{a_1a_2\ldots a_{r - 1}} - \frac{1}{a_{n + 1}a_{n + 2}\ldots a_{n + r - 1}}\right]$
\item Find the sum to $n$ terms of the series $\frac{1}{1.2.3.4} + \frac{1}{2.3.4.5} + \frac{1}{3.4.5.6} + \ldots$.
\item Find the sum to $n$ terms of the series $\frac{3}{2.4.6} + \frac{4}{2.3.5} + \frac{5}{3.4.6} + \ldots$.
\item Find $\frac{1}{1.3} + \frac{2}{1.3.5} + \frac{3}{1.3.5.7} + \ldots \text{~to~}n\text{~terms}$.
\item Find $\frac{2}{1.3}.\frac{1}{3} + \frac{3}{3.5}.\frac{1}{3^2} + \frac{4}{5.7}.\frac{1}{3^3} + \ldots
  \text{~to~}n\text{~terms}$.
\item Find the sum of $n$ terms of the series $\frac{1}{3} + \frac{3}{3.7} + \frac{5}{3.7.11} + \frac{7}{3.7.11.15} +
  \ldots$.
\item Find the sum of the series: $1 + 2(1 - a) + 3(1 - a)(1 - 2a) + 4(1 - a)(1 -2a)(1 - 3a) +
  \ldots\text{~to~}m\text{~terms}$.
\item Find the sum of the series $1 + \frac{x}{b_1} + \frac{x(x + b_1)}{b_1b_2} + \frac{x(x + b_1)(x + b_2)}{b_1b_2b_3} +
  \ldots + \frac{x(x + b_1)\ldots(x + b_{n - 1})}{b_1b_2\ldots b_n}$.
\item Let $S_k(n) = 1^k + 2^k + \ldots + n^k,$ show that $nS_k(n) = S_{k+1}(n) + S_k(n - 1) + S_k(n - 2) + \ldots +
  S_k(2) + S_k(1)$.
\item Find the sum of all the numbers of the form $n^3$ which lie between $100$ and $10000$.
\item If $S$ be the sum of the $n$ consecutive integers beginning with $a$ and $t$ the sum of their squares, show that
  $nt - S^2$ is independent of $a$.
\item If $\sum_{x = 5}^{n + 5}4(x - 3) = Pn^2 + Qn + R,$ find the value of $P + Q.$
\item Find the sum to $2n$ terms of the series $5^3 + 4.6^3 + 7^3 + 4.8^3 + 9^3 + 4.10^3 + \ldots$.
\item Find the sum to $n$ terms of the series $\left(\frac{2n + 1}{2n - 1}\right) + 3\left(\frac{2n + 1}{2n - 1}\right)^2
  + 5\left(\frac{2n + 1}{2n - 1}\right)^3 + \ldots$.
\item Find the sum to $n$ terms of the series $1 + 5\left(\frac{4n + 1}{4n - 3}\right) + 9\left(\frac{4n + 1}{4n -
  3}\right)^2 + 13\left(\frac{4n + 1}{4n - 3}\right)^3 + \ldots$.
\item Prove that the numbers of the sequence $121, 12321, 1234321, \ldots$ are each a perfect square of an odd integer.
\item Prove that the sum to $n$ terms of the series $\frac{3}{1^2} + \frac{5}{1^2 + 2^2} + \frac{7}{1^2 + 2^2 + 3^2} +
  \frac{9}{1^2 + 2^2 + 3^2 + 4^2} + \ldots$ is $6n/(n + 1)$.
\item Find the sum to $n$ terms of the series $\frac{1}{(1 + x)(1 + 2x)} + \frac{1}{(1 + 2x)(1 + 3x)} + \frac{1}{(1 +
  3x)(1 + 4x)} + \ldots$.
\item Find the sum to $n$ terms of the series $\frac{1}{(1 + x)(1 + ax)} + \frac{a}{(1 + ax)(1 + a^2x)} + \frac{1}{(1 +
  a^2x)(1 + a^3x)} + \ldots$.
\item Find the $n$th term of the series $-1, -3, 3, 23, 63, 129, \ldots$.
\item Find the sum to $n$ terms of the series $\frac{1}{\sqrt{1} + \sqrt{3}} + \frac{1}{\sqrt{3} + \sqrt{5}} +
  \frac{1}{\sqrt{5} + \sqrt{7}} + \ldots$.
\item If $a_1, a_2, \ldots, a_n, \ldots$ are in A.P. with first term $a$ and common difference $d,$ then prove that
  $a_1a_2 + a_2a_3$ $+ \ldots + a_na_{n +1} = \frac{[a + (n - 1)d](a + nd) - (a - d)a(a + d)}{3d} = \frac{n}{3}[3a^2 + 2and + (n^2
  - 1)d^2]$.
\item If $a_1, a_2, \ldots, a_n, \ldots$ are in A.P. with first term $a$ and common difference $d,$ then prove that
  $a_1a_2a_3 + a_2a_3a_4$ $+ \ldots + a_na_{n +1}a_{n + 2}$ $=$

  $\frac{[a + (n - 1)d](a + nd)[a + (n + 1)d][a + (n + 2)d] - (a -
  d)a(a + d)(a + 2d)}{4d}=$

  $\frac{n}{4}[4a^3 + 6(n + 1)a^2d + 2(2n^2 + 3n - 1)ad^2 + (n^3 - 2n^2 - n - 2)d^3]$.
\item Find the sum to $n$ terms of the series $\frac{3}{1^2.2^2} + \frac{5}{2^2.3^2} + \frac{7}{3^2.4^2} + \ldots$.
\item Let $S_n$ denote the sum to $n$ terms of the series $1.2 + 2.3 + 3.4 + \ldots$ and $\sigma_{n - 1}$ that to $n - 1$
  terms of the series $\frac{1}{1.2.3.4} + \frac{1}{2.3.4.5} + \frac{1}{3.4.5.6} + \ldots$ Then prove that
  $\displaystyle18S_n\sigma_{n - 1} - S_n = -2$.
\item Find $\frac{5}{1.2}.\frac{1}{3} + \frac{7}{2.3}.\frac{1}{3^2} + \frac{9}{3.4}.\frac{1}{3^3} +
  \ldots\text{~to~}n\text{~terms}$.
\item If $\frac{1}{1^2} + \frac{1}{2^2} + \frac{1}{3^2} + \frac{1}{4^2} + \ldots\infty = \frac{\pi^2}{6}$ then find
  $\frac{1}{1^2} + \frac{1}{3^2} + \frac{1}{5^2} + \ldots \infty$.
\item If $\frac{1}{1^2} + \frac{1}{2^2} + \frac{1}{3^2} + \frac{1}{4^2} + \ldots \infty = \frac{\pi^2}{6},$ then find $1
  - \frac{1}{2^2} + \frac{1}{3^2} - \frac{1}{4^2} + \ldots \infty$.
\item If $H_n = 1 + \frac{1}{2} + \frac{1}{3} + \ldots + \frac{1}{n},$ then prove that $H_n = n - \left(\frac{1}{2} +
  \frac{2}{3} + \frac{3}{4} + \ldots + \frac{n - 1}{n}\right)$.
\item Show that $\frac{1}{x + 1} + \frac{2}{x^2 + 1} + \frac{4}{x^4 + 1} + \ldots + \frac{2^n}{x^{2^{n}} + 1} =
  \frac{1}{x + 1} - \frac{2^{n + 1}}{x^{2^{n + 1}} -1}$.
\item Show that $\left(1 + \frac{1}{3}\right)\left(1 + \frac{1}{3^2}\right)\left(1 + \frac{1}{3^4}\right) \ldots \left(1
  + \frac{1}{3^{2^{n}}}\right) = \frac{3}{2}\left(1 - \frac{1}{3^{2^{n + 1}}}\right)$.
\item If $x + y + z = 1$ and $x, y, z$ are positive numbers show that $(1 - x)(1 - y)(1 - z)\geq 8xyz$.
\item If $a>0, b > 0$ and $c> 0,$ prove that $(a + b + c)\left(\frac{1}{a} + \frac{1}{b} + \frac{1}{c}\right)\geq 9$.
\item If $a + b + c = 3$ and $a>0, b>0, c>0,$ find the greatest value of $a^2b^3c^2$.
\item Let $a_i + b_i = 1(i = 1, 2, \ldots, n)$ and $a = \frac{1}{n}(a_1 + a_2 + \ldots + a_n), b=\frac{1}{n}(b_1 + b_2 +
  \ldots + b_n),$ show that $a_1b_1 + a_2b_2 + \ldots + a_nb_n = nab - (a_1 - a)^2 - (a_2 - a)^2 - \ldots - (a_n - a)^2$.
\item A sequence $a_1, a_2, a_3, \ldots, a_n$ of real numbers is such that $a_1=0, |a_2| = |a_1 + 1|, |a_3| = |a_2 + 1|,
  \ldots,$ $|a_n| = |a_{n - 1} + 1|.$ Prove that the arithmetic mean $(a_1 + a_2 + \ldots + a_n)/n$ of these numbers cannot be less
  than $-1/2$.
\item If $a,b,c > 0,$ show that $(a + b)(b + c)(a + c) \geq 8abc$.
\item If $x + y + z = a,$ show that $\frac{1}{x} + \frac{1}{y} + \frac{1}{z}\geq \frac{9}{a}$.
\item If $n$ is a positive integer, show that $n^n \geq 1.3.5\ldots (2n - 1)$.
\item Find the greatest value of $(7 - x)^4(2 + x)^5$ if $-2<x<7$.
\item If $a, b, c > 0,$ show that $\frac{bc}{b + c} + \frac{ca}{c + a} + \frac{ab}{a + b}\leq \frac{a + b + c}{2}$.
\item If $a, b, c > 0,$ show that $\frac{b + c}{a} + \frac{c + a}{b} + \frac{a + b}{c}\geq 6$.
\item If $x_i > 0, i = 1, 2, 3, \ldots, n$ show that $(x_1 + x_2 + \ldots + x_n)\left(\frac{1}{x_1} + \frac{1}{x_2} +
  \ldots + \frac{1}{x_n}\right) \geq n^2$.
\item If $x,y$ are positive real numbers and $m, n$ are positive integers, then show that $\frac{x^ny^m}{(1 + x^{2n})(1 +
  y^{2m})}\leq \frac{1}{4}$.
\item If the arithmetic mean of $(b - c)^2, (c - a)^2$ and $(a - b)^2$ is the same as that of $(b + c - 2a)^2, (c + a -
  2b)^2$ and $(a + b - 2c)^2$, show that $a = b = c$.
\end{enumerate}
