\chapter{Progressions}
There are three different progressions: arithmetic progression, geometric progression and harmonic progression. We start this
chapter with arithmetic progression or A.P.

\section{Arithmetic Progressions}
Consider sequences like $1, 2, 3, 4, \ldots$ or $-1, -2, -3, -4, \ldots$ or $1, 3, 5, 7, \ldots$ or $a, a + d, a + 2d, \ldots$

These sequences increase or decrease with a common difference. When quantities increase or decrease with a common difference they
are said to be in \textit{Arithmetic Progression}. The \textit{common difference} can be found by subtracting any term of the
series that follows it. For example for the first series it is $1$ and for the last it is $d$.

Consider the series $a, a + d, a + 2d, a + 3d, \ldots$

Simple observation tells us that $1$st term is $a$, $2$nd term is $a + d$, the $3$rd term is $a + 2d$ and hence the $n$th term will
be $a + (n - 1)d$. These terms are typically written as $t_1, t_2, t_3, \ldots, t_n$.

\subsection{$n$th Term of Arithmetic Progression}
Following above discussion, we can clearly say that the $n$th term of an arithmetic progression is given by $t_n = a + (n - 1)d$,
where $a$ is called the first term and $d$ the common difference.

\subsection{Sum of an Arithmetic Progression}
Let $S_n$ represent the sum of first $n$ terms of an arithmetic progression, then we can write.

$$S_n = a + (a + d) + (a + 2d) + \ldots + [a + (n - 2)d] + [a + (n - 1)d]$$
Writing the terms in reverse order we have
$$S_n = [a + (n - 1)s] + [a + (n - 2)d] + \ldots + (a + d) + a$$
Adding term by term, we get
$$2S_n = [2a + (n - 1)d] + [2a + (n - 1)d] + \ldots\text{~to~}n\text{~terms~}$$
$$2S_n = n[2a + (n - 1)d]\Rightarrow S_n = \frac{n}{2}[2a + (n - 1)d]$$

We also see that $S_n = \frac{n}{2}(t_1 + t_n)$

We also see that if a series is $1 + 2 + 3 + \ldots + n = \sum_{i=0}^ni = \frac{n(n + 1)}{2}$.

\subsection{Arithmetic Mean}
When three quantities are in arithmetic progression the quantity in the middle is known to be arithmetic mean of the other two. For
example, if $a, b, c$ are in A.P., then $b$ is said to be arithmetic mean of $a$ and $c$. In general, it is written $b = \frac{a +
  c}{2}$. This can be examined further. Let $b = a + d$, then $c = a + 2d$. Clearly, $b = \frac{a + c}{2}$.

It is also possible to insert $n$ numbers between any two numbers such that all of them are in A.P. Consider two numbers $a$ and
$b$ in between which we want to insert $n$ numbers such that they are in A.P. Clearly, $b$ will become $n +2$th term of A.P. Let
common difference be $d$ then we can write $b = a + (n + 1)d \Rightarrow d = \frac{b - a}{n + 1}$. Now all the $n$ arithmetic means
can be deduced. Let those be $m1, m2, \ldots, m_n$ then $m_1 = a + \frac{b - a}{n + 1}, m_2 = a + \frac{2(b - a)}{n + 1}, \ldots,
m_n = a + \frac{n(b - a)}{n + 1}$.

Suppose there are $n$ terms of an A.P., then the arithmetic mean of those $n$ terms is given by $\frac{t_1 + t_2 + \ldots +
  t_n}{n}$.

\subsection{Deducing Number of Terms}
We know that $S_n = \frac{n}{2}[2a + (n - 1)d]$. Say $S_n, a$ and $d$ are known and we have to evaluate $n$. This being a quadratic
equaion will have two roots for $n$. If the results are positive and integral then there is no problem in interpreting the
results. In some cases for a negative root a suitable interpretation can be given.

\textbf{Example:} How many terms of the series $-8, -6, -4, \ldots$ must be added for the sum to be $36$?

$\frac{n}{2}[-16 + (n - 1)2] = 36\Rightarrow n^2 - 9n - 36 = 0 \Rightarrow n = 12, -3$

If we take $12$ terms of the series, we have $-8, -6, -4, -2, 0, 2, 4, 6, 8, 10, 12, 14$. The sum of these terms is $36$ and sum of
last three terms is also $36$ which is represented by $n = -3$.

\subsection{Properties of an A.P.}
\begin{enumerate}
\item If a fixed number is added to or subtracted from each item of a given A.P., then the resulting is also an A.P., and it has
  the same common difference as that of the given A.P.
\item If each term of an A.P. is multiplied or divided by a non-zero fixed constant then the resulting sequence is also an A.P. The
  common difference is multiplied or divided by the same factor.
\item If $a_1, a_2, a_3, \ldots$ and $b_1, b_2, b_3, \ldots$ are two arithmetic progressions then $a_1 + b_1, a_2 + b_2, a_3 + b_3,
  \ldots$ are also in A.P.
\item If we have to choose three unknown terms in an A.P. then it is best to choose them as $a - d, a, a + d$.
\item If we have to choose four unknown terms in an A.P. then it is best to choose them as $a - 3d, a - d, a + d, a + 3d$.
\item In an A.P., the sum of terms equidistant from the beginning and end is constant and is equal to the sum of first and last
  term.
\item Any term of an A.P., except the first, is equal to half the sum of terms which are equidistant from it:
  $$a_n = \frac{1}{2}(a_{n - k} + a_{n + k}), ~k<n,\text{~and for~} k = 1$$
  $$a_n = \frac{1}{2}(a_{n - 1} + a_{n + 1})$$
\item $t_n = S_n - S_{n - 1}, n\geq 2$
\item If $t_n = pn + q$ i.e. a linear expression in $n$ then it will form an A.P. of common difference $p = t_n - t_{n - 1}$ and
  first term $p + q$. For example, if $t_n = 3n + 4$, then it is an A.P. of common difference $3$ anda the first term as $7$.
\item If $S_n = an^2 + bn + c$ i.e. a quadratic function in $n$, then the series in an A.P. where $a = 2a,$ twice the coefficient
  of $n^2$.
\end{enumerate}

\subsection{Sum of Squares and Cubes and More}
We observe that
$$i^3 - (i - 1)^3 = 3i^3 - 3i + 1 \Rightarrow \sum_{i = 1}^n[i^3 - (i - 1)^3] = 3\sum_{i = 0}^ni^2 - \frac{3n(n + 1)}{2} + n$$
$$n^3 = 3\sum_{i = 0}^ni^2 - \frac{3n(n + 1)}{2} + n \Rightarrow 3\sum_{i=0}^ni^2 = n^3 + \frac{3n(n + 1)}{2} - n$$
$$\sum_{i=0}^ni^2 = \frac{n(n + 1)(2n + 1)}{6}$$

Following in a similar fashion, we can show that

$$\sum_{i=0}^n = \left\{\frac{n(n + 1)}{2}\right\}^2$$

More powers can be evaluated in a similar fashion.

\section{Geometric Progressions}
A succession of numbers is said to be in geometric progressions or geometric sequence if the ratio of any term and the term
preceeding it is constant throughout. This constant is called \textit{common ratio} of the G.P.

Example: $1, 2, 4, 8, 16, \ldots$

Here, $\frac{t_2}{t_1} = \frac{t_3}{t_2} = \ldots = 2$.

Also, $1, 3, 9, 27,\ldots$ are in geometric progression whose first term is $1$ and common ratio is $3$.

Also, $2, -4, 8, -16, \ldots$ are in geometric progression whose firts term is $2$ and common ratio is $-2$.

\subsection{Properties of a G.P.}
\begin{enumerate}
\item If the each term of a G.P. be multiplied by a non-zero number, then the sequence obtained is also a G.P.

  \textbf{Proof:} Let the given G.P. be $a, ar, ar^2, ar^3, \ldots$

  Let $k$ be a non-zero number, the sequence obtained by multiplying each term of the given G.P. by $k$ is $ak, ark, ar^2k, ar^3k,
  \ldots$

  Clearly, the series is in G.P. with the same common ratio as previous ratio i.e. $r$.

  Again, dividing each term of G.P. $a, ar, ar^2, a^3, \ldots$ we obtain the sequence $\frac{a}{k}, \frac{ar}{k}, \frac{ar^2}{k},
  \ldots$

  It is clear that this new sequence is also a G.P., whose common ratio is $r$.
\item The reciprocals of the terms of a G.P. are also in G.P.

  \textbf{Proof:} Let the G.P. be $a, ar, ar^2, \ldots$, the sequence whose terms are reciprocals of this G.P. is $\frac{1}{a},
  \frac{1}{ar}, \frac{1}{ar^2}, \ldots$

  It is clear that this sequence is in G.P., whose first term is $\frac{1}{a}$ and common ratio is $\frac{1}{r}$.
\end{enumerate}

\subsection{Sum of the First $n$ Terms of a G.P.}
Let $a$ be the first term and $r$ be the common ratio of a G.P. and $S_n$ be the sum of its first $n$ terms

\textbf{Case I:} When $r\neq 1$
$$S_n = a + ar + ar^2 + \ldots + ar^{n - 2} + ar^{n - 1}$$
$$rS_n = ar + ar^2 + \ldots + ar^{n - 1} + ar^n$$

Subtracting, we get $(1 - r)S_n = a - ar^n = a(1 - r^n)$

$\therefore S_n = \frac{a(1 - r^n)}{1 - r} = \frac{a(r^n - 1)}{r - 1}$

\textbf{Case II:} When $r = 1$

$S_n = a + a + \ldots + a = na$ and this G.P. is also an A.P. whose common difference is $0$.

\subsection{Sum of Infinite Terms of a G.P.}
If $|r|\geq 1$ then sum would be $\pm\infty$. However, if $|r|< 1$ then sum would be finite.

We have obtained that $S_n = \frac{a(1 - r^n)}{1 - r}$

We see that as $n$ approaches $\infty, r^n$ will approach $0$. Thus, $S_\infty = \frac{a}{1 - r}$

\subsection{Recurring Decimals}
Recurring decimals are a very interesting and nice example to demonstrate the infinite G. P. and the value can be obtained by the
formula derived in previous section. Consider a recurring decimal $\dot{7}$.

$$.\dot{7} = .777777 ... \text{to }\infty$$
$$= .7 + .07 + .007 + .0007 + \ldots$$
$$= \frac{7}{10} + \frac{7}{100} + \frac{7}{1000} + \ldots$$
$$= \frac{7}{10} + \frac{7}{10^2} + \frac{7}{10^3} + \ldots$$
$$= 7\left(\frac{1}{10} + \frac{1}{10^2} + \frac{1}{10^3} + \ldots\right)$$
$$= \frac{7}{9}$$

\subsection{Geometric Mean}
