\chapter{Preface}
This is a book on algebra, which, covers basics of algebra till high school level. It covers the most essential topics to take up a
bachelor's course where knowledge of algebra is required. There is no specific purpose for writing this book. This is a book for
self study and is not recommended for courses in schools and universities. I will try to cover as much as I can and will keep
adding new material over a long period. I have no interest in writing a book in a fixedly way which serves a university or college
course as I have always loved freedom. Life, freedom and honor in that order are important.

Algebra is probably one of the most fundamental subjects in Mathematics as further study of subjects like trigonometry, coordinate
geometry and rest all depend on it. That is the primary reason I have chosen it to be the first subject in mathematics to be dealt
with. It is very important to understand algebra for the readers if they want to advance further in mathematics.

\section*{Why I Wrote This Book?}
I wrote this book for myself! I did not write it for anybody else. My knowledge, which I have acquired by reading books written by
many great mathematicians and authors and interactions with many intelligent people, is what has been put in the book. I have just
tried to add my flavor to it. Think of it as notes for me. Just that I like to organize my notes so it has taken form of a book and
nothing more. If you benefit from this then that is a pure coincidence and not intentional at all.

\section*{How to Read This Book?}
No I will not simply tell that you must solve the problems. My advice would be more detailed. Every chapter will have theory. Read
that first. Make sure you understand that. Of course, you have to meet the prerequisites for the book. Then, go on and try to solve
the problems. In this book, there are no pure problems. Almost all have answers except those which are of similar kind and
repetitive in nature for the sake of practice. If you can solve the problem then all good else look at the answer and try to
understand that. Then, few days later take on the problem again. If you fail to understand the answer you can always email me with
your work and I will try to answer to the best of my ability. However, if you have a local expert seek his/her advice first. Just
that email is bad for mathematics.

Note that mathematics is not only about solving problems. If you understand the theory well, then you will be able to solve
problems easily. However, problems do help enforce with the enforcement of theory in your mind.

\section*{Who Should Read This Book?}
Since this book is written for self study anyone with interest in algebra can read it. That does not mean that school or college
students cannot read it. You need to be selective as to what you need for your particular requirements. This is mostly high school
course with a little bit of lower classes' course thrown in with a bit of detail here and there.

\section*{Prerequisite}
You should have knowledge till grade 10th course. Attempt has been made to keep it simple and give as much as background to the
topic which is reasonable and required. However, not everything will be covered below grade 10.

\section*{Goals for Readers}
The goal of for reading this book is becoming proficient in solving simple and basic problems of algebra. Another goal would be to
be able to study other subjects which require this knowledge like trigonometry or calculus or physics or chemistry or other
subjects. If you can solve 95\% problems after 2 years of reading this book then you have achieved this goal.

All of us possess a certain level of intelligence. At average any person can read this book. But what is most important is you have
to have interest in the subject. Your interest gets multiplied with your intelligence and thus you will be more capable than you
think you can be. One more point is focus and effort. It is not something new which I am telling but I am saying it again just to
emphasize the point. Trust me if you are reading this book for just scoring a nice grade in your course then I have failed in my
purpose of explaining my ideas.

\section*{Acknowledgements}
I am in great debt of my family and free software community because both of
these groups have been integral part of my life. Family has prvided direct
support while free software community has provided the freedom and freed me
from the slavery which comes as a package with commercial software. I am
especially grateful to my wife, son and parents because it is their time which
I have borrowed to put in the book. To pay my thanks from free software
community  I will take one name and that is Richard Stallman who started all
this  and is still fighting this never-ending war. When I was doing the Algebra
book then I realized how difficult it is to put Math on web in HTML format and
why Donald Knuth wrote \TeX{}. Also, \TeX{} was one of the first softwares to
be released as a free software. HTML has not yet matured to represent
mathematical content. MathML is supported by Mozilla. Webskit browsers started
supporting  then dropped and I would refrain to comment on commercial browser's
support which do not even comply to starndard because they think they are the
standards tehmselves and others will bend to their will.

Now as this book is being written using \LaTeX{}~ so obviously Leslie Lamport
and all the people involved with it have my thanks along with Donald Knuth. I
use Emacs with Auctex and hope that someday I will use it in a much more
productive way someday.

I have used Asymptote as a tool for drawing all the diagrams. It is a wonderful
package and works very nicely.

I would like to thank my parents, wife and son for taking out their fair share
of time and the support which they have extended to me during my bad
times. After that I would like to pay my most sincere gratitude to my teachers
particularly H. N. Singh, Yogendra Yadav, Satyanand Satyarthi, Kumar Shailesh
and Prof. T. K. Basu. Now is the turn of people from software community. I must
thank the entire free software community for all the resources they have
developed to make computing better. However, few names I know and here they
go. Richard Stallman is the first, Donald Knuth, Dennis Ritchie, Ken Thompson,
Bjarne Stroustrup after that.

I am not a native English speaker and this book has just gone through one pair
of eyes therefore chances are high that it will have lots of errors. At the
same time it may contain lots of technical errors. Please feel free to drop me
an email at
\href{mailto:shivshankar.dayal@gmail.com}{shivshankar.dayal@gmail.com} where I
will try to respond to each mail as
much as possible. Please use your real names in email not something like
coolguy.
\begin{flushright}
Shiv Shankar Dayal\\
Nalanda,\\
India, 2015
\end{flushright}
